\part*{Why Axiomization}
Before we enter the world of limits, we will axiomize the real numbers. We will start from as few as possible  
axioms to construct the natural numbers, integers, rational, and finally the real numbers. And along the way, 
basic operations and relations such as addition, multiplication, and order (i.e. $<,>,=$) will be defined, with 
their properties verified (e.g. The commutative law).

You might wonder the reason for all these long and tedious works. The laws of algebra and the existence of the 
reals have accompanied us for many years and we are very familiar with them now. Why should we verify all of 
them? Can't we just take them for granted and go ahead to limits? Well, here comes the very meaning of 
axiomization.

Now looking back, for these properties, we were just taught them. We have been using them without doubting their 
correctness. And to doubt it seems very funny. However, we must admit that we are merely thinking that they are 
right. Mathematics is a rigorous subject. It does not accept such things as ``I believe that it is right.'' So to 
show that they are right, we need to \emph{prove} them.

Nevertheless, we cannot prove something from absolutely nothing. We can only get nothing from the void. We should 
at least start from something. These beginnings are called the axioms. We admit their correctness without proof. 
Then one might ask a question: \emph{How about treat the existence and the properties of the reals, the 
operations, and the relations as axioms?}. This is a good question. But have you ever thought about what if some 
of these properties are inconsistent, that is, they violate each other, for example, property $A$ denies $B$? You 
can bet that they are consistent, but to show the consistence, we can do nothing but to prove. (However, G\"odel's 
second incomplete theorem states that many axiom systems cannot demonstrate its own consistence. So we'd better say
that for some axiom systems we can do nothing but to believe their consistence.)

So our idea is, to give as few axioms that are consistent with each other as possible (though most of the time we 
can only believe so), and then we start from this very fundamental. We will define and prove all other things 
along the way. Note that definitions can also be treated as some kind of axioms, after all. And when our journey 
is finished, all properties are verified. We can just proceed as we did before, but this time we are more certain 
and have a better understanding of what we are doing.