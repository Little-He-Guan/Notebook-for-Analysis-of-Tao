\part{Set Theory}

\section{Fundamentals}
\paragraph{Exercise 3.1.1} \label{exercise3.1.1}
\begin{proof}
Reflexive: $\forall x \in S, x \in S$.

Symmetric: 
\begin{align*}
X = Y
&\Longleftrightarrow \\
\forall x \in X, x \in Y \wedge \forall x \in Y, x \in X
&\Longleftrightarrow \\
Y = X
\end{align*}

Transitive:
$X=Y \Longrightarrow \forall x \in X, x \in Y$. Because $x \in Y$ and $Y = Z$, we can conclude that 
$\forall x \in X, x \in Z$. Conduct the process from inversely, we can get $\forall x \in Z, x \in X$. 
Therefore, $X=Z$.
\end{proof}

The reason for the content beneath Axiom 3.2 is clearly demonstrated in the proof of Lemma 3.1.6.

In Remarks 3.1.9, there are three ``Why''s. The reason can be concluded as: Because of the ``if and only 
if'' in Axiom 3.3, or more precisely, ``only if'', if $x$ is a element in one of such sets, $x$ must 
$=a$ or $b$. And because of the ``if'', $x$ is thus in another set. So the two sets are equal according 
to Definition 3.1.4.

\paragraph{Exercise 3.1.2} \label{exercise3.1.2}
\begin{proof}
According to Axiom 3.2, $\varnothing$ exists, and is thus an object as stated by Axiom 3.1. Therefore, 
by Axiom 3.3, $\{\varnothing\}$ also exists. $\varnothing$ is an element of $\{\varnothing\}$, but it 
is not an element of $\varnothing$ because any object $\notin$ $\varnothing$.

For the same reason, any set that contains element(s) is not the same set as $\varnothing$. Furthermore, 
there exists an object $\{\varnothing\}$ (Axiom 3.3 and 3.1), which is an element of 
$\{\varnothing, \{\varnothing\}\}$, but which is not an element of $\{\varnothing\}$. So the two sets 
are not equal.
\end{proof}

\paragraph{Remarks 3.1.12}
\begin{proof}
Let $x \in A'\cup B$. $x \in A' \Longrightarrow x \in A$ And if $x \notin A'$, $x \in B$. So either way 
$x \in A\cup B$ and vice versa.
\end{proof}

\paragraph{Exercise 3.1.3} \label{exercise3.1.3}
\begin{proof}
(1)
\[
x \in A \cup B \equiv (x \in A \vee x \in B)
\]
\[
x \in A \Longrightarrow x \in B \cup A
\]
\[
x \in B \Longrightarrow x \in B \cup A
\]
So $x \in A \cup B \Longrightarrow x \in B \cup A$. And vice versa.

(2)
$x \in A \Rightarrow x \in A \cup A$ and $x \in A \cup A \Rightarrow x \in A$.

(3)
\begin{align*}
x \in A \cup \varnothing 
&\Longrightarrow \\
x \in A \vee x \in \varnothing
&\Longrightarrow \\
x \in A \tag{$\forall a, a \notin \varnothing$}
\end{align*}

And obviously $x \in A \Rightarrow x \in A \cup \varnothing$. So $A \cup \varnothing = A$.

By transitivity of equality, and commutativity of pairwise union, we can conclude the others.
\end{proof}

\paragraph{Examples 3.1.17}
\begin{proof}
\[
\forall x(x\in A \Longrightarrow x \in A)
\]

And
\[
\forall x(x \in \varnothing \Longrightarrow x \in A)
\]
is vacuously true.
\end{proof}

\paragraph{Exercise 3.1.4} \label{exercise3.1.4}
\begin{proof}
(1) 
On one hand,
\[
A \subseteq B \equiv \forall x(x \in A \Longrightarrow x \in B)
\]. 
On the other hand,
\[
B \subseteq A \equiv \forall x(x \in B \Longrightarrow x \in A)
\].
Thus $A=B$.

(2)
First, we prove that $A \subsetneq B \Longrightarrow \exists x(x \in B \wedge x \notin A)$. Suppose the 
contradiction, that is, $\forall x(x\in B \Longrightarrow x \in A)$, which is impossible since 
$(A \subseteq B \equiv \forall x(x\in A \Longrightarrow x \in B))\wedge A \neq B$.

According to what's proven in the book, 
$A \subsetneq B \wedge B \subsetneq C \Longrightarrow A \subseteq C$.

Now we prove that $\exists x(x \in C \wedge x \notin A)$. Since $x \in A \Longrightarrow x \in B$, 
$x \notin B \Longrightarrow x \notin A$. Because $B \subsetneq C$, 
$\exists x(x \in C \wedge x \notin B)$, and thus for such $x$, $ x \notin A$. Then $A \neq C$.

So $A \subsetneq C$. 
\end{proof}

\paragraph{Axiom 3.5}
(1) Because $x \in \{x \in A : P(x)\} \Rightarrow x \in A$.

(2) Because both $\in$ and $P(x)$ obey the axiom of substitution.

\paragraph{Exercise 3.1.5} \label{exercise3.1.5}
\begin{proof}
First we prove that $A \subseteq B \equiv A \cup B = B$. On one hand,
\begin{align*}
A \subseteq B 
&\equiv \\
\forall x(x \in A \Longrightarrow x \in B)
&\Longrightarrow \\
\forall x((x \in A \vee x \in B) \Longrightarrow x \in B)
&\equiv \\
A \cup B = B
\end{align*}.

On the other hand,
\[
\forall x((x \in A \vee x \in B) \Longrightarrow x \in B) \Longrightarrow
\forall x(x \in A \Longrightarrow x \in B)
\]. The statement is therefore proven.

Then we prove that $A \subseteq B \equiv A \cap B = A$. On one hand, 
\begin{align*}
(A \cap B = A \equiv \forall x(x \in A \wedge x \in B \equiv x \in A)) 
&\Longrightarrow \\
(\forall x(x \in A \Rightarrow x \in B) \equiv (A \subseteq B))
\end{align*}.

On the other hand, 
\[
\forall x(x\in A\wedge x \in B \Longrightarrow x \in A)
\]
is always true (Vacuously true if $x \notin B$).

Logical equality is transitive, and thus all of the three statements are equal.
\end{proof}

\paragraph{Proposition 3.1.28} (Exercise 3.1.6) \label{exercise3.1.6}
\begin{proof}
(a) The two are identical to 
\[
\forall x(x \in A \vee x \in \varnothing \equiv x \in A)
\], 
and 
\[
\nexists x(x \in A \wedge x \in \varnothing)
\], 
which are all true since $\forall x(x \notin \varnothing)$.

(b) We have $A \subseteq X$. According to what we have proven in 
\hyperref[exercise3.1.5]{Exercise 3.1.5}, the two statements are all true.

(c) Obvious since 
\[
\forall x(x \in A \vee x \in A \equiv x \in A)
\]
and 
\[
\forall x(x \in A \wedge x \in A \equiv x \in A)
\]

(d) All true since \emph{logical or} and \emph{logical and} are commutative.

(e) See Lemma 3.1.13. I believe that this can be concluded by the fact that \emph{logical or} and 
\emph{logical and} are also associative.

(f) 
First we prove the latter. On one hand, suppose 
\[
x \in A \cup (B \cap C)
\] is ture.

If $x \in A$, then $x \in$ both $A \cup B$ and $A \cup C$, and thus $\in$ 
$(A \cup B)\cap(A \cup C)$.

If $x \notin A$, then $x \in B \cap C$, then $x \in$ both $A \cup B$ and $A \cup C$, and thus $\in$ 
$(A \cup B)\cap(A \cup C)$.

On the other hand, 
suppose 
\[
x \in (A \cup B)\cap(A \cup C)
\] is true.

If $x \in A$, obviously $x \in A \cup (B \cap C)$.

If $x \notin A$, then $x$ must $\in B \cap C$, and thus also $\in A \cup (B \cap C)$.

Now we prove the former. On one hand, suppose
\[
x \in A \cap (B \cup C)
\] is true.

If $x \in A \wedge x \in B$, then $x \in A \cap B$, and thus $\in (A \cap B)\cup(A \cap C)$.

If $x \notin A \vee x \notin B$, then
\begin{enumerate}
\item if $x \notin A$, this is impossible.
\item if $x \in A$, then $x \notin B$. But $x \in B \cup C$, so $x \in C$. And thus 
$x \in A \cap C \Rightarrow x \in (A \cap B)\cup(A \cap C)$.
\end{enumerate}

On the other hand, suppose that 
\[
x \in (A \cap B)\cup(A \cap C)
\] is true.

First we can see that $x \in A$. 

If $x \in B$, then $x \in B \cup C$, and thus $\in A \cap (B \cup C)$.

If $x \notin B$, then $x \in C$. So $x \in B \cup C$, and thus $\in A \cap (B \cup C)$.

(g)
Now we prove the former: On one hand, suppose that 
\[
x \in A \cup (X-A)
\].

If $x \in A$, then $x \in X$ since $A \subseteq X$.

If $x \notin A$, then $x \in X-A$, and thus also $\in X$.

On the other hand, suppose that 
\[
x \in X
\]

If $x \in A$, then $x \in A \cup (X-A)$.

If $x \notin A$, then $x \in X-A$, and thus $\in A \cup (X-A)$.

(h)
$x \in X - A$ requires $x \notin A$. So $\forall x(x \in A \cap (X-A))$ is always false. 
Thus
\[
\forall x(x \in A \cap (X-A) \Longleftrightarrow x \in \varnothing)
\] (vacuously true).
\end{proof}

\paragraph{Exercise 3.1.7} \label{exercise3.1.7}
\begin{proof}
(1)
$\forall x(x \in A \cap B \Longrightarrow x \in A)$. Similarly, we can prove that 
$A \cap B \subseteq B$. (This can also be achieved via the commutativity).

(2) 
On one hand, suppose that 
\[
C \subseteq A \wedge C \subseteq B
\] is true.
Then, 
\[
\forall x(x \in C \Longrightarrow x \in A \wedge x \in B \Longrightarrow x \in A \cap B)
\].

On the other hand, suppose that
\[
C \subseteq A \cap B
\] is true.
Then, 
\[
\forall x(x \in C \Longrightarrow x \in A \wedge x \in B)
\].
That is, $C \subseteq A \wedge C \subseteq B$.

(3) It is immediately given by 
\[
\forall x(x \in A \Longrightarrow x \in A \cup B)
\]. 
Since $\cup$ is commutative, the latter case is proven.

(4) On one hand, suppose that $A \subseteq C \wedge B \subseteq C$ and let $x \in A \cup B$.

If $x \in A$, then $x \in C$.

If $x \notin A$, then $x \in B$, and thus $x \in C$.

On the other hand, suppose that $A \cup B \subseteq C$. Then, 
\[
\forall x(x \in A \Longrightarrow x \in A \cup B \Longrightarrow x \in C)
\]
\[
\forall x(x \in B \Longrightarrow x \in A \cup B \Longrightarrow x \in C) \qedhere
\].
\end{proof}

\paragraph{Exercise 3.1.8} \label{exercise3.1.8}
\begin{proof}
The former: On one hand, 
Suppose that 
\[
x \in A \cap (A \cup B)
\].

If $x \in A$, then $x \in A$.

If $x \notin A$, this is impossible.

On the other hand, suppose that $x \in A$.
Then $x \in A \wedge x \in (A \cup B)$, so $x \in A \cap (A \cup B)$.

The latter: On one hand, suppose that $x \in A \cup (A \cap B)$.
\[
x \in A \Longrightarrow x \in A.
\]
\[
x \notin A \Longrightarrow x \in (A \cap B) \Longrightarrow x \in A
\].

On the other hand, Suppose that $x \in A$, then $x \in A \cup (A \cap B)$.
\end{proof}

\paragraph{Exercise 3.1.9} \label{exercise3.1.9}
\begin{proof}
\begin{lem} \label{lem10}
\[
\nexists x\forall B\forall A(x \in A \wedge x \in B \wedge A \cap B = \varnothing)
\]
\end{lem}
\begin{proof}
Suppose the contradiction: $x \in A \wedge x \in B \wedge A \cap B = \varnothing$, then 
$x \in A \cap B$, and thus $\in \varnothing$, which is impossible.
\end{proof}

The former: On one hand, suppose that $x \in A$. Then $x \notin B$ by Lemma \ref{lem10}. And
\[
x \in A \Longrightarrow x \in A \cup B \Longrightarrow x \in X
\].
So $x \in (X-B)$.

On the other hand, suppose that $x \in (X-B)$, then $x \in A \cup B$. But $x \notin B$, so $x \in A$ by 
Lemma \ref{lem10}.

The latter is immediately proven since $\cap,\cup$ are commutative.
\end{proof}

\paragraph{Exercise 3.1.10} \label{exercise3.1.10}
\begin{proof}
Firstly we prove that $(A-B)\cap (A\cap B) = \varnothing$.

$x \in (A\cap B)$ gives $x \in B$, but $x \in (A-B)$ gives $x \notin B$. So the two statements can not 
be true simultaneously. Which means 
\[
x\in (A-B)\cap (A\cap B) \Longrightarrow x \in \varnothing
\]

And obviously 
\[
x\in (A-B)\cap (A\cap B) \Longleftarrow x \in \varnothing
\].

Similarly we can conclude all of the three sets are disjoint by the fact that $\nexists x \in$ either 
two of the three sets.

Now we are showing that their union is $A \cup B$.

On one hand, suppose that 
\[
x \in (A-B)\cap(A\cap B)\cap(B-A)
\].
$x$ can at most be in one of these sets since they are disjoint.
If $x \in A$, then $x \in A \cup B$.

If $x \notin A$, then $x \in (B-A)$, and thus $x \in B$. So $x \in A \cup B$.

On the other hand, suppose that $x \in A \cup B$.
Then $x$ either
\begin{enumerate}
\item $\in A$, but $\notin B$, or
\item $\in B$, but $\notin A$, or
\item $\in$ both $A,B$.
\end{enumerate}

If (1), then $x \in (A-B)$.

If (2), then $x \in (B-A)$.

If (3), then $x \in A\cap B$.

In conclusion, we can see that $x \in (A-B)\cap(A\cap B)\cap(B-A)$.
\end{proof}

\paragraph{Exercise 3.1.11} \label{exercise3.1.11}
\begin{proof}
Let $S$ be a set.
Let $P(x,y)$ be a property pertaining to $x \in S$ and any object $y$, and is true iff 
$Q(x) \wedge y = x$, where $Q(x)$ is a property pertaining to $x \in S$.

According to Axiom 3.6, there exists a set $Z$, such that 
$y \in Z \equiv x \in S \wedge P(x,y)$, which means $y \in Z \equiv x \in S \wedge Q(x) \wedge x = y$. 
So is the axiom of specification proven.
\end{proof}

\section{Russell's paradox}
I think one major reason for building such a ``cumbersome'' axiom system is to restrict the way to 
construct sets. We can not construct just any set we want, there only exist certain kinds of sets.

\paragraph{Exercise 3.2.1} \label{exercise3.2.1}
\begin{proof}
(Axiom 3.2) To prove the existence of the empty set, simply choose a property that is false for all 
objects.

(Axiom 3.3) To prove the existence of a \emph{pair set}, say $\{a,b\}$, let $P(x)$ be a property 
pertaining to any object $x$, and is true iff $x = a \vee x = b$.

(Axiom 3.4) Let the property be $P(x): x \in A \vee x \in B$.

(Axiom 3.5) Let the property be $Q(x): x \in A \wedge P(x)$, where $P(x)$ is a property pertaining to 
elements of $A$.

(Axiom 3.6) Let the property be $Q(y): P(x,y)$ is true for some $x \in A$.
\end{proof}

\paragraph{Exercise 3.2.2} \label{exercise3.2.2}
\begin{proof}
(1)
Suppose the contradiction: $\exists A(A \in A)$. Then by Axiom 3.3, construct a set $B:= \{A\}$. $A$ is 
the only element in $B$. $A$ is a set. $A$ is not disjoint from $B$, for $A \in A \wedge A \in B$. 

(2)
Suppose the contradiction: $A \in B \wedge B \in A$. Construct a set $S: \{A,B\}$. $A$ is an element of 
$S$. $A$ is a set. $A$ is not disjoint from $S$, for $B \in A \wedge B \in S$.
\end{proof}

\paragraph{Exercise 3.2.3} \label{exercise3.2.3}
On one hand, if Axiom 3.8 is true, we can choose a property $P(x)$ which is true for all objects. Thus 
we have $\Omega$.

On the other hand, if there exists such a set as $\Omega$, we can use Axiom 3.5 to construct any set 
we want from it. (e.g. If we want a set to have these elements: $a,b,\dots$, we can let 
$P(x):= x = a \vee x = b, \vee \dots$.)

\section{Functions}
In Example 3.3.3, Tao asked why $x'=x \Rightarrow f(x')=f(x)$. The reason is, the property $P(x,y)$ 
obeys the axiom of substitution. Thus, $P(x,y)\equiv P(x',y)$. According to definition, since 
$x' \in X$, $y$ is unique.

In Example 3.3.9, Tao asked why all functions whose domain is $\varnothing$ and whose range is the same 
are equal. The reason is $x \in \varnothing \Longrightarrow f(x) = g(x)$ is vacuously true.

\paragraph{Exercise 3.3.1} \label{exercise3.3.1}
\begin{proof}
The properties of equality are all true since in definition, we only use $f(x) = g(x)$, in which the $=$ 
obeys these rules, plus the fact that the output is unique.

Then the substitution:
\begin{align*}
f = \overset{\sim}{f} &\Longrightarrow \\
f(x) = \overset{\sim}{f}(x) &\Longrightarrow \\
g(f(x)) = g(\overset{\sim}{f}(x))
\end{align*}. 
And then $\overset{\sim}{g}(\overset{\sim}{f}(x)) = g(\overset{\sim}{f}(x)) = g(x)$.
\end{proof}

\paragraph{Exercise 3.3.2} \label{exercise3.3.2}
\begin{proof}
The former: Suppose the contradiction:
\[
\exists x \exists x'(g(f(x)) = g(f(x')) \wedge x \neq x')
\]
Then, 
\begin{align*}
g(f(x)) = g(f(x')) &\Longrightarrow \\
f(x)= f(x') \tag{$g$ is injective} &\Longrightarrow \\
x = x' \tag{$f$ is injective}
\end{align*}, 
which is impossible.

The latter: Suppose the contradiction:
\[
\exists z \forall x(z \in Z \wedge g \circ f(x) \neq z) 
\]
Then, we can conclude that $\exists y \forall x(y \in Y \wedge y \neq f(x))$, since $g$ is surjective. 
This is impossible as $f$ is surjective.
\end{proof}

\paragraph{Exercise 3.3.3} \label{exercise3.3.3}
\begin{proof}
\begin{large}
\textbf{Attention:}
\end{large}
Different interpretations for injectivity may result in different conclusions. 
I have asked a question at 
\href{https://math.stackexchange.com/questions/3800240/how-to-interpret-the-definition-of-injectivity}{Stack Exchange} regarding this problem.

Let the range be $Y$, and the function be $f$.
Injectivity:
\[
\forall x'\forall x((x \in \varnothing \wedge x' \in \varnothing) \Longrightarrow
(x \neq x' \Longrightarrow f(x) \neq f(x')))
\], 
which is always vacuously true.

Surjectivity:
\[
\forall y(y \in Y \Longrightarrow \exists x(x \in \varnothing \wedge f(x) = y))
\], 
which is false if $Y \neq \varnothing$, and which is vacuously true if $Y = \varnothing$.

Bijective: True if $Y = \varnothing$.
\end{proof}

\paragraph{Exercise 3.3.4} \label{exercise3.3.4}
\begin{proof}
The former: $f,\overset{\sim}{f}$ have the same range and domain. 
\[
\forall x(g \circ f = g \circ \overset{\sim}{f} \Longrightarrow g(f(x)) = g(\overset{\sim}{f}(x)))
\]
We know that $g$ is injective, so $\forall x \in X, f(x) = \overset{\sim}{f}(x)$. Thus 
$f = \overset{\sim}{f}$.

It is not true if $g$ is not injective. Consider an extreme condition, where $g$ is constant. So 
whatever $f,\overset{\sim}{f}$ are, $g \circ f = g \circ \overset{\sim}{f}$ are always equal.

The latter: Suppose the contradiction:$g \neq \overset{\sim}{g}$.
$g,\overset{\sim}{g}$ have the same range and domain. But they are not equal, so 
$\exists y(y \in Y \wedge g(y) \neq \overset{\sim}{g}(y))$. Because $f$ is surjective, 
$\exists x(x \in X \wedge f(x) = y)$. However, $g \circ f(x) = \overset{\sim}{g} \circ f(x)$, so 
this is impossible.

It is not true if $f$ is not surjective. We can make $g(y) = \overset{\sim}{g}(y)$ when $y=f(x)$, but 
as well make $g(y') \neq \overset{\sim}{g}(y')$ if $\nexists x(y'=f(x))$.
\end{proof}

\paragraph{Exercise 3.3.5} \label{exercise3.3.5}
\begin{proof}
Injectivity:
Suppose the contradiction, that 
\[
\exists x \exists x'(x \neq x' \wedge f(x) = f(x'))
\], which immediately gives 
\[
g(f(x)) = g(f(x'))
\], and thus is impossible.

$g$ has not to be also injective, because $f$ being so ensures that an unique input $x$ gives an unique 
input to $g$.

Surjectivity:
If $g$ is not surjective, then $\exists z \forall y(z \in Z \wedge y \in Y \wedge z \neq g(y))$
And whatever $x$ is, $f(x) \in Y$, so $g(f(x)) \neq z$, which is a contradiction.

$f$ has not to be surjective as long as its ``real'' domain is large enough to form the set $Z$ through 
$g$. For example (Informal), let $g$ be $z = |y|, \mathbb{R} \rightarrow \mathbb{R}^{+}\cup\{0\}$, and 
let $f$ be $y = x, \mathbb{R}^{+}\cup\{0\} \rightarrow \mathbb{R}$.
\end{proof}

\paragraph{Exercise 3.3.6} \label{exercise3.3.6}
\begin{proof}
The latter:
By definition, $P(y,x)$ of $x = f^{-1}(y)$ is $f(x)=y$. Substitute $x$ with $f^{-1}(y)$, and here we 
have $f(f^{-1}(y)) = y$, where $y \in Y$.

The former: Let $y = f(x)$. According to what we have proven, \\
$f(f^{-1}(y)) = y$. Substitute $y$ with 
$f(x)$, we have $f(f^{-1}(f(x))) = f(x)$. Since that $f(x)$ is injective, we have $f^{-1}(f(x)) = x$.

Now we need to show that $f^{-1}$ is bijective. Assume that it is not injective, thus 
$\exists x \exists x'(x \in Y \wedge x' \in Y \Longrightarrow(x\neq x' \Longrightarrow f^{-1}(x) = 
f^{-1}(x')))$.
However, according to the latter conclusion, $f^{-1}(x) = f^{-1}(x') \Longrightarrow x=x'$, a 
contradiction, so $f^{-1}$ must be injective.

And it is also surjective. $\forall x \in X$, $\exists y \in Y, f^{-1}(y)=x$. According to the former 
conclusion, $y$ is $f(x)$.

So now $f^{-1}$ is bijective, and thus has its inverse. By definition, $P(x,y)$ of 
$y = (f^{-1})^{-1}(x)$ is $f^{-1}(y) = x$, where $x \in X$. According to the former conclusion, 
$f^{-1}(f(x)) = x$. Thus
\[
f^{-1}(y) = f^{-1}(f(x)) \Longrightarrow y = f(x) \Longrightarrow (f^{-1})^{-1}(x) = f(x)
\], which is true $\forall x \in X$. And since they have the same domain and range, $(f^{-1})^{-1} = f$.
\end{proof}

\paragraph{Exercise 3.3.7} \label{exercise3.3.7}
\begin{proof}
Injectivity: 
\[
g \circ f(x) = g \circ f(x') \Longrightarrow f(x) = f(x') \Longrightarrow x = x'
\]

Surjectivity:
For each $z \in Z$, we need to find $x \in X$ such that $g \circ f(x) = z$. By the surjectivity of $g$, 
we can find $y \in Y$ such that $g(y) = z$. We can also find $a \in X$ such that $f(a) = y$ as $f$ is 
surjective. So $a$ is our desired $x$.

The $P(z,x)$ of $x = (g \circ f)^{-1}(z)$ is $z = g \circ f(x)$. Consider the following expression:
\begin{align*}
f^{-1} \circ g^{-1} (z)
&= f^{-1} \circ g^{-1} (g \circ f(x)) \\
&= f^{-1}(\textcolor{red}{g^{-1}(g(}f(x)\textcolor{red}{))}) \\
&= f^{-1}(f(x)) \\
&= x
\end{align*}
So $(g \circ f)^{-1} = f^{-1} \circ g^{-1}$. Therefore, they are equal as they have the same domain and 
range.
\end{proof}

\paragraph{Exercise 3.3.8} \label{exercise3.3.8}
\begin{proof}
(a) First they have the same domain and range. Finally, 
\[
\forall x(x\in X \Longrightarrow x=x \Longrightarrow \iota_{Y \rightarrow Z} \circ 
\iota_{X \rightarrow Y} = \iota_{X \rightarrow Z})
\]

(b) On one hand, they have the same domain and range.

On the other hand, 
\begin{align*}
f \circ \iota_{A \rightarrow A}(x) 
&= f(\iota_{A \rightarrow A}(x)) \\
&= f(x) \\
&= \iota_{B \rightarrow B}(f(x)) \\
&= \iota_{B \rightarrow B} \circ f (x)
\end{align*}

(c) It is easy to see that they have the same domain and range. 

\[
f \circ f^{-1} (b) = b = \iota_{B \rightarrow B}
\]
\[
f \circ f^{-1} (a) = a = \iota_{A \rightarrow A}
\]

(d) It is easy to see that they have the same domain and range.

Let $h$ be 
$h(x) = f(x)$, if $x \in X$, $h(x) = g(x)$, if $x \in Y$. 

For each $x \in X$, $\iota_{X \rightarrow X \cup Y}(x) = x$, so 
$h(\iota_{X \rightarrow X \cup Y}(x)) =f(x)$.

Similarly we can prove $h(\iota_{Y \rightarrow X \cup Y}(x)) = g(x)$ for each $x \in Y$.
\end{proof}

\section{Images and inverse Images}
\paragraph{Definition 3.4.1}
To prove that $f(S)$ is well-defined by using the axiom of specification, we need to apply it to set 
$Y$, not $X$. Let $P(y)$ be a property pertaining to each $y \in Y$, which is true iff 
$\exists x(x \in S \wedge f(x) = y)$. According to the axiom of specification, there exists a set that 
contains every $y \in Y$ such that $P(y)$ is true.

In some places where Tao asked ``(Why?)'', the reason is obvious, so I don't write them here. 

\paragraph{Example 3.4.6}
This is because 
\[
f^{-1}(f(\{-1,0,1,2\})) = f^{-1}(\{1,0,4\}) = \{-1,1,0,2,-2\}
\].

More generally, if $f$ (whose domain is $X$, and whose range is $Y$) is not injective, then 
\[
\exists x \exists x'((x \in X \wedge x' \in X) \wedge (x\neq x' \wedge f(x) = f(x')))
\]. 
Let $D \subseteq X$ such that $x \in D \wedge x' \notin D$. Then $f(x) = f(x') \in f(D)$. And thus 
$x,x' \in f^{-1}(f(D)) \Longrightarrow f^{-1}(f(D)) \neq D$. 

\paragraph{Exercise 3.4.1} \label{exercise3.4.1}
\begin{proof} 
$f^{-1}(V)$ may be interpreted in two different ways:

(1) Interpret $f^{-1}(V)$ as an inverse image, that is,
\[
(\forall x \in X)(x \in f^{-1}(V) \equiv f(x) \in V)
\]
\[
(\forall x \notin X)(x \notin f^{-1}(V))
\]

(2) Interpret $f^{-1}(V)$ as an image, where we regard $f^{-1}$ as a function. So, 
\[
\forall x(\exists y(y \in V \wedge x = f^{-1}(y)) \equiv x \in f^{-1}(V))
\]

We need to show that if the two statements are well-defined($x \in X$) , they are logically equivalent.

Let $S_1$ be the set defined in form (1), $S_2$ be the set defined in form (2). For every $x \in S_1$, 
$f(x) \in V$. Let $y' = f(x), x' = f^{-1}(y')$, then by definition (2) we have $x' \in S_2$. But 
$x' = x$, so $\forall x(x \in S_1 \Longrightarrow x \in S_2)$.

On the other hand, for every $a \in S_2$, $\exists b \in V$, such that $a = f^{-1}(b)$. Then $a \in X$. 
$f(a) = f(f^{-1}(b)) = b \in V$, so $a \in S_1$. Thus, $S_1 = S_2$.
\end{proof}

\paragraph{Exercise 3.4.2} \label{exercise3.4.2}

\paragraph{Exercise 3.4.6} \label{exercise3.4.6}
\begin{proof}
\textbf{My own proof:} According to Axiom 3.10, we can construct the set $X^X$. Apply the axiom 
of replacement to each element of $X^X$, we construct a set $Z$ such that
\[
\forall x(x \in Z \equiv \exists f(f \in X^X \wedge x = f(X)))
\]

Let $Y = \{\varnothing\} \cup Z$.

Now we prove that $Y$ is the set we want. On one hand, for any $S \subseteq X$, 
if $S = \varnothing$, then $S \in Y$, as $Y = \{\varnothing\} \cup Z$.

If $S \neq \varnothing$, there exists a surjective function $g: X \rightarrow S$. $g\in X^X$, and 
$g(X) = S$, so $S \in Z$, and thus $S \in Y$. (To show the existence of $g$, for example, let $x \in X, 
g(x) = x$ if $x \in S$, and for $x \in X \wedge x \notin S$, $g(x)$ can be any element of $X$.)


On the other hand, for any $S' \nsubseteq X$, $\exists a(a \in S' \wedge a \notin X)$. To prove that 
$S' \notin Y$, we need to show that $\nexists f(f \in X^X \wedge S' = f(X))$. We know that 
$\nexists x(x \in X \wedge f(x) = a)$, so $a \notin f(X)$. Therefore $S' \neq f(X)$, so $S' \notin Y$.

$Y$ is the set we want.

I posted a question \href{https://math.stackexchange.com/questions/3803487/is-this-proof-to-the-existence-of-a-set-that-contains-all-subsets-of-another-set}{here} for verification for this proof. 
Thanks to answers of people at Stack Exchange so that my proof can be refined.

\textbf{Proof By Tao's Hint:}
For each $S \subseteq X$, let a function $f_S$ be $f_S(x) = 1$ if $x \in S$, and $f_S(x)=0$ if 
$x \in X \wedge x \notin S$. Then $f^{-1}_S(\{1\})$ gives $S$. 

Now we show that any element in 
$\{0,1\}^X$ is some $f_S$. Let $g \in \{0,1\}^X$. Then if $\forall x \in X, g(x) = 0$, then 
$g = f_{\varnothing}$. Otherwise, there exists a set that contains all $x$ such that $g(x) = 0$ by 
axiom of specification, namely $R$. Then $g = f_R$.

On the other hand, each $f_S$ is obviously an element of $\{0,1\}^X$.

Use the axiom of replacement, we construct a set $Y$ such that
\[
\forall x (x \in Y \equiv \exists f(f \in \{0,1\}^X \wedge x = f^{-1}(\{1\})))
\]

According to what we have proven, $Y$ is the set we want.
\end{proof}