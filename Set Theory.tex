% Copyright (C) He Guanyuming 2020
% The file is licensed under the MIT license.

\part{Set Theory}

\section{Fundamentals}
\paragraph{Exercise 3.1.1} \label{exercise3.1.1}
\begin{proof}
Reflexive: $\forall x \in S, x \in S$.

Symmetric: 
\begin{align*}
X = Y
&\Longleftrightarrow \\
\forall x \in X, x \in Y \wedge \forall x \in Y, x \in X
&\Longleftrightarrow \\
Y = X
\end{align*}

Transitive:
$X=Y \Longrightarrow \forall x \in X, x \in Y$. Because $x \in Y$ and $Y = Z$, we can conclude that 
$\forall x \in X, x \in Z$. Conduct the process from inversely, we can get $\forall x \in Z, x \in X$. 
Therefore, $X=Z$.
\end{proof}

The reason for the content beneath Axiom 3.2 is clearly demonstrated in the proof of Lemma 3.1.6.

In Remarks 3.1.9, there are three ``Why''s. The reason can be concluded as: Because of the ``if and only 
if'' in Axiom 3.3, or more precisely, ``only if'', if $x$ is a element in one of such sets, $x$ must 
$=a$ or $b$. And because of the ``if'', $x$ is thus in another set. So the two sets are equal according 
to Definition 3.1.4.

\paragraph{Exercise 3.1.2} \label{exercise3.1.2}
\begin{proof}
According to Axiom 3.2, $\varnothing$ exists, and is thus an object as stated by Axiom 3.1. Therefore, 
by Axiom 3.3, $\{\varnothing\}$ also exists. $\varnothing$ is an element of $\{\varnothing\}$, but it 
is not an element of $\varnothing$ because any object $\notin$ $\varnothing$.

For the same reason, any set that contains element(s) is not the same set as $\varnothing$. Furthermore, 
there exists an object $\{\varnothing\}$ (Axiom 3.3 and 3.1), which is an element of 
$\{\varnothing, \{\varnothing\}\}$, but which is not an element of $\{\varnothing\}$. So the two sets 
are not equal.
\end{proof}

\paragraph{Remarks 3.1.12}
\begin{proof}
Let $x \in A'\cup B$. $x \in A' \Longrightarrow x \in A$ And if $x \notin A'$, $x \in B$. So either way 
$x \in A\cup B$ and vice versa.
\end{proof}

\paragraph{Exercise 3.1.3} \label{exercise3.1.3}
\begin{proof}
(1)
\[
x \in A \cup B \equiv (x \in A \vee x \in B)
\]
\[
x \in A \Longrightarrow x \in B \cup A
\]
\[
x \in B \Longrightarrow x \in B \cup A
\]
So $x \in A \cup B \Longrightarrow x \in B \cup A$. And vice versa.

(2)
$x \in A \Rightarrow x \in A \cup A$ and $x \in A \cup A \Rightarrow x \in A$.

(3)
\begin{align*}
x \in A \cup \varnothing 
&\Longrightarrow \\
x \in A \vee x \in \varnothing
&\Longrightarrow \\
x \in A \tag{$\forall a, a \notin \varnothing$}
\end{align*}

And obviously $x \in A \Rightarrow x \in A \cup \varnothing$. So $A \cup \varnothing = A$.

By transitivity of equality, and commutativity of pairwise union, we can conclude the others.
\end{proof}

\paragraph{Examples 3.1.17}
\begin{proof}
\[
\forall x(x\in A \Longrightarrow x \in A)
\]

And
\[
\forall x(x \in \varnothing \Longrightarrow x \in A)
\]
is vacuously true.
\end{proof}

\paragraph{Exercise 3.1.4} \label{exercise3.1.4}
\begin{proof}
(1) 
On one hand,
\[
A \subseteq B \equiv \forall x(x \in A \Longrightarrow x \in B)
\]. 
On the other hand,
\[
B \subseteq A \equiv \forall x(x \in B \Longrightarrow x \in A)
\].
Thus $A=B$.

(2)
First, we prove that $A \subsetneq B \Longrightarrow \exists x(x \in B \wedge x \notin A)$. Suppose the 
contradiction, that is, $\forall x(x\in B \Longrightarrow x \in A)$, which is impossible since 
$(A \subseteq B \equiv \forall x(x\in A \Longrightarrow x \in B))\wedge A \neq B$.

According to what's proven in the book, 
$A \subsetneq B \wedge B \subsetneq C \Longrightarrow A \subseteq C$.

Now we prove that $\exists x(x \in C \wedge x \notin A)$. Since $x \in A \Longrightarrow x \in B$, 
$x \notin B \Longrightarrow x \notin A$. Because $B \subsetneq C$, 
$\exists x(x \in C \wedge x \notin B)$, and thus for such $x$, $ x \notin A$. Then $A \neq C$.

So $A \subsetneq C$. 
\end{proof}

\paragraph{Axiom 3.5}
(1) Because $x \in \{x \in A : P(x)\} \Rightarrow x \in A$.

(2) Because both $\in$ and $P(x)$ obey the axiom of substitution.

\paragraph{Exercise 3.1.5} \label{exercise3.1.5}
\begin{proof}
First we prove that $A \subseteq B \equiv A \cup B = B$. On one hand,
\begin{align*}
A \subseteq B 
&\equiv \\
\forall x(x \in A \Longrightarrow x \in B)
&\Longrightarrow \\
\forall x((x \in A \vee x \in B) \Longrightarrow x \in B)
&\equiv \\
A \cup B = B
\end{align*}.

On the other hand,
\[
\forall x((x \in A \vee x \in B) \Longrightarrow x \in B) \Longrightarrow
\forall x(x \in A \Longrightarrow x \in B)
\]. The statement is therefore proven.

Then we prove that $A \subseteq B \equiv A \cap B = A$. On one hand, 
\begin{align*}
(A \cap B = A \equiv \forall x(x \in A \wedge x \in B \equiv x \in A)) 
&\Longrightarrow \\
(\forall x(x \in A \Rightarrow x \in B) \equiv (A \subseteq B))
\end{align*}.

On the other hand, 
\[
\forall x(x\in A\wedge x \in B \Longrightarrow x \in A)
\]
is always true (Vacuously true if $x \notin B$).

Logical equality is transitive, and thus all of the three statements are equal.
\end{proof}

\paragraph{Proposition 3.1.28} (Exercise 3.1.6) \label{exercise3.1.6}
\begin{proof}
(a) The two are identical to 
\[
\forall x(x \in A \vee x \in \varnothing \equiv x \in A)
\], 
and 
\[
\nexists x(x \in A \wedge x \in \varnothing)
\], 
which are all true since $\forall x(x \notin \varnothing)$.

(b) We have $A \subseteq X$. According to what we have proven in 
\hyperref[exercise3.1.5]{Exercise 3.1.5}, the two statements are all true.

(c) Obvious since 
\[
\forall x(x \in A \vee x \in A \equiv x \in A)
\]
and 
\[
\forall x(x \in A \wedge x \in A \equiv x \in A)
\]

(d) All true since \emph{logical or} and \emph{logical and} are commutative.

(e) See Lemma 3.1.13. I believe that this can be concluded by the fact that \emph{logical or} and 
\emph{logical and} are also associative.

(f) 
First we prove the latter. On one hand, suppose 
\[
x \in A \cup (B \cap C)
\] is ture.

If $x \in A$, then $x \in$ both $A \cup B$ and $A \cup C$, and thus $\in$ 
$(A \cup B)\cap(A \cup C)$.

If $x \notin A$, then $x \in B \cap C$, then $x \in$ both $A \cup B$ and $A \cup C$, and thus $\in$ 
$(A \cup B)\cap(A \cup C)$.

On the other hand, 
suppose 
\[
x \in (A \cup B)\cap(A \cup C)
\] is true.

If $x \in A$, obviously $x \in A \cup (B \cap C)$.

If $x \notin A$, then $x$ must $\in B \cap C$, and thus also $\in A \cup (B \cap C)$.

Now we prove the former. On one hand, suppose
\[
x \in A \cap (B \cup C)
\] is true.

If $x \in A \wedge x \in B$, then $x \in A \cap B$, and thus $\in (A \cap B)\cup(A \cap C)$.

If $x \notin A \vee x \notin B$, then
\begin{enumerate}
\item if $x \notin A$, this is impossible.
\item if $x \in A$, then $x \notin B$. But $x \in B \cup C$, so $x \in C$. And thus 
$x \in A \cap C \Rightarrow x \in (A \cap B)\cup(A \cap C)$.
\end{enumerate}

On the other hand, suppose that 
\[
x \in (A \cap B)\cup(A \cap C)
\] is true.

First we can see that $x \in A$. 

If $x \in B$, then $x \in B \cup C$, and thus $\in A \cap (B \cup C)$.

If $x \notin B$, then $x \in C$. So $x \in B \cup C$, and thus $\in A \cap (B \cup C)$.

(g)
Now we prove the former: On one hand, suppose that 
\[
x \in A \cup (X-A)
\].

If $x \in A$, then $x \in X$ since $A \subseteq X$.

If $x \notin A$, then $x \in X-A$, and thus also $\in X$.

On the other hand, suppose that 
\[
x \in X
\]

If $x \in A$, then $x \in A \cup (X-A)$.

If $x \notin A$, then $x \in X-A$, and thus $\in A \cup (X-A)$.

(h)
$x \in X - A$ requires $x \notin A$. So $\forall x(x \in A \cap (X-A))$ is always false. 
Thus
\[
\forall x(x \in A \cap (X-A) \Longleftrightarrow x \in \varnothing)
\] (vacuously true).
\end{proof}

\paragraph{Exercise 3.1.7} \label{exercise3.1.7}
\begin{proof}
(1)
$\forall x(x \in A \cap B \Longrightarrow x \in A)$. Similarly, we can prove that 
$A \cap B \subseteq B$. (This can also be achieved via the commutativity).

(2) 
On one hand, suppose that 
\[
C \subseteq A \wedge C \subseteq B
\] is true.
Then, 
\[
\forall x(x \in C \Longrightarrow x \in A \wedge x \in B \Longrightarrow x \in A \cap B)
\].

On the other hand, suppose that
\[
C \subseteq A \cap B
\] is true.
Then, 
\[
\forall x(x \in C \Longrightarrow x \in A \wedge x \in B)
\].
That is, $C \subseteq A \wedge C \subseteq B$.

(3) It is immediately given by 
\[
\forall x(x \in A \Longrightarrow x \in A \cup B)
\]. 
Since $\cup$ is commutative, the latter case is proven.

(4) On one hand, suppose that $A \subseteq C \wedge B \subseteq C$ and let $x \in A \cup B$.

If $x \in A$, then $x \in C$.

If $x \notin A$, then $x \in B$, and thus $x \in C$.

On the other hand, suppose that $A \cup B \subseteq C$. Then, 
\[
\forall x(x \in A \Longrightarrow x \in A \cup B \Longrightarrow x \in C)
\]
\[
\forall x(x \in B \Longrightarrow x \in A \cup B \Longrightarrow x \in C) \qedhere
\].
\end{proof}

\paragraph{Exercise 3.1.8} \label{exercise3.1.8}
\begin{proof}
The former: On one hand, 
Suppose that 
\[
x \in A \cap (A \cup B)
\].

If $x \in A$, then $x \in A$.

If $x \notin A$, this is impossible.

On the other hand, suppose that $x \in A$.
Then $x \in A \wedge x \in (A \cup B)$, so $x \in A \cap (A \cup B)$.

The latter: On one hand, suppose that $x \in A \cup (A \cap B)$.
\[
x \in A \Longrightarrow x \in A.
\]
\[
x \notin A \Longrightarrow x \in (A \cap B) \Longrightarrow x \in A
\].

On the other hand, Suppose that $x \in A$, then $x \in A \cup (A \cap B)$.
\end{proof}

\paragraph{Exercise 3.1.9} \label{exercise3.1.9}
\begin{proof}
\begin{lem} \label{lem10}
\[
\nexists x\forall B\forall A(x \in A \wedge x \in B \wedge A \cap B = \varnothing)
\]
\end{lem}
\begin{proof}
Suppose the contradiction: $x \in A \wedge x \in B \wedge A \cap B = \varnothing$, then 
$x \in A \cap B$, and thus $\in \varnothing$, which is impossible.
\end{proof}

The former: On one hand, suppose that $x \in A$. Then $x \notin B$ by Lemma \ref{lem10}. And
\[
x \in A \Longrightarrow x \in A \cup B \Longrightarrow x \in X
\].
So $x \in (X-B)$.

On the other hand, suppose that $x \in (X-B)$, then $x \in A \cup B$. But $x \notin B$, so $x \in A$ by 
Lemma \ref{lem10}.

The latter is immediately proven since $\cap,\cup$ are commutative.
\end{proof}

\paragraph{Exercise 3.1.10} \label{exercise3.1.10}
\begin{proof}
Firstly we prove that $(A-B)\cap (A\cap B) = \varnothing$.

$x \in (A\cap B)$ gives $x \in B$, but $x \in (A-B)$ gives $x \notin B$. So the two statements can not 
be true simultaneously. Which means 
\[
x\in (A-B)\cap (A\cap B) \Longrightarrow x \in \varnothing
\]

And obviously 
\[
x\in (A-B)\cap (A\cap B) \Longleftarrow x \in \varnothing
\].

Similarly we can conclude all of the three sets are disjoint by the fact that $\nexists x \in$ either 
two of the three sets.

Now we are showing that their union is $A \cup B$.

On one hand, suppose that 
\[
x \in (A-B)\cap(A\cap B)\cap(B-A)
\].
$x$ can at most be in one of these sets since they are disjoint.
If $x \in A$, then $x \in A \cup B$.

If $x \notin A$, then $x \in (B-A)$, and thus $x \in B$. So $x \in A \cup B$.

On the other hand, suppose that $x \in A \cup B$.
Then $x$ either
\begin{enumerate}
\item $\in A$, but $\notin B$, or
\item $\in B$, but $\notin A$, or
\item $\in$ both $A,B$.
\end{enumerate}

If (1), then $x \in (A-B)$.

If (2), then $x \in (B-A)$.

If (3), then $x \in A\cap B$.

In conclusion, we can see that $x \in (A-B)\cap(A\cap B)\cap(B-A)$.
\end{proof}

\paragraph{Exercise 3.1.11} \label{exercise3.1.11}
\begin{proof}
Let $S$ be a set.
Let $P(x,y)$ be a property pertaining to $x \in S$ and any object $y$, and is true iff 
$Q(x) \wedge y = x$, where $Q(x)$ is a property pertaining to $x \in S$.

According to Axiom 3.6, there exists a set $Z$, such that 
$y \in Z \equiv x \in S \wedge P(x,y)$, which means $y \in Z \equiv x \in S \wedge Q(x) \wedge x = y$. 
So is the axiom of specification proven.
\end{proof}

\section{Russell's paradox}
I think one major reason for building such a ``cumbersome'' axiom system is to restrict the way to 
construct sets. We can not construct just any set we want, there only exist certain kinds of sets.

\paragraph{Exercise 3.2.1} \label{exercise3.2.1}
\begin{proof}
(Axiom 3.2) To prove the existence of the empty set, simply choose a property that is false for all 
objects.

(Axiom 3.3) To prove the existence of a \emph{pair set}, say $\{a,b\}$, let $P(x)$ be a property 
pertaining to any object $x$, and is true iff $x = a \vee x = b$.

(Axiom 3.4) Let the property be $P(x): x \in A \vee x \in B$.

(Axiom 3.5) Let the property be $Q(x): x \in A \wedge P(x)$, where $P(x)$ is a property pertaining to 
elements of $A$.

(Axiom 3.6) Let the property be $Q(y): P(x,y)$ is true for some $x \in A$.
\end{proof}

\paragraph{Exercise 3.2.2} \label{exercise3.2.2}
\begin{proof}
(1)
Suppose the contradiction: $\exists A(A \in A)$. Then by Axiom 3.3, construct a set $B:= \{A\}$. $A$ is 
the only element in $B$. $A$ is a set. $A$ is not disjoint from $B$, for $A \in A \wedge A \in B$. 

(2)
Suppose the contradiction: $A \in B \wedge B \in A$. Construct a set $S: \{A,B\}$. $A$ is an element of 
$S$. $A$ is a set. $A$ is not disjoint from $S$, for $B \in A \wedge B \in S$.
\end{proof}

\paragraph{Exercise 3.2.3} \label{exercise3.2.3}
On one hand, if Axiom 3.8 is true, we can choose a property $P(x)$ which is true for all objects. Thus 
we have $\Omega$.

On the other hand, if there exists such a set as $\Omega$, we can use Axiom 3.5 to construct any set 
we want from it. (e.g. If we want a set to have these elements: $a,b,\dots$, we can let 
$P(x):= x = a \vee x = b, \vee \dots$.)

\section{Functions}
In Example 3.3.3, Tao asked why $x'=x \Rightarrow f(x')=f(x)$. The reason is, the property $P(x,y)$ 
obeys the axiom of substitution. Thus, $P(x,y)\equiv P(x',y)$. According to definition, since 
$x' \in X$, $y$ is unique.

In Example 3.3.9, Tao asked why all functions whose domain is $\varnothing$ and whose range is the same 
are equal. The reason is $x \in \varnothing \Longrightarrow f(x) = g(x)$ is vacuously true.

\paragraph{Exercise 3.3.1} \label{exercise3.3.1}
\begin{proof}
The properties of equality are all true since in definition, we only use $f(x) = g(x)$, in which the $=$ 
obeys these rules, plus the fact that the output is unique.

Then the substitution:
\begin{align*}
f = \overset{\sim}{f} &\Longrightarrow \\
f(x) = \overset{\sim}{f}(x) &\Longrightarrow \\
g(f(x)) = g(\overset{\sim}{f}(x))
\end{align*}. 
And then $\overset{\sim}{g}(\overset{\sim}{f}(x)) = g(\overset{\sim}{f}(x)) = g(x)$.
\end{proof}

\paragraph{Exercise 3.3.2} \label{exercise3.3.2}
\begin{proof}
The former: Suppose the contradiction:
\[
\exists x \exists x'(g(f(x)) = g(f(x')) \wedge x \neq x')
\]
Then, 
\begin{align*}
g(f(x)) = g(f(x')) &\Longrightarrow \\
f(x)= f(x') \tag{$g$ is injective} &\Longrightarrow \\
x = x' \tag{$f$ is injective}
\end{align*}, 
which is impossible.

The latter: Suppose the contradiction:
\[
\exists z \forall x(z \in Z \wedge g \circ f(x) \neq z) 
\]
Then, we can conclude that $\exists y \forall x(y \in Y \wedge y \neq f(x))$, since $g$ is surjective. 
This is impossible as $f$ is surjective.
\end{proof}

\paragraph{Exercise 3.3.3} \label{exercise3.3.3}
\begin{proof}
\begin{large}
\textbf{Attention:}
\end{large}
Different interpretations for injectivity may result in different conclusions. 
I have asked a question at 
\href{https://math.stackexchange.com/questions/3800240/how-to-interpret-the-definition-of-injectivity}{Stack Exchange} regarding this problem.

Let the range be $Y$, and the function be $f$.
Injectivity:
\[
\forall x'\forall x((x \in \varnothing \wedge x' \in \varnothing) \Longrightarrow
(x \neq x' \Longrightarrow f(x) \neq f(x')))
\], 
which is always vacuously true.

Surjectivity:
\[
\forall y(y \in Y \Longrightarrow \exists x(x \in \varnothing \wedge f(x) = y))
\], 
which is false if $Y \neq \varnothing$, and which is vacuously true if $Y = \varnothing$.

Bijective: True if $Y = \varnothing$.
\end{proof}

\paragraph{Exercise 3.3.4} \label{exercise3.3.4}
\begin{proof}
The former: $f,\overset{\sim}{f}$ have the same range and domain. 
\[
\forall x(g \circ f = g \circ \overset{\sim}{f} \Longrightarrow g(f(x)) = g(\overset{\sim}{f}(x)))
\]
We know that $g$ is injective, so $\forall x \in X, f(x) = \overset{\sim}{f}(x)$. Thus 
$f = \overset{\sim}{f}$.

It is not true if $g$ is not injective. Consider an extreme condition, where $g$ is constant. So 
whatever $f,\overset{\sim}{f}$ are, $g \circ f = g \circ \overset{\sim}{f}$ are always equal.

The latter: Suppose the contradiction:$g \neq \overset{\sim}{g}$.
$g,\overset{\sim}{g}$ have the same range and domain. But they are not equal, so 
$\exists y(y \in Y \wedge g(y) \neq \overset{\sim}{g}(y))$. Because $f$ is surjective, 
$\exists x(x \in X \wedge f(x) = y)$. However, $g \circ f(x) = \overset{\sim}{g} \circ f(x)$, so 
this is impossible.

It is not true if $f$ is not surjective. We can make $g(y) = \overset{\sim}{g}(y)$ when $y=f(x)$, but 
as well make $g(y') \neq \overset{\sim}{g}(y')$ if $\nexists x(y'=f(x))$.
\end{proof}

\paragraph{Exercise 3.3.5} \label{exercise3.3.5}
\begin{proof}
Injectivity:
Suppose the contradiction, that 
\[
\exists x \exists x'(x \neq x' \wedge f(x) = f(x'))
\], which immediately gives 
\[
g(f(x)) = g(f(x'))
\], and thus is impossible.

$g$ has not to be also injective, because $f$ being so ensures that an unique input $x$ gives an unique 
input to $g$.

Surjectivity:
If $g$ is not surjective, then $\exists z \forall y(z \in Z \wedge y \in Y \wedge z \neq g(y))$
And whatever $x$ is, $f(x) \in Y$, so $g(f(x)) \neq z$, which is a contradiction.

$f$ has not to be surjective as long as its ``real'' domain is large enough to form the set $Z$ through 
$g$. For example (Informal), let $g$ be $z = |y|, \mathbb{R} \rightarrow \mathbb{R}^{+}\cup\{0\}$, and 
let $f$ be $y = x, \mathbb{R}^{+}\cup\{0\} \rightarrow \mathbb{R}$.
\end{proof}

\paragraph{Exercise 3.3.6} \label{exercise3.3.6}
\begin{proof}
The latter:
By definition, $P(y,x)$ of $x = f^{-1}(y)$ is $f(x)=y$. Substitute $x$ with $f^{-1}(y)$, and here we 
have $f(f^{-1}(y)) = y$, where $y \in Y$.

The former: Let $y = f(x)$. According to what we have proven, \\
$f(f^{-1}(y)) = y$. Substitute $y$ with 
$f(x)$, we have $f(f^{-1}(f(x))) = f(x)$. Since that $f(x)$ is injective, we have $f^{-1}(f(x)) = x$.

Now we need to show that $f^{-1}$ is bijective. Assume that it is not injective, thus 
$\exists x \exists x'(x \in Y \wedge x' \in Y \Longrightarrow(x\neq x' \Longrightarrow f^{-1}(x) = 
f^{-1}(x')))$.
However, according to the latter conclusion, $f^{-1}(x) = f^{-1}(x') \Longrightarrow x=x'$, a 
contradiction, so $f^{-1}$ must be injective.

And it is also surjective. $\forall x \in X$, $\exists y \in Y, f^{-1}(y)=x$. According to the former 
conclusion, $y$ is $f(x)$.

So now $f^{-1}$ is bijective, and thus has its inverse. By definition, $P(x,y)$ of 
$y = (f^{-1})^{-1}(x)$ is $f^{-1}(y) = x$, where $x \in X$. According to the former conclusion, 
$f^{-1}(f(x)) = x$. Thus
\[
f^{-1}(y) = f^{-1}(f(x)) \Longrightarrow y = f(x) \Longrightarrow (f^{-1})^{-1}(x) = f(x)
\], which is true $\forall x \in X$. And since they have the same domain and range, $(f^{-1})^{-1} = f$.
\end{proof}

\paragraph{Exercise 3.3.7} \label{exercise3.3.7}
\begin{proof}
Injectivity: 
\[
g \circ f(x) = g \circ f(x') \Longrightarrow f(x) = f(x') \Longrightarrow x = x'
\]

Surjectivity:
For each $z \in Z$, we need to find $x \in X$ such that $g \circ f(x) = z$. By the surjectivity of $g$, 
we can find $y \in Y$ such that $g(y) = z$. We can also find $a \in X$ such that $f(a) = y$ as $f$ is 
surjective. So $a$ is our desired $x$.

The $P(z,x)$ of $x = (g \circ f)^{-1}(z)$ is $z = g \circ f(x)$. Consider the following expression:
\begin{align*}
f^{-1} \circ g^{-1} (z)
&= f^{-1} \circ g^{-1} (g \circ f(x)) \\
&= f^{-1}(\textcolor{red}{g^{-1}(g(}f(x)\textcolor{red}{))}) \\
&= f^{-1}(f(x)) \\
&= x
\end{align*}
So $(g \circ f)^{-1} = f^{-1} \circ g^{-1}$. Therefore, they are equal as they have the same domain and 
range.
\end{proof}

\paragraph{Exercise 3.3.8} \label{exercise3.3.8}
\begin{proof}
(a) First they have the same domain and range. Finally, 
\[
\forall x(x\in X \Longrightarrow x=x \Longrightarrow \iota_{Y \rightarrow Z} \circ 
\iota_{X \rightarrow Y} = \iota_{X \rightarrow Z})
\]

(b) On one hand, they have the same domain and range.

On the other hand, 
\begin{align*}
f \circ \iota_{A \rightarrow A}(x) 
&= f(\iota_{A \rightarrow A}(x)) \\
&= f(x) \\
&= \iota_{B \rightarrow B}(f(x)) \\
&= \iota_{B \rightarrow B} \circ f (x)
\end{align*}

(c) It is easy to see that they have the same domain and range. 

\[
f \circ f^{-1} (b) = b = \iota_{B \rightarrow B}
\]
\[
f \circ f^{-1} (a) = a = \iota_{A \rightarrow A}
\]

(d) It is easy to see that they have the same domain and range.

Let $h$ be 
$h(x) = f(x)$, if $x \in X$, $h(x) = g(x)$, if $x \in Y$. 

For each $x \in X$, $\iota_{X \rightarrow X \cup Y}(x) = x$, so 
$h(\iota_{X \rightarrow X \cup Y}(x)) =f(x)$.

Similarly we can prove $h(\iota_{Y \rightarrow X \cup Y}(x)) = g(x)$ for each $x \in Y$.
\end{proof}

\section{Images and inverse Images}
\paragraph{Definition 3.4.1}
To prove that $f(S)$ is well-defined by using the axiom of specification, we need to apply it to set 
$Y$, not $X$. Let $P(y)$ be a property pertaining to each $y \in Y$, which is true iff 
$\exists x(x \in S \wedge f(x) = y)$. According to the axiom of specification, there exists a set that 
contains every $y \in Y$ such that $P(y)$ is true.

In some places where Tao asked ``(Why?)'', the reason is obvious, so I don't write them here. 

\paragraph{Example 3.4.6}
This is because 
\[
f^{-1}(f(\{-1,0,1,2\})) = f^{-1}(\{1,0,4\}) = \{-1,1,0,2,-2\}
\].

More generally, if $f$ (whose domain is $X$, and whose range is $Y$) is not injective, then 
\[
\exists x \exists x'((x \in X \wedge x' \in X) \wedge (x\neq x' \wedge f(x) = f(x')))
\]. 
Let $D \subseteq X$ such that $x \in D \wedge x' \notin D$. Then $f(x) = f(x') \in f(D)$. And thus 
$x,x' \in f^{-1}(f(D)) \Longrightarrow f^{-1}(f(D)) \neq D$. 

\paragraph{Exercise 3.4.1} \label{exercise3.4.1}
\begin{proof} 
$f^{-1}(V)$ may be interpreted in two different ways:

(1) Interpret $f^{-1}(V)$ as an inverse image, that is,
\[
(\forall x \in X)(x \in f^{-1}(V) \equiv f(x) \in V)
\]
\[
(\forall x \notin X)(x \notin f^{-1}(V))
\]

(2) Interpret $f^{-1}(V)$ as an image, where we regard $f^{-1}$ as a function. So, 
\[
\forall x(\exists y(y \in V \wedge x = f^{-1}(y)) \equiv x \in f^{-1}(V))
\]

We need to show that if the two statements are well-defined($x \in X$) , they are logically equivalent.

Let $S_1$ be the set defined in form (1), $S_2$ be the set defined in form (2). For every $x \in S_1$, 
$f(x) \in V$. Let $y' = f(x), x' = f^{-1}(y')$, then by definition (2) we have $x' \in S_2$. But 
$x' = x$, so $\forall x(x \in S_1 \Longrightarrow x \in S_2)$.

On the other hand, for every $a \in S_2$, $\exists b \in V$, such that $a = f^{-1}(b)$. Then $a \in X$. 
$f(a) = f(f^{-1}(b)) = b \in V$, so $a \in S_1$. Thus, $S_1 = S_2$.
\end{proof}

\paragraph{Exercise 3.4.2} \label{exercise3.4.2}
(1) Generally we can say $S \subseteq f^{-1}(f(S))$ but we cannot say that they are equal; (2) we can 
say $f(f^{-1}(U)) \subseteq U$ but we cannot say that they are equal.
\begin{proof}
(1) $x \in S \Longrightarrow f(x) \in f(S) \Longrightarrow x \in f^{-1}(f(S))$. However, it is possible 
that $\exists x(x \in X \wedge x \notin S \wedge f(x) \in f(S))$

(2) 
\[
x \in f^{-1}(U) \Longrightarrow f(x) \in U \Longrightarrow (y \in f(f^{-1}(U)) \Longrightarrow y \in U)
\]
However, it is still possible that 
\[
\exists y(y \in U \wedge \forall x(x \in X \Longrightarrow f(x) \neq y))
\]
\end{proof}

\paragraph{Exercise 3.4.3} \label{exercise3.4.3}
\begin{proof}
(1)
\[
x \in A \cap B \Longrightarrow f(x) \in f(A) \wedge f(x) \in f(B) \Longrightarrow 
f(x) \in f(A) \cap f(B)
\]
\[
y \in f(A \cap B) \equiv \exists x(x \in A \cap B \wedge f(x) = y)
\]
So $y \in f(A) \cap f(B)$, thus $f(A \cap B) \subseteq f(A) \cap f(B)$.

(2)
\[
x \in A \setminus B \Longrightarrow f(x) \in f(A\setminus B)
\]
\[
y \in f(A)\setminus f(B) \Longrightarrow \exists x(x \in A \wedge x \notin B \wedge f(x) = y)
\]
So $y \in f(A\setminus B)$, thus $f(A)\setminus f(B) \subseteq f(A\setminus B)$.

(3)
On one hand, 
\[
y \in f(A \cup B) \equiv \exists x (x \in A \cup B \wedge f(x) = y)
\]
\begin{align*}
x \in A \cup B \Longrightarrow x \in A \vee x \in B \Longrightarrow \\
f(x) \in f(A) \vee f(x) \in f(B) \Longrightarrow f(x) \in f(A)\cup f(B)
\end{align*}

On the other hand, 
\[
y \in f(A) \cup f(B) \Longrightarrow \exists x((x \in A \vee x \in B) \wedge f(x) = y) 
\]
\[
x \in A \vee x \in B \Longrightarrow x \in A \cup B \Longrightarrow f(x) \in f(A\cup B)
\]
\end{proof}

(1) $\subseteq$ can not be improved. Since it is possible that 
\[
\exists x \exists x'(x \in A \wedge x' \in B \wedge x \neq x' \wedge f(x) = f(x'))
\]

(2) $\subseteq$ can not be improved. Since it is possible that
\[
\exists x \exists x'(x \in A \setminus B \wedge x' \in B \wedge f(x) = f(x'))
\]

\paragraph{Exercise 3.4.4} \label{exercise3.4.4}
\begin{proof}
(1) 
\begin{align*}
x \in f^{-1}(U \cup V) \equiv (x \in X \wedge f(x) \in U \cup V) \equiv \\
(x \in X \wedge (f(x) \in U \vee f(x) \in V))
\end{align*}
\begin{align*}
x \in  f^{-1}(U) \cup  f^{-1}(V) \equiv (x \in X \wedge f(x) \in U) \vee (x \in X \wedge f(x) \in V) \\
\equiv (x \in X \wedge (f(x) \in U \vee f(x) \in V))
\end{align*}

(2) and (3) can be proven in similar manners. 
\end{proof}

\paragraph{Exercise 3.4.5} \label{exercise3.4.5}
\begin{proof}
(1) On one hand, if $f(f^{-1}(S)) = S$ for every $S \subseteq Y$, then \\
$f(f^{-1}(Y)) = Y$. That means, 
$y \in Y \Longrightarrow \exists x (x \in f^{-1}(Y) \wedge f(x) \in Y)$. So even $f^{-1}(Y)$ 
is enough for $f$ to be surjective. And $f^{-1}(Y) \subseteq X$, so $f$ is surjective.

On the other hand, if $f$ is surjective, then for each $S \subseteq Y$, 
\[
y \in S \Longrightarrow \exists x(x \in X \wedge f(x) = y)
\]
Such $x$ are elements of $f^{-1}(S)$ of course, so $f(f^{-1}(S)) = S$.

(2) On one hand, we show that $\forall S(S \subseteq X \Longrightarrow f^{-1}(f(S)) = S)$ implies that 
$f$ is injective. Suppose the contradiction, that when 
\[
\forall S(S \subseteq X \Longrightarrow f^{-1}(f(S)) = S)
\], but $f$ is not injective. Since $f$ is not injective, 
\[
\exists x \exists x'(x \in X \wedge x' \in X \wedge x \neq x' \wedge f(x) = f(x'))
\]
Let $S \subseteq X$ and $x \in S \wedge x' \notin S$. There is always such a set $S$.
For example, we can let $S = X\setminus \{x'\}$. So we have $f^{-1}(f(S)) \neq S$ because $x' \in$ it.

On the other hand, if $f$ is injective, then for every $S \subseteq X$, and for every $x \in S$, we have 
$f(x) \in f(S)$. And $f(x)$ are the only elements in $f(S)$, that is, 
$y \in f(S) \Longrightarrow y = f(x)$ for some $x \in S$. So now we know that 
$S \subseteq f^{-1}(f(S))$. Moreover, for every $x' \in f^{-1}(f(S))$, $f(x') \in f(S)$. We can let 
$f(x') = y = f(x)$. As $f$ is injective, $x = x'$, so $x' \in S$. That means $f^{-1}(f(S)) \subseteq S$. 
So $f^{-1}(f(S)) =  S$.
\end{proof}

\paragraph{Exercise 3.4.6} \label{exercise3.4.6}
\begin{proof}
My own proof: According to Axiom 3.10, we can construct the set $X^X$. Apply the axiom 
of replacement to each element of $X^X$, we construct a set $Z$ such that
\[
\forall x(x \in Z \equiv \exists f(f \in X^X \wedge x = f(X)))
\]

Let $Y = \{\varnothing\} \cup Z$.

Now we prove that $Y$ is the set we want. On one hand, for any $S \subseteq X$, 
if $S = \varnothing$, then $S \in Y$, as $Y = \{\varnothing\} \cup Z$.

If $S \neq \varnothing$, there exists a surjective function $g: X \rightarrow S$. $g\in X^X$, and 
$g(X) = S$, so $S \in Z$, and thus $S \in Y$. (To show the existence of $g$, for example, let $x \in X, 
g(x) = x$ if $x \in S$, and for $x \in X \wedge x \notin S$, $g(x)$ can be any element of $X$.)


On the other hand, for any $S' \nsubseteq X$, $\exists a(a \in S' \wedge a \notin X)$. To prove that 
$S' \notin Y$, we need to show that $\nexists f(f \in X^X \wedge S' = f(X))$. We know that 
$\nexists x(x \in X \wedge f(x) = a)$, so $a \notin f(X)$. Therefore $S' \neq f(X)$, so $S' \notin Y$.

$Y$ is the set we want.

I posted a question \href{https://math.stackexchange.com/questions/3803487/is-this-proof-to-the-existence-of-a-set-that-contains-all-subsets-of-another-set}{here} for verification for this proof. 
Thanks to answers of people at Stack Exchange so that my proof can be refined.

Proof By Tao's Hint:
For each $S \subseteq X$, let a function $f_S$ be $f_S(x) = 1$ if $x \in S$, and $f_S(x)=0$ if 
$x \in X \wedge x \notin S$. Then $f^{-1}_S(\{1\})$ gives $S$. 

Now we show that any element in 
$\{0,1\}^X$ is some $f_S$. Let $g \in \{0,1\}^X$. Then if $\forall x \in X, g(x) = 0$, then 
$g = f_{\varnothing}$. Otherwise, there exists a set that contains all $x$ such that $g(x) = 0$ by 
axiom of specification, namely $R$. Then $g = f_R$.

On the other hand, each $f_S$ is obviously an element of $\{0,1\}^X$.

Use the axiom of replacement, we construct a set $Y$ such that
\[
\forall x (x \in Y \equiv \exists f(f \in \{0,1\}^X \wedge x = f^{-1}(\{1\})))
\]

According to what we have proven, $Y$ is the set we want.
\end{proof}

\paragraph{Exercise 3.4.7} \label{exercise3.4.7}
\begin{proof}
As stated by the previous exercise, there exists a set $\mathbb{X}$ whose elements are all subsets of 
$X$, and a set $\mathbb{Y}$ whose elements are all subsets of $Y$.

For every element $x \in \mathbb{X}$, apply the axiom of replacement to $\mathbb{Y}$, to obtain a set 
$S_x := \{y^x\}$ for every element $y \in \mathbb{Y}$. 

According to the axiom of union, using $\mathbb{X}$ as the index set, we have the set
\[
Z = \bigcup_{x \in \mathbb{X}} S_x
\]

Apply again the axiom of union to $Z$ to obtain $R$, which contains all elements of elements of $Z$. Now 
we show that $R$ is the set we want.

On one hand, let $f$ be an arbitrary function with the domain of $X' \subseteq X$, and the range of $Y' 
\subseteq Y$. We can see that $f \in {Y'}^{X'} \in S_{X'}$. ${Y'}^{X'}$ becomes an element of $Z$. And 
thus $f$ becomes an element in $R$. 

On the other hand, from the construction of $R$, we can see that $R$ contains only these elements.

\end{proof}

\paragraph{Exercise 3.4.8} \label{exercise3.4.8}
\begin{proof}
Let $A,B$ be two arbitrary sets. They are also objects as stated by Axiom 3.1. So according to 
Axiom 3.3, there exists a set $S=\{A,B\}$. By Axiom 3.11, we have a set $Z$ such that 
\[
\forall x(x \in Z \equiv \exists X(X \in S \wedge x \in X))
\]

Now we show that $Z$ is the set we want. If $x \in A \vee x \in B$, then 
$\exists X(X \in S \wedge x \in X)$ is true. So $x \in Z$.

If $x \notin A \wedge x \notin B$, then $\forall X(X \in S \Longrightarrow x \notin X)$, that is, 
$\exists X(X \in S \wedge x \in X)$ is false. So $x \notin Z$.

$Z$ is therefore the set we want. 
\end{proof}

\paragraph{Example 3.4.11}
In (3.3), why do Tao choose some element $\beta$ of $I$? This is because we need to apply the axiom of 
specification to $A_\beta$ with the restriction $x \in A_\alpha$ for all $\alpha \in I$.

\paragraph{Exercise 3.4.9} \label{exercise3.4.9}
\begin{proof}
This is quiet easy to prove. Let the left-handed side set be $S$, the RHS set be $S'$. For any 
$x \in S$, $x \in A_\alpha$ for all $\alpha \in I$. So $x \in A_{\beta'}$. And $x \in A_\alpha$ for all 
$\alpha \in I$. Therefore $x \in S'$. 

It is nearly the same the prove $x \in S' \Longrightarrow x \in S$.
\end{proof}

\paragraph{Exercise 3.4.10} \label{exercise3.4.10}
\begin{proof}
For the sake of convenience, let 
$(\bigcup_{\alpha \in I} A_{\alpha})\cup(\bigcup_{\alpha \in J}A_{\alpha})$ be $S$, \\
$\bigcup_{\alpha \in I \cup J} A_{\alpha}$ be $S'$, 
$(\bigcap_{\alpha \in I} A_{\alpha})\cap(\bigcap_{\alpha \in J}A_{\alpha})$ be $Z$,
$\bigcap_{\alpha \in I \cup J} A_{\alpha}$ be $Z'$.

(1) When $I,J \neq \varnothing$: 
On one hand, 
\[
x \in S \Longrightarrow (x \in \bigcup_{\alpha \in I} A_{\alpha} \vee 
x \in \bigcup_{\alpha \in J}A_{\alpha})
\]
If $x \in \bigcup_{\alpha \in I} A_{\alpha}$, then $x \in \bigcup_{\alpha \in I \cup J} A_{\alpha}$.
If $x \in \bigcup_{\alpha \in J} A_{\alpha}$, then $x \in \bigcup_{\alpha \in I \cup J} A_{\alpha}$.

On the other hand, if $x \in S'$, then there exists an object $a \in I \cup J$ such that $x \in A_a$.
If $a \in I$ then $x \in x \in \bigcup_{\alpha \in I} A_{\alpha} \Longrightarrow x \in S$.
If $a \in J$ then $x \in x \in \bigcup_{\alpha \in J} A_{\alpha} \Longrightarrow x \in S$.

When $I,J$ are both empty, $S,S'$ are all empty.

When there is only one of $I,J$ is empty, say it is $I$, then 
$S = \varnothing \cup \bigcup_{\alpha \in J} = \bigcup_{\alpha \in J}$. And 
$S' = \bigcup_{\alpha \in \varnothing \cup J} A_{\alpha} = \bigcup_{\alpha \in J}$.

(2)
\[
x \in Z \equiv \forall a(a \in I \Longrightarrow x \in A_a) \wedge 
\forall b(b \in J \Longrightarrow x \in A_b)
\], which is equal to $\forall a(a \in I \cup J \Longrightarrow x \in A_a) \equiv x \in Z'$.
\end{proof}

\paragraph{Exercise 3.4.11} \label{exercise3.4.11}
\begin{proof}
(1) Let the LHS be $S$, the RHS be $S'$. 
\begin{align*}
x \in S &\equiv \\
x \in X \wedge x \notin \bigcup_{\alpha \in I} A_{\alpha} &\equiv \\
x \in X \wedge \forall a(a \in I \Longrightarrow x \notin A_{a})
\end{align*}
\begin{align*}
x \in S' &\equiv \\
\forall a(a \in I \Longrightarrow x \in X \setminus A_a) &\equiv \\
x \in X \wedge \forall a(a \in I \Longrightarrow x \notin A_a)
\end{align*}.

So $S=S'$.

(2) Let the LHS be $Z$, the RHS be $Z'$. 
\begin{align*}
x \in Z &\equiv \\
x \in X \wedge x \notin \bigcap_{\alpha \in I} A_{\alpha} &\equiv \\
x \in X \wedge \neg(\forall a(a \in I \Longrightarrow x \in A_a)) &\equiv \\
x \in X \wedge \exists a(a \in I \Longrightarrow x \notin A_a)
\end{align*}
\begin{align*}
x \in Z' &\equiv \\
x \in X \wedge \bigvee_{\alpha \in I}(x \notin A_{\alpha}) &\equiv \\
x \in X \wedge \exists a(a \in I \Longrightarrow x \notin A_a)
\end{align*}

Thus, $Z = Z'$
\end{proof}

\section{Cartesian products}
\paragraph{Exercise 3.5.1} \label{exercise3.5.1}
\begin{proof}
First we show that $(x,y) = \{\{x\},\{x,y\}\}$ is a good definition. 
Let $S_1$ denote $(x_1,y_1) = \{\{x_1\},\{x_1,y_1\}\}$, $S_2$ denote 
$(x_2,y_2) = \{\{x_2\},\{x_2,y_2\}\}$.

On one hand, if $x_1=x_2\wedge y_1=y_2$, then obviously $S_1=S_2$ for they have the same elements.

On the other hand, if $S_1 = S_2$, then 
\[
\{x_1\} \in S_2 \wedge \{x_1,y_1\} \in S_2 \wedge
\{x_2\} \in S_1 \wedge \{x_2,y_2\} \in S_1 
\].
We have that 
\begin{align*}
\{x_1\} \in S_2 &\equiv \{x_1\} = \{x_2\} \vee \{x_1\} = \{x_2,y_2\} \\
&\equiv x_1=x_2 \vee x_1=x_2=y_2 \\
&\Longrightarrow x_1=x_2
\end{align*}
\begin{align*}
\{x_1,y_1\} \in S_2 &\equiv \{x_1,y_1\} = \{x_2\} \vee \{x_1,y_1\} = \{x_2,y_2\} \\
&\equiv x_1=x_2=y_1 \\
&\vee \textcolor{red}{((x_1=x_2\wedge y_1=y_2)\vee(x_1=y_2\wedge y_1=x_2))} 
\end{align*}
Similarly we have that
\begin{align*}
\{x_2,y_2\} \in S_1 
&\equiv x_2=x_1=y_2 \\
&\vee \textcolor{red}{((x_2=x_1\wedge y_2=y_1)\vee(x_2=y_1\wedge y_2=x_1))}
\end{align*}
We may notice that the red-colored text are two same statements. Thus from $\{x_1,y_1\} \in S_2$ and 
$\{x_2,y_2\} \in S_1$ we can always conclude that $y_1=y_2$. Therefore, 
$S_1 = S_2 \Longrightarrow x_1=x_2\wedge y_1=y_2$.

Then we show that if $X,Y$ are two sets, then $X \times Y$ is also a set. For each element $x \in X$, 
construct a set $S_x$, where we replace each element $y \in Y$ with $(x,y)$. Then construct the set 
$\bigcup_{x \in X}S_x$.
\end{proof}

\paragraph{Exercise 3.5.2} \label{exercise3.5.2}
\begin{proof}
Since $x,y$ are two functions, they are equal means that $\forall 1\leq i \leq n$, $x(i) = y(i)$. That 
is, $x_i = y_i, 1\leq i \leq n$.

Now we show that $\displaystyle \prod_{1\leq i\leq n}X_i$ is a set. Let set $F$ be the set that contains 
all partial functions from $N = \{i \in \mathbb{N}:1\leq i\leq n\}$ to 
$\displaystyle X = \bigcup_{1\leq i\leq n}X_i$ (Exercise 3.4.7). Use the axiom of specification, select 
such elements $f$ from $F$ that:
\begin{enumerate}
\item the element is surjective, and
\item its domain is $N$, and 
\item $f(i) \in X_i$
\end{enumerate}, 
and use all of them to construct a set $Z$, which is the set we want.
\end{proof}

\paragraph{Exercise 3.5.3} \label{exercise3.5.3}
\begin{proof}
The definition is entirely based on the equality of objects (e.g. $x = x'$). The proof is immediately 
done since this equality is reflective ($x = x$), symmetric ($x = x' \equiv x' = x$), and transitive 
($x_0 = x_1 \wedge x_1 = x_2 \Longrightarrow x_0 = x_2$).
\end{proof}

\paragraph{Exercise 3.5.4} \label{exercise3.5.4}
\begin{proof}
(1)
\begin{align*}
(x,y) \in A \times (B \cup C) &\equiv x \in A \wedge y \in (B \cup C) \\
&\equiv x \in A \wedge (y \in B \vee y \in C) \\
&\equiv (x \in A \wedge y \in B) \vee (x \in A \wedge y \in C) \\
&\equiv ((x,y) \in A \times B) \vee ((x,y) \in A \times C) \\
&\equiv (x,y) \in (A \times B) \cup (A \times C)
\end{align*}

(2)
\begin{align*}
(x,y) \in A \times (B \cap C) &\equiv x \in A \wedge y \in (B \cap C) \\
&\equiv x \in A \wedge (y \in B \wedge y \in C) \\
&\equiv (x \in A \wedge y \in B) \wedge (x \in A \wedge y \in C) \\
&\equiv ((x,y) \in A \times B) \wedge ((x,y) \in A \times C) \\
&\equiv (x,y) \in (A \times B) \cap (A \times C)
\end{align*}

(3)
\begin{align*}
(x,y) \in A \times (B \setminus C) &\equiv x \in A \wedge y \in (B \setminus C) \\
&\equiv x \in A \wedge (y \in B \wedge \neg (y \in C)) \\
&\equiv (x \in A \wedge y \in B) \wedge \neg (x \in A \wedge y \in C) \\
\tag{The statement $x \in A$ implies $\neg (x \in A \wedge y \in C) 
\Longrightarrow \neg (y \in C)$}\\
&\equiv ((x,y) \in A \times B) \wedge \neg((x,y) \in A \times C) \\
&\equiv (x,y) \in (A \times B) \setminus (A \times C)
\end{align*}
\end{proof}

\paragraph{Exercise 3.5.5} \label{exercise3.5.5}
\begin{proof}
(1)
\begin{align*}
(x,y) \in (A \times B) \cap (C \times D) 
&\equiv (x,y) \in (A \times B) \wedge (x,y) \in (C \times D) \\
&\equiv (x \in A \wedge y \in B) \wedge (x \in C \wedge y \in D) \\
&\equiv (x \in A \wedge x \in C) \wedge (y \in B \wedge y \in D) \\
&\equiv x \in A \cap C \wedge y \in B \cap D \\
&\equiv (x,y) \in (A \cap C) \times (B \cap D)
\end{align*}

(2) It is not true since 
\begin{align*}
(x,y) \in (A \times B) \cup (C \times D) 
&\equiv (x,y) \in (A \times B) \vee (x,y) \in (C \times D) \\
&\equiv (x \in A \wedge y \in B) \vee (x \in C \wedge y \in D) \\
&\nLeftrightarrow (x \in A \vee x \in C) \wedge (y \in B \vee y \in D)
\end{align*}
Generally 
\[
(x \in A \wedge y \in B) \vee (x \in C \wedge y \in D) \Longrightarrow 
(x \in A \vee x \in C) \wedge (y \in B \vee y \in D)
\], 
but
\[
(x \in A \vee x \in C) \wedge (y \in B \vee y \in D) \nRightarrow
(x \in A \wedge y \in B) \vee (x \in C \wedge y \in D)
\].

(3) It is not true since
\begin{align*}
(x,y) \in (A \times B) \setminus (C \times D) 
&\equiv (x,y) \in (A \times B) \wedge (x,y) \notin (C \times D) \\
&\equiv (x \in A \wedge y \in B) \wedge (x \notin C \vee y \notin D) \\
&\nLeftrightarrow (x \in A \wedge x \notin C) \wedge (y \in B \wedge y \notin D)
\end{align*}
\end{proof}

\paragraph{Exercise 3.5.6} \label{exercise3.5.6}
\begin{proof}
(1) On one hand, if $A \subseteq C$ and $B \subseteq D$, then 
\begin{align*}
(x,y) \in A \times B &\equiv x \in A \wedge y \in B \\
&\Longrightarrow x \in C \wedge y \in D \\
&\Longrightarrow (x,y) \in C \times D
\end{align*}, 
which means $A \times B \subseteq C \times D$.

On the other hand, if $A \times B \subseteq C \times D$, but we suppose that 
\[
\neg(A \subseteq C \wedge B \subseteq D)
\]. 
We only consider that $A \nsubseteq C$, the other situations are similar. Then 
$\exists x(x \in A \wedge x \notin C)$. Let $p = (x,y)$, where $y \in B$, then $p \in A \times B$. 
But $x \notin C$, so $p \notin C \times D$, a contradiction. Therefore, 
\[
A \times B \subseteq C \times D \Longrightarrow A \subseteq C \wedge B \subseteq D
\]

(2) On one hand, if $A = C \wedge B = D$, then
\begin{align*}
(x,y) \in A \times B &\equiv x \in A \wedge y \in B \\
&\equiv x \in C \wedge y \in D \\
&\equiv (x,y) \in C \times D
\end{align*}.

On the other hand, if $A \times B = C \times D$, but we suppose that $\neg(A = C \wedge B = D)$. 
We only consider that $A \neq C$, the other situations are similar. Then we only consider 
$\exists x(x \in A \wedge x \notin C)$, for the other situations are similar. 

(3) It is easy to prove that $X \times \varnothing = \varnothing$ and 
$\varnothing \times X = \varnothing$. Let $A = \varnothing$, we can see that even if $B \nsubseteq D$, 
$A \times B \subseteq C \times D$. 

Let $A = D = \varnothing$, then even if $A \neq C$, $A \times B = C \times D$.
\end{proof}

\paragraph{Exercise 3.5.7} \label{exercise3.5.7}
\begin{proof}
Existence: Let $h(t):=(f(t),y(t))$. It is easy to verify that $h(t) \in X \times Y$, and that given a 
$t \in Z$, $h(t)$ is unique. Therefore, $h$ is a function. And it is obvious that 
$\pi_{X\times Y \rightarrow X} \circ h = f$ and that $\pi_{X\times Y \rightarrow Y} \circ h = g$.

Uniqueness: $\pi_{X\times Y \rightarrow X} \circ h = f$ and $\pi_{X\times Y \rightarrow Y} \circ h = g$ 
imply that if there is another function $h'$ that satisfies the requirements, then $h'(t) = h(t)$. So $h$ 
is unique. 
\end{proof}

\paragraph{Exercise 3.5.8} \label{exercise3.5.8}
\begin{proof}
On one hand, if for some $i, X_i = \varnothing$, then 
\[
\forall (x_i)_{1\leq i \leq n}(\bigwedge^{n}_{i =1}x_i \in X_i \equiv (x_i)_{1\leq i \leq n} \in 
\varnothing)
\], 
which means that $\varnothing = \prod_{i=1}^{n}X_i$.

On the other hand, if $\prod_{i=1}^{n}X_i = \varnothing$ but we suppose that $X_i \neq \varnothing$. Then 
for each $i$, $\exists x_i \in X_i$. We thus have a tuple $(x_i)_{1\leq i \leq n}$, which should be an 
element of $\prod_{i=1}^{n}X_i$. Therefore we have a contradiction.
\end{proof}

\paragraph{Exercise 3.5.9} \label{exercise3.5.9}
\begin{proof}
On one hand, let $x \in (\bigcup_{\alpha \in I}A_{\alpha})\cap(\bigcup_{\beta \in J}B_{\beta})$. Then 
\[
\exists a(a \in I \wedge x \in A_a) \wedge \exists b(b \in J \wedge x \in B_b)
\]
It is obvious that $x \in A_a \cap B_b$ and that $(a,b) \in I \times J$. Therefore 
\[
x \in \bigcup_{(\alpha,\beta) \in I \times J}(A_\alpha \cap B_\beta)
\].

On the other hand, let $x \in \bigcup_{(\alpha,\beta) \in I \times J}(A_\alpha \cap B_\beta)$. Then 
\begin{align*}
\exists (a,b) \in I \times J(x \in A_a \cap B_b) 
&\Longrightarrow  x \in A_a \wedge x \in B_b \\
&\Longrightarrow x \in \bigcup_{\alpha \in I}A_{\alpha} \wedge x \in \bigcup_{\beta \in J}B_{\beta} \\
&\Longrightarrow x \in (\bigcup_{\alpha \in I}A_{\alpha})\cap(\bigcup_{\beta \in J}B_{\beta})
\end{align*}
\end{proof}

\paragraph{Exercise 3.5.10} \label{exercise3.5.10}
\begin{proof}
We denote $\overset{\sim}{f}$ as $f'$, the graph of $f$ as $G$, and the graph of $f'$ as $G'$ for the 
sake of simplification.

(1) On one hand, if $f = f'$, then for every $(x,f(x)) \in G$, we can find $(x,f'(x)) \in G'$, and 
obviously $(x,f(x)) = (x,f'(x))$, and vice versa.

On the other hand, if $G = G'$, then for each $(x,f(x)) \in G$, $(x,f(x)) \in G'$. Note that each 
element of $G'$ obeys the form $(x,f'(x))$, so $f(x) = f'(x)$ for every $x \in X$, that is, $f=f'$.

(2) Existence: Let $f(x)$ be such a value that $(x,f(x)) \in G$. Thus the value is unique, so $f$ is a 
function. According to its definition, the graph of $f$ is $G$.

Uniqueness: As proven in (1), if $f,f'$ have the same graph, then they are equal.
\end{proof}

\paragraph{Exercise 3.5.11} \label{exercise3.5.11}
I think this exercise is meaningless. Lemma 3.4.6 is proven by the fact that $X^Y$ exists, which depends 
on Axiom 3.10. Then the exercise asks us to prove Axiom 3.10 using Lemma 3.4.6. So I looked up some books 
about set theory and found out that the power set axiom is essentially Lemma 3.4.6, not Axiom 3.10.

Nevertheless, here is the proof:
\begin{proof}
Let set $Z$ contains all subsets of $X\times Y$. The specify such element in $Z$ that obey the vertical 
line test, and let them form the set $S$. According to the previous exercise, for each element in $S$, 
there exists an unique function whose graph is the element. Then we replace all elements in $S$ with 
these functions to construct the set $F$. Obviously, each element in $F$ is a function with the domain 
$X$ and the range $Y$.

Now we show that every function $f$ from $X$ to $Y$ is in $F$. Denote the graph of $f$ as $G$. We know 
that $G$ obeys the vertical line test and $G \subseteq X \times Y$, so $G \in S$. Since $G$ is the graph 
of $f$, $f \in F$.
\end{proof}

\paragraph{Exercise 3.5.12} \label{exercise3.5.12}
I am confused by this exercise. It seems that simply applying induction to $a$ can solve the 
problem, just like what we did in Proposition 2.1.16. What is wrong?

By the way, according to the \href{https://terrytao.wordpress.com/books/analysis-i/}{corrections}, 
edit the exercise as the following:
\begin{quotation}
Let $X$ be an arbitrary set containing at least an element $c$ and obeys the Peano axioms. Let $f$ be a 
function from $N \times X$ to $X$. ... 

Show that there exists an unique function $a$ from $X$ to $X$ such that 
\[
a(0) = c
\] 
and 
\[
a(n++) = f(n,a(n)), \forall n \in X
\]
...

such that $a_N(0) = c$ and $a_{N}(n++) = f(n,a_N(n))$ ... 
\end{quotation}

Note that all properties (e.g. orders, addition) in section 2 are deduced from the Peano axioms and their 
definitions. Since $X$ obeys these rules, we use such properties on elements of $X$ without proof.

The proof is now reserved for further research.
\begin{proof}
\end{proof}

\paragraph{Exercise 3.5.13} \label{exercise3.5.13}
\begin{proof}
Use induction.

Existence: We need to prove that for all $n \in \mathbb{N}$, $f(n)$ is defined. Use induction:
$f(0) = 0'$ is define. And the definition is unique for $0$ is not the successor of any natural number. 
Now suppose that $f(n) = n'$ is defined, then $f(S(n)) = S'(f(n)) = S'(n')$ is also defined. The 
definition is also unique. So we know that $f$ exists.

Injectivity: We need to prove that $f(m) = f(n) \Longrightarrow m = n$. If $f(m) = f(n)$, then $m' = n'$, 
and thus $m=n$. 

Surjectivity: Use induction: 
The basic case is, for $0' \in \mathbb{N}'$, $f(0) = 0'$. 

Now suppose that for $n' \in \mathbb{N}'$, we can find $n \in \mathbb{N}$ such that $f(n) = n'$, then 
for $S'(n')$, we have $f(S(n)) = S'(n')$. We can close the induction now.
\end{proof}

\section{Cardinality of Sets}

\paragraph{Exercise 3.6.1} \label{exercise3.6.1}
\begin{proof}
Reflexivity: Let $f(x):= x, X \rightarrow X$. $f$ is bijective since $f^{-1}(x) = x$ exists.

Symmetry: If $X,Y$ have the same cardinality, then $\exists f:X\rightarrow Y$ which is bijective. So 
$f^{-1}$ exists, and is also a bijection. Thus $Y,X$ have the same cardinality. Since then, we can say 
that two sets have the same cardinality without caring about the order.

Transitivity: If $X,Y$ have the same cardinality, and $Y,Z$ also have the same cardinality, then there 
exist two bijections: $f:X \rightarrow Y$ and $g:Y \rightarrow Z$. It is easy to verify that $g \circ f$ 
is also a bijection and is from $X$ to $Z$ (See \exerciseref{3.3.7}).
\end{proof}

\paragraph{Remark 3.6.6}
It is $f(n) := S(n)$. We are now proving something stronger
\begin{lem} \label{lem3.6.6}
For any natural number $m,n$, $\{i \in \mathbb{N}:0\leq i\leq n\}$ and 
$\{i \in \mathbb{N}:m\leq i\leq n+m\}$ have the same cardinality.
\end{lem}
\begin{proof}
Use induction on $m$. When $m=0$, the statement is obviously true. Simply give the function $f(n):=n$.

Suppose that for some $m$, we have proven the statement. Then there exists a bijection: 
\[
f:\{i \in \mathbb{N}:0\leq i\leq n\} \rightarrow \{i \in \mathbb{N}:m\leq i\leq n+m\}
\].
Let $g$ be a function from $\{i \in \mathbb{N}:0\leq i\leq n\}$ to $\mathbb{N}$ such that 
$g(x) = S(f(x))$. We prove that $g$ is a bijection from $\{i \in \mathbb{N}:0\leq i\leq n\}$ to 
$\{i \in \mathbb{N}:S(m)\leq i\leq n+S(m)\}$.

First we prove that $g(n)$ always in $\{i \in \mathbb{N}:S(m)\leq i\leq n+S(m)\}$, which is immediately 
given by the fact that addition preserves order. 

Surjectivity: For any $a \in \{i \in \mathbb{N}:S(m)\leq i\leq n+S(m)\}$, $a$ is positive. Then $a$ is 
always some number's successor, that is $a = S(b) = b+1$ for some natural number $b$. Since addition 
preserves order, $b \in \{i \in \mathbb{N}:m\leq i\leq n+m\}$. $f$ being surjective implies that there is 
some $x$ in the domain such that $f(x) = b$, and $g(x) = f(x) + 1 = a$.

Injectivity: By cancellation law, $f(x) + 1 \neq f(x') + 1 \equiv f(x) \neq f(x') \equiv x \neq x'$.

We can now close the induction.
\end{proof}

\paragraph{Lemma 3.6.9}
Empty functions are not injective when the range is not empty (See \exerciseref{3.3.3}). 

Now we show that $g$ is bijective:
\begin{proof}
Injectivity: $f$ being injective implies that 
\[
\forall x \forall x'((x \in X \wedge x' \in X) \Longrightarrow (f(x) = f(x') \Rightarrow x = x'))
\]
For $a,a' \in X - \{x\}$, they also $\in X$. If $g(a) = g(a')$, then either directly $f(a) = f(a')$ or 
$f(a) - 1 = f(a') - 1$, which gives $f(a) = f(a')$. Thus $a = a'$. (Note that subtraction is not defined 
yet, see the footnote about this in the book).

Surjectivity: The surjectivity of $f$ gives 
\[
(\forall 1 \leq i \leq n)(\exists a(a \in X \wedge f(a) = i))
\].

If $f(x) = n$, then $g(a) = f(a)$ for all meaningful $a$. Then for $1 \leq i \leq n-1$, we can find $a$ 
such that $a \in X \wedge a \neq x$, that is, $x \in X - \{x\}$. So $g(a)$ is meaningful, then $g$ is 
surjective.

If $f(x) \neq n$, then $f(x) < n$. For those $1 \leq i < f(x)$, $g$ is obviously surjective. For 
$n-1 \geq i \geq f(x)$, since $S(i) \leq n$, $\exists a(a \in X \wedge f(a) = S(i))$. And we know that 
$S(i) \neq f(x)$, then $a \in X - \{x\}$. So $g(a) = f(a) - 1 = i$.
\end{proof}

\paragraph{Exercise 3.6.2} \label{exercise3.6.2}
\begin{proof}
On one hand, if $X$ is empty, then we know that the empty function whose range is also empty is injective, 
(See \exerciseref{3.3.3}) so its cardinality is $0$. 

On the other hand, if $\# X = 0$ but $X \neq \varnothing$, then there exists an bijection 
$f:X \rightarrow \varnothing$, which is impossible.
\end{proof}

\paragraph{Exercise 3.6.3} \label{exercise3.6.3}
\begin{proof}
When $n = 0$, this is vacuously true. The base case then becomes $n=1$. We simply let $M = f(1)$.

Suppose that the statement for $n$ is true. And for $1\leq i\leq n$ we have the number $M$. Then $f(S(n))$ 
either $\geq$ or $<$ $M$. On the former case, let $f(S(n))$ be $M'$, and on the latter case, let $M' = M$. 
It is east to verify that $M'$ is the number we want.
\end{proof}

From now on we will denote $\{i\in \mathbb{N}:1\leq i \leq n\}$ as $\mathbb{N}_n$

\paragraph{Exercise 3.6.4} \label{exercise3.6.4}
\begin{proof}
(a) 
Let $n = \#X$. There is an injective  $f$ from $X$ to $\{i\in \mathbb{N}:1\leq i\leq n\}$. Let $g$ be a 
function from $X \cup \{x\}$ to $\{i\in \mathbb{N}:1\leq i\leq n+1\}$ such that $g(a) = f(a)$ if 
$a\neq x$, and $g(x) = n+1$. Now we show that $g$ is bijective.

Injectivity: We know that $\forall x \in X$, $g$ is already injective. Since that $g(x) = n+1 \neq g(a)$ 
for all $a \in X$, so $g$ is injective on $X \cup \{x\}$.

Surjectivity: We know that $\forall i \in \{i\in \mathbb{N}:1\leq i\leq n\}$, we can find 
$a \in X \cup \{x\}$ such that $g(a) = i$. And we have $g(x) = n+1$, so 
$\forall a \in \{i\in \mathbb{N}:1\leq i\leq n+1\}$, we can find $a \in X \cup \{x\}$ such that 
$g(a) = i$. 

(b)
First we prove that if $X,Y$ are disjoint, then $\#X + \#Y = \#(X\cup Y)$. Let $f$ be a bijection from $X$ 
to $\mathbb{N}_{\#X}$, and $g$ be a bijection from $Y$ to $\mathbb{N}_{\#Y}$. According to 
\hyperref[lem3.6.6]{this Lemma}, there exists a bijection $h$ from $\mathbb{N}_{\#Y}$ to 
$\{i\in \mathbb{N}:\#X+1\leq i \leq \#X+\#Y\}$. Thus $h \circ g$ is also a bijection. Let $u$ be a 
function from $X \cup Y$ to $\mathbb{N}_{\#X} \cup\{i\in \mathbb{N}:\#X+1\leq i \leq \#X+\#Y\}$. Now we 
show that $u$ is bijective.

Injectivity: For $x \neq x'$ in the domain. If $x,x'$ are both in $X$ or $Y$, then $f(x)\neq f(x')$ is 
immediately given by the injectivity of $f$ and $h \circ g$. If one of them is in $X$, and the other is 
in $Y$, then they can also never be equal because the ranges of the two functions are disjoint. 

Surjectivity: It is easy to verify that the range is equal to $\mathbb{N}_{\#X + \#Y}$. For any $y$ in the 
range, if $y \in$ the range of $f$, then $u$ is surjective since $f$ is, and if $y \in$ the range of $h 
\circ g$, $u$ is surjective for the same reason. The range consists of only this two sets, so $u$ is 
surjective on the whole range.

The proof is over. This also implies that $X \cup Y$ is finite. Now we need only to show that 
$\#(X \cup Y) < \#X + \#Y$ when $X,Y$ are not disjoint. It is easy to see that
\begin{align*}
\#A + \#B 
&= \#(A - A \cap B) + \#(A \cap B) + \#(B - A \cap B) + \#(A \cap B) \\
&= (\#(A - A \cap B) + \#(A \cap B) + \#(B - A \cap B)) + \#(A \cap B) \\
&= \#(A \cup B) + \#(A \cap B) \\
&> \#(A \cup B)
\end{align*}

(c)
If $X \subseteq Y \wedge X \neq Y$, then $\#(Y \setminus X) \neq 0$. 
\[
\#Y = \#X + \#(Y \setminus X) > \#X
\].

If $X = Y$, then $\#(Y \setminus X) = 0$, and $\#Y$ becomes $\#X$.

(d)
$f: X \rightarrow f(X)$ is always surjective. If $f$ is also injective, then $f$ is bijective. On this 
occasion, $\#f(X) = \#X$. If $f$ is not injective, we can select a set $X' \subseteq X \wedge X' \neq X$, 
on which $f$ is bijective. Then $\#X' = \#f(X') = \#f(X)$. According to (c), $\#X' < \#X$, so 
$\#f(X) < \#X$.

(e)
Suppose that $\#Y = n$. Use induction on $n$. 

When $n=0$, $Y$ is empty, then $\#(X \times Y) = 0 = \#X \times 0$. Here we additionally prove that 
when $n=1$, this is also true for further usage. When $n=1$, let $Y = \{a\}$. Then the bijection is 
$f(x):=(x,a), X \rightarrow X \times \{a\}$.

Suppose that we have proven for some $n$, $\#(X \times Y) = \#X \times \#Y$. Then when $\#Y = S(n)$, 
let $Y = Y\setminus\{x\}\cup\{x\}$, where $x \in Y$. Lemma 3.6.9 tells us that 
$\#(Y\setminus\{x\}) = S(n)-1 = n$. And \exerciseref{3.5.4} tells us that 
$X \times Y = X \times (Y\setminus\{x\}) \cup X \times \{x\}$. 
\begin{align*}
\#(X \times Y) 
&= \#(X \times (Y\setminus\{x\}) \cup X \times \{x\}) \\
&= \#(X \times (Y\setminus\{x\})) + \#(X \times \{x\}) \\
&= \#X \times n + \#X \\
&= \#X \times S(n)
\end{align*}

We can now close the induction.

(f)
We should first define $m^n$ for natural numbers $m,n$. 
\begin{definition}

\begin{itemize}
\item $m^0=1$,
\item $m^{S(n)} = m^n \times m$
\end{itemize}
\end{definition}

Suppose that $\#Y = m,\#X = n$. Use induction on $n$. 

When $n=0$, $X$ is empty, then $Y^X$ has one function $f:\varnothing \rightarrow Y$.

Suppose that we have proven the statement for some $n$. Before we proceed the proof, we need some lemmas.
\begin{lem}
If $X$ is not empty, 
\[
\#Y^{X\setminus\{x'\}\cup\{x'\}} = \#Y^{X\setminus\{x'\}} \times \#Y
\], 
where $x'$ is an element of $X$.
\end{lem}
\begin{proof}
By (e) we know that 
\[
\#Y^{X\setminus\{x'\}} \times \#Y = \#(Y^{X\setminus\{x'\}} \times Y)
\].

Try to build a bijection between $Y^{X\setminus\{x'\}} \times Y$ and $Y^X$. Let $f' \in Y^X$.

Let $h$ be a function from $Y^X$ to $Y^{X\setminus\{x'\}} \times Y$ such that
\[
h(f') = (f,f'(x')), 
\]
where $f(x):=f'(x)$ when $x \neq x'$. Now we show that $h$ is bijective.

Injectivity: 
If ${f_1}' \neq {f_2}'$, then 
\[
{f_1}'(x') \neq {f_2}'(x') \vee \exists x(x \neq x' \wedge {f_1}'(x) \neq {f_2}'(x))
\]
That is, 
\[
{f_1}'(x') \neq {f_2}'(x') \vee f_1 \neq f_2,
\]
which means 
\[
(f_1,{f_1}'(x')) \neq (f_2,{f_2}'(x')).
\]

Surjectivity:
For any $(f,a) \in Y^{X\setminus\{x'\}} \times Y$, let $f'$ be $f$ if $x\neq x'$, and $f'(x') = a$. Then 
$f' \in Y^X$ and $h(f') = (f,a)$.

So, 
\[
\#Y^X = \#(Y^{X\setminus\{x'\}} \times Y)
\], which gives the lemma.
\end{proof}

Now we proceed the proof. Suppose that $\#X = n+1$, then $\#(X \setminus\{x'\}) = n$. By induction 
hypothesis, $\#(Y^{X \setminus\{x'\}}) = m^n$. 

By the lemma, 
\[
\#Y^X = \#Y^{X\setminus\{x'\}\cup\{x'\}} = \#Y^{X\setminus\{x'\}} \times \#Y,
\]
which equals to $m^n \times m$.

Now we can close the induction.

We have proven that the cardinality of power sets obeys the definition of power. This ensures the 
exercise.
\end{proof}

\paragraph{Exercise 3.6.5} \label{exercise3.6.5}
\begin{proof}
Let $f((x,y)):= (y,x), A\times B \rightarrow B \times A$. The bijectivity is obvious. 

Now we are using set theory to prove the commutativity of multiplication of natural number. For any 
natural number $m,n$, construct two sets: $M = \mathbb{N}_{m}, N = \mathbb{N}_{n}$. According to (e) in 
Proposition 3.6.14, we have that $\#(M \times N) = \#M \times \#N$. Then by what we have just proven, 
\[
\#(M \times N) = \#(N \times M) \Longrightarrow \#M \times \#N = \#N \times \#M \Longrightarrow mn = nm
\]
\end{proof}

\paragraph{Exercise 3.6.6} \label{exercise3.6.6}
\begin{proof}
Let $c \in C$, $f \in (A^B)^C$. Then $f(c)$ is a function $B\rightarrow A$. Let 
$b \in B, h \in A^{B\times C}$. Let
\[
g:A^{B\times C} \rightarrow (A^B)^{C}
\]
be such a function that for all $b,c$,
\[
g(h) = f \equiv h(b,c) = (f(c))\,(b)
\]
. Now we show that $g$ is bijective.

Injectivity: 
If $h \neq h'$, then $\exists b_0,c_0(h(b_0,c_0) \neq h'(b_0,c_0))$. Let $g(h) =f, g(h') = f'$. Then we 
know that $(f(c_0))\,(b_0) \neq (f'(c_0))\,(b_0)$, so $f(c_0) \neq f'(c_0) \Longrightarrow f \neq f'$. 
That means, $g(h) \neq g(h')$.

Surjectivity:
For any $f \in (A^B)^{C}$, let $h$ be such a function $\in A^{B\times C}$ that for all $b \in B,c \in C$, 
$h(b,c) := (f(c))\,(b)$. It is easy to see that $h$ is well-defined. So $g(h) = f$.

Note that by Proposition 3.6.14 we have $\#M^N = m^n$ and $\#(M \times N) = mn$, where $\#M =m, \#N =n$. 
Suppose that $\#A = a, \#B = b, \#C = c$, then
\[
\#(A^B)^C = (\#A^B)^{\#C} = (a^b)^c
\]
\[
\#A^{B\times C} = \#A^{\#(B \times C)} = a^{bc}
\]
So we have proven that $(a^b)^c = a^{bc}$.

Now we try to prove $a^b \times a^c = a^{b+c}$. Let $B,C$ be disjoint sets with the cardinality $b,c$ 
respectively. What we need to show is that
\[
\#(A^B \times A^C) = \#(A^{B \cup C}).
\]

Similarly, let 
\[
f: (A^{B \cup C}) \rightarrow (A^B \times A^C)
\]
be such a function that 
\[
f(g) = (u,v) \equiv \forall x(x \in B \Rightarrow g(x) = u(x) \wedge x \in C \Rightarrow g(x) = v(x)),
\]
where $g\in A^{B \cup C}, (u,v) \in A^B \times A^C$.

We can verify the bijectivity of $f$ nearly in the same way as way did previously. So I won't write it 
down here.

Then, we know $B \cap C = \varnothing \Rightarrow \#(B \cup C) = \#B + \#C$. So we can conclude that 
\[
a^b \times a^c = a^{b+c}
\]
\end{proof}

\paragraph{Exercise 3.6.7} \label{exercise3.6.7}
\begin{proof}
On one hand, if $\#A = a \leq \#B = b$, we show that $A$ has lesser or equal cardinalty to $B$. Let $f$ 
be a bijection from $A$ to $\mathbb{N}_{a}$, $g$ be a bijection from $B$ to $\mathbb{N}_b$. Let 
$\iota(x):=x, \mathbb{N}_{a} \rightarrow \mathbb{N}_b$. Then $g^{-1} \circ \iota \circ f$ is an injection 
from $A$ to $B$.

On the other hand, suppose that there is an injection $f$ from $A$ to $B$. We know that 
$f:A\rightarrow f(A)$ is bijective. So $\#A = \#f(A)$. Since $f(A) \subseteq B$, $\#f(A) \leq B$ (See (c) 
in Proposition 3.6.14). That is, $\#A \leq \#B$
\end{proof}

\paragraph{Exercise 3.6.8} \label{exercise3.6.8}
\begin{proof}
$f:A \rightarrow f(A)$ is bijective. So $f^{-1}: f(A) \rightarrow A$ is surjective. Let $g$ be defined as:
\begin{itemize}
\item $b \in f(A) \Longrightarrow g(b) = f^{-1}(b)$
\item $b \in B\setminus f(A) \Longrightarrow g(b)$ is any element of $A$.
\end{itemize}

Then $g$ is surjective.
\end{proof}

\paragraph{Exercise 3.6.9} \label{exercise3.6.9}
\begin{proof}
\begin{align*}
\#A + \#B 
&= \#(A - A \cap B) + \#(A \cap B) + \#(B - A \cap B) + \#(A \cap B) \\
&= (\#(A - A \cap B) + \#(A \cap B) + \#(B - A \cap B)) + \#(A \cap B) \\
&= \#(A \cup B) + \#(A \cap B)
\end{align*}
\end{proof}

\paragraph{Exercise 3.6.10} \label{exercise3.6.10}
\begin{proof}
Presume the contradiction:
\[
\forall i(i \in \{1,\dots,n\} \Longrightarrow \#(A_i) < 2)
\]

Use mathematical induction for (b) in Proposition 3.6.14, we can easily get:
\[
\#\bigcup_{i \in \{1,\dots,n\}}A_i \leq \sum_{i \in \{1,\dots,n\}} \#A_i
\]
We can also use mathematical induction to furthermore enhance what we proved while dealing with natural 
numbers to:
\[
\bigwedge_{i} a_{i} \leq b_{i} \Longrightarrow \sum_{i} a_{i} \leq \sum_{i} b_i
\].

Then because $\# A_i \leq 1$, so 
\[
\sum_{i \in \{1,\dots,n\}} A_i \leq (\sum_{i \in \{1,\dots,n\}} 1 = n)
\], 
which is impossible.
\end{proof}