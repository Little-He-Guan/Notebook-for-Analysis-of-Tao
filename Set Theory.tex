\part{Set Theory}

\section{Fundamentals}
\paragraph{Exercise 3.1.1} \label{exercise3.1.1}
\begin{proof}
Reflexive: $\forall x \in S, x \in S$.

Symmetric: 
\begin{align*}
X = Y
&\Longleftrightarrow \\
\forall x \in X, x \in Y \wedge \forall x \in Y, x \in X
&\Longleftrightarrow \\
Y = X
\end{align*}

Transitive:
$X=Y \Longrightarrow \forall x \in X, x \in Y$. Because $x \in Y$ and $Y = Z$, we can conclude that 
$\forall x \in X, x \in Z$. Conduct the process from inversely, we can get $\forall x \in Z, x \in X$. 
Therefore, $X=Z$.
\end{proof}

The reason for the content beneath Axiom 3.2 is clearly demonstrated in the proof of Lemma 3.1.6.

In Remarks 3.1.9, there are three ``Why''s. The reason can be concluded as: Because of the ``if and only 
if'' in Axiom 3.3, or more precisely, ``only if'', if $x$ is a element in one of such sets, $x$ must 
$=a$ or $b$. And because of the ``if'', $x$ is thus in another set. So the two sets are equal according 
to Definition 3.1.4.

\paragraph{Exercise 3.1.2} \label{exercise3.1.2}
\begin{proof}
According to Axiom 3.2, $\varnothing$ exists, and is thus an object as stated by Axiom 3.1. Therefore, 
by Axiom 3.3, $\{\varnothing\}$ also exists. $\varnothing$ is an element of $\{\varnothing\}$, but it 
is not an element of $\varnothing$ because any object $\notin$ $\varnothing$.

For the same reason, any set that contains element(s) is not the same set as $\varnothing$. Furthermore, 
there exists an object $\{\varnothing\}$ (Axiom 3.3 and 3.1), which is an element of 
$\{\varnothing, \{\varnothing\}\}$, but which is not an element of $\{\varnothing\}$. So the two sets 
are not equal.
\end{proof}

\paragraph{Remarks 3.1.12}
\begin{proof}
Let $x \in A'\cup B$. $x \in A' \Longrightarrow x \in A$ And if $x \notin A'$, $x \in B$. So either way 
$x \in A\cup B$ and vice versa.
\end{proof}

\paragraph{Exercise 3.1.3} \label{exercise3.1.3}
\begin{proof}
(1)
\[
x \in A \cup B \equiv (x \in A \vee x \in B)
\]
\[
x \in A \Longrightarrow x \in B \cup A
\]
\[
x \in B \Longrightarrow x \in B \cup A
\]
So $x \in A \cup B \Longrightarrow x \in B \cup A$. And vice versa.

(2)
$x \in A \Rightarrow x \in A \cup A$ and $x \in A \cup A \Rightarrow x \in A$.

(3)
\begin{align*}
x \in A \cup \varnothing 
&\Longrightarrow \\
x \in A \vee x \in \varnothing
&\Longrightarrow \\
x \in A \tag{$\forall a, a \notin \varnothing$}
\end{align*}

And obviously $x \in A \Rightarrow x \in A \cup \varnothing$. So $A \cup \varnothing = A$.

By transitivity of equality, and commutativity of pairwise union, we can conclude the others.
\end{proof}

\paragraph{Examples 3.1.17}
\begin{proof}
\[
\forall x(x\in A \Longrightarrow x \in A)
\]

And
\[
\forall x(x \in \varnothing \Longrightarrow x \in A)
\]
is vacuously true.
\end{proof}

\paragraph{Exercise 3.1.4} \label{exercise3.1.4}
\begin{proof}
(1) 
On one hand,
\[
A \subseteq B \equiv \forall x(x \in A \Longrightarrow x \in B)
\]. 
On the other hand,
\[
B \subseteq A \equiv \forall x(x \in B \Longrightarrow x \in A)
\].
Thus $A=B$.

(2)
First, we prove that $A \subsetneq B \Longrightarrow \exists x(x \in B \wedge x \notin A)$. Suppose the 
contradiction, that is, $\forall x(x\in B \Longrightarrow x \in A)$, which is impossible since 
$(A \subseteq B \equiv \forall x(x\in A \Longrightarrow x \in B))\wedge A \neq B$.

According to what's proven in the book, 
$A \subsetneq B \wedge B \subsetneq C \Longrightarrow A \subseteq C$.

Now we prove that $\exists x(x \in C \wedge x \notin A)$. Since $x \in A \Longrightarrow x \in B$, 
$x \notin B \Longrightarrow x \notin A$. Because $B \subsetneq C$, 
$\exists x(x \in C \wedge x \notin B)$, and thus for such $x$, $ x \notin A$. Then $A \neq C$.

So $A \subsetneq C$. 
\end{proof}

\paragraph{Axiom 3.5}
(1) Because $x \in \{x \in A : P(x)\} \Rightarrow x \in A$.

(2) Because both $\in$ and $P(x)$ obey the axiom of substitution.

\paragraph{Exercise 3.1.5} \label{exercise3.1.5}
\begin{proof}
First we prove that $A \subseteq B \equiv A \cup B = B$. On one hand,
\begin{align*}
A \subseteq B 
&\equiv \\
\forall x(x \in A \Longrightarrow x \in B)
&\Longrightarrow \\
\forall x((x \in A \vee x \in B) \Longrightarrow x \in B)
&\equiv \\
A \cup B = B
\end{align*}.

On the other hand,
\[
\forall x((x \in A \vee x \in B) \Longrightarrow x \in B) \Longrightarrow
\forall x(x \in A \Longrightarrow x \in B)
\]. The statement is therefore proven.

Then we prove that $A \subseteq B \equiv A \cap B = A$. On one hand, 
\begin{align*}
(A \cap B = A \equiv \forall x(x \in A \wedge x \in B \equiv x \in A)) 
&\Longrightarrow \\
(\forall x(x \in A \Rightarrow x \in B) \equiv (A \subseteq B))
\end{align*}.

On the other hand, 
\[
\forall x(x\in A\wedge x \in B \Longrightarrow x \in A)
\]
is always true (Vacuously true if $x \notin B$).

Logical equality is transitive, and thus all of the three statements are equal.
\end{proof}

\paragraph{Proposition 3.1.28} (Exercise 3.1.6) \label{exercise3.1.6}
\begin{proof}
(a) The two are identical to 
\[
\forall x(x \in A \vee x \in \varnothing \equiv x \in A)
\], 
and 
\[
\nexists x(x \in A \wedge x \in \varnothing)
\], 
which are all true since $\forall x(x \notin \varnothing)$.

(b) We have $A \subseteq X$. According to what we have proven in 
\hyperref[exercise3.1.5]{Exercise 3.1.5}, the two statements are all true.

(c) Obvious since 
\[
\forall x(x \in A \vee x \in A \equiv x \in A)
\]
and 
\[
\forall x(x \in A \wedge x \in A \equiv x \in A)
\]

(d) All true since \emph{logical or} and \emph{logical and} are commutative.

(e) See Lemma 3.1.13. I believe that this can be concluded by the fact that \emph{logical or} and 
\emph{logical and} are also associative.

(f) 
First we prove the latter. On one hand, suppose 
\[
x \in A \cup (B \cap C)
\] is ture.

If $x \in A$, then $x \in$ both $A \cup B$ and $A \cup C$, and thus $\in$ 
$(A \cup B)\cap(A \cup C)$.

If $x \notin A$, then $x \in B \cap C$, then $x \in$ both $A \cup B$ and $A \cup C$, and thus $\in$ 
$(A \cup B)\cap(A \cup C)$.

On the other hand, 
suppose 
\[
x \in (A \cup B)\cap(A \cup C)
\] is true.

If $x \in A$, obviously $x \in A \cup (B \cap C)$.

If $x \notin A$, then $x$ must $\in B \cap C$, and thus also $\in A \cup (B \cap C)$.

Now we prove the former. On one hand, suppose
\[
x \in A \cap (B \cup C)
\] is true.

If $x \in A \wedge x \in B$, then $x \in A \cap B$, and thus $\in (A \cap B)\cup(A \cap C)$.

If $x \notin A \vee x \notin B$, then
\begin{enumerate}
\item if $x \notin A$, this is impossible.
\item if $x \in A$, then $x \notin B$. But $x \in B \cup C$, so $x \in C$. And thus 
$x \in A \cap C \Rightarrow x \in (A \cap B)\cup(A \cap C)$.
\end{enumerate}

On the other hand, suppose that 
\[
x \in (A \cap B)\cup(A \cap C)
\] is true.

First we can see that $x \in A$. 

If $x \in B$, then $x \in B \cup C$, and thus $\in A \cap (B \cup C)$.

If $x \notin B$, then $x \in C$. So $x \in B \cup C$, and thus $\in A \cap (B \cup C)$.

(g)
Now we prove the former: On one hand, suppose that 
\[
x \in A \cup (X-A)
\].

If $x \in A$, then $x \in X$ since $A \subseteq X$.

If $x \notin A$, then $x \in X-A$, and thus also $\in X$.

On the other hand, suppose that 
\[
x \in X
\]

If $x \in A$, then $x \in A \cup (X-A)$.

If $x \notin A$, then $x \in X-A$, and thus $\in A \cup (X-A)$.

(h)
$x \in X - A$ requires $x \notin A$. So $\forall x(x \in A \cap (X-A))$ is always false. 
Thus
\[
\forall x(x \in A \cap (X-A) \Longleftrightarrow x \in \varnothing)
\] (vacuously true).
\end{proof}

\paragraph{Exercise 3.1.7} \label{exercise3.1.7}
\begin{proof}
(1)
$\forall x(x \in A \cap B \Longrightarrow x \in A)$. Similarly, we can prove that 
$A \cap B \subseteq B$. (This can also be achieved via the commutativity).

(2) 
On one hand, suppose that 
\[
C \subseteq A \wedge C \subseteq B
\] is true.
Then, 
\[
\forall x(x \in C \Longrightarrow x \in A \wedge x \in B \Longrightarrow x \in A \cap B)
\].

On the other hand, suppose that
\[
C \subseteq A \cap B
\] is true.
Then, 
\[
\forall x(x \in C \Longrightarrow x \in A \wedge x \in B)
\].
That is, $C \subseteq A \wedge C \subseteq B$.

(3) It is immediately given by 
\[
\forall x(x \in A \Longrightarrow x \in A \cup B)
\]. 
Since $\cup$ is commutative, the latter case is proven.

(4) On one hand, suppose that $A \subseteq C \wedge B \subseteq C$ and let $x \in A \cup B$.

If $x \in A$, then $x \in C$.

If $x \notin A$, then $x \in B$, and thus $x \in C$.

On the other hand, suppose that $A \cup B \subseteq C$. Then, 
\[
\forall x(x \in A \Longrightarrow x \in A \cup B \Longrightarrow x \in C)
\]
\[
\forall x(x \in B \Longrightarrow x \in A \cup B \Longrightarrow x \in C) \qedhere
\].
\end{proof}

\paragraph{Exercise 3.1.8} \label{exercise3.1.8}
\begin{proof}
The former: On one hand, 
Suppose that 
\[
x \in A \cap (A \cup B)
\].

If $x \in A$, then $x \in A$.

If $x \notin A$, this is impossible.

On the other hand, suppose that $x \in A$.
Then $x \in A \wedge x \in (A \cup B)$, so $x \in A \cap (A \cup B)$.

The latter: On one hand, suppose that $x \in A \cup (A \cap B)$.
\[
x \in A \Longrightarrow x \in A.
\]
\[
x \notin A \Longrightarrow x \in (A \cap B) \Longrightarrow x \in A
\].

On the other hand, Suppose that $x \in A$, then $x \in A \cup (A \cap B)$.
\end{proof}

\paragraph{Exercise 3.1.9} \label{exercise3.1.9}
\begin{proof}
\begin{lem} \label{lem10}
\[
\nexists x\forall B\forall A(x \in A \wedge x \in B \wedge A \cap B = \varnothing)
\]
\end{lem}
\begin{proof}
Suppose the contradiction: $x \in A \wedge x \in B \wedge A \cap B = \varnothing$, then 
$x \in A \cap B$, and thus $\in \varnothing$, which is impossible.
\end{proof}

The former: On one hand, suppose that $x \in A$. Then $x \notin B$ by Lemma \ref{lem10}. And
\[
x \in A \Longrightarrow x \in A \cup B \Longrightarrow x \in X
\].
So $x \in (X-B)$.

On the other hand, suppose that $x \in (X-B)$, then $x \in A \cup B$. But $x \notin B$, so $x \in A$ by 
Lemma \ref{lem10}.

The latter is immediately proven since $\cap,\cup$ are commutative.
\end{proof}

\paragraph{Exercise 3.1.10} \label{exercise3.1.10}
\begin{proof}
Firstly we prove that $(A-B)\cap (A\cap B) = \varnothing$.

$x \in (A\cap B)$ gives $x \in B$, but $x \in (A-B)$ gives $x \notin B$. So the two statements can not 
be true simultaneously. Which means 
\[
x\in (A-B)\cap (A\cap B) \Longrightarrow x \in \varnothing
\]

And obviously 
\[
x\in (A-B)\cap (A\cap B) \Longleftarrow x \in \varnothing
\].

Similarly we can conclude all of the three sets are disjoint by the fact that $\nexists x \in$ either 
two of the three sets.

Now we are showing that their union is $A \cup B$.

On one hand, suppose that 
\[
x \in (A-B)\cap(A\cap B)\cap(B-A)
\].
$x$ can at most be in one of these sets since they are disjoint.
If $x \in A$, then $x \in A \cup B$.

If $x \notin A$, then $x \in (B-A)$, and thus $x \in B$. So $x \in A \cup B$.

On the other hand, suppose that $x \in A \cup B$.
Then $x$ either
\begin{enumerate}
\item $\in A$, but $\notin B$, or
\item $\in B$, but $\notin A$, or
\item $\in$ both $A,B$.
\end{enumerate}

If (1), then $x \in (A-B)$.

If (2), then $x \in (B-A)$.

If (3), then $x \in A\cap B$.

In conclusion, we can see that $x \in (A-B)\cap(A\cap B)\cap(B-A)$.
\end{proof}

\paragraph{Exercise 3.1.11} \label{exercise3.1.11}
\begin{proof}
Let $P(x,y)$ be a property pertaining to any object $x,y$, and is true iff $Q(x)$.

According to Axiom 3.6, for any set $S$, there exists a set $Z$, such that 
$x \in Z \equiv x \in S \wedge P(x,y)$, which means $x \in Z \equiv x \in S \wedge Q(x)$. So 
is the axiom of specification proven.
\end{proof}