\section{General Principles}
This section describes the overall principles of the document. It illuminates how notations are explained, 
in what structure this document is written and so forth. This section should be read and understood 
comprehensively prior to reading the main content of the document.

\subsection{Definitions}
\begin{description}
\item[The document] The phrase \emph{the document} means this document (what you're reading) itself.
\item[The book] The phrase \emph{the book} represents Tao's \emph{Analysis} (both volume I and II).
\end{description}

\subsection{Indices}
The book has two volumes: 
\emph{Analysis I} and \emph{Analysis II}. We may notice that the indices of the two volumes both start 
from 1. It may lead to some confusions. So in the document, the indices are organized in such a way that:
If the content comes from \emph{Analysis I}, the corresponding index is the same as the book's. 
Otherwise, the corresponding index is prefixed with ``2.''.

For example, Exercise 3.1.3 in \emph{Analysis I} is indexed as Exercise 3.1.3 in the document, but 
Exercise 3.1.3 in \emph{Analysis II} is indexed as Exercise 2.3.1.3.

\subsection{Notations}
In the answers to some exercises, you may notice that the content are divided by numbers enclosed with 
parentheses (e.g. \textbf{(1)}, \textbf{(2)}). Tao often puts multiple questions into a single exercise, 
so these numbers indicates the number of the sub-questions.

For example, Exercise 3.5.4 is 
\begin{quotation}
Exercise 3.5.4. Let $A,B,C$ be sets. Show that $A\times(B\cup C) = (A\times B)\cup(A\times C)$,
that $A\times(B\cap C) = (A\times B)\cap(A\times C)$, and that 
$ A\times(B\setminus C) = (A\times B)\setminus(A\times C)$.
\end{quotation}
Then (1) indicates the question ``Show that $A\times(B\cup C) = (A\times B)\cup(A\times C)$.'', 
(2) indicates the question ``Show that $A\times(B\cap C) = (A\times B)\cap(A\times C)$.'', and 
(3) indicates the question ``Show that $A\times(B\setminus C) = (A\times B)\setminus(A\times C)$''.

In logical contents, 
\[
\Longrightarrow, \Rightarrow, \longrightarrow, \rightarrow, 
\]
have the same meaning ``implies''. And 
\[
\Longleftarrow, \Leftarrow, \longleftarrow, \leftarrow,
\]
also have the same meaning ($P \leftarrow Q$ means that $Q$ implies $P$).
Finally, these following symbols all indicate logical equality.
\[
\leftrightarrow, \longleftrightarrow, \Leftrightarrow, \Longleftrightarrow, \equiv
\].

For nested quantifiers, their order is ``from left to right''. For example, the following statement 
\[
\forall x \exists y (P(x,y))
\]
means that for all object $x$, their exists a object $y$ such that $P(x,y)$ is true.
That is,
\[
\forall x(\exists y(P(x,y)))
\]

Tao uses $++$ to denote the successor of a natural number. However, in the document, it is denoted by 
$S(n)$ most of the times.

\[
\bigvee_{i=1}^{n} P(i), \bigwedge_{i=1}^{n} P(i)
\]
mean that for $1\leq i \leq n$, at least one $P(i)$ is true; and for all $1\leq i \leq n$, $P(i)$ is 
true, respectively.

Some sets that have special meanings (e.g. the set of all natural numbers, the set of all real numbers) 
are denoted in whiteboard font (e.g. $\mathbb{N}, \mathbb{R}$).