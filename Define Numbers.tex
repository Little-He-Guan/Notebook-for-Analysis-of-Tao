% Copyright (C) He Guanyuming 2020
% The file is licensed under the MIT license.

\part{Natural Numbers}
\section{The Peano Axioms}
I have learned the Peano axioms. They are descriptive rather than constructive. That is, when we are 
using this axiom system, we assume that natural numbers do exist, and their properties are described by 
these axioms. The intuition of the Peano axioms might have been counting from 0 (or 1) by the successor 
function, but if we try to understand the Peano axioms as constructive, for example, giving 0 first and 
constructing other numbers via the successor function, things may look a little weird, and the axioms 
may seem incomplete.

There are some remarkable things regarding the axioms. For the first axiom, Peano originally used 1 
instead of 0. This is merely a difference of symbols here, though 0 and 1 have unique meanings in other 
areas, so 0 is more widely used today than 1. Peano also gave four axioms about the equality relation, 
and the first three of them (i.e.~except the one that says the equality relation is closed under 
natural numbers) are used as more generic assumptions for general mathematical objects.

Once we have described the basic properties of natural numbers and the successor function $S$, we can 
apply our common symbol system to it. We define $1 := S(0),\ 2 :=S(1)$ and so on.

\subsection{Addition of Natural Numbers}
Then we define operations, such as addition, on natural numbers. 

\paragraph{Addition}
The intuition is, the successor function acts like a $+1$ function. That is, 
\begin{equation}
n+1 := S(n) \label{eq.add1}
\end{equation}
And to add 2 to a number is to merely apply $S$ two times to it. So from the informal equation 
\ref{eq.add1}, we can furthermore define 
\begin{equation}
n+2 := S(S(n)) = S(n+1) \label{eq.add2}
\end{equation}

Note that by definition, $2=S(1)$. Apply the substitution to equation \ref{eq.add2}, we can see that 
\[
n + \textcolor{red}{S(1)} := S(n+\textcolor{red}{1})
\] We may notice that if we define $n+0 := n$, then equation \ref{eq.add1} can be rewritten as 
\[
n + \textcolor{red}{S(0)} := S(n+\textcolor{red}{0})
\]

So now we could try to assume two rules here:
\begin{definition}
\begin{enumerate}
\item $0+n:=n$,
\item $S(m)+n:=S(m+n)$
\end{enumerate}
\end{definition} and see if it is a good definition of addition.

For every natural number $n$, we first have $0+n=n$. Then if we want to know what $1+n$ is, we have
\begin{align*}
1+n 
&= S(0)+n\ \tag{By Def.~of 1} \\
&= S(0+n)\ \tag{By the second rule} \\
&= S(n)\ \tag{By the first rule}
\end{align*}

Repeat the process to gain more results:
\begin{align*}
2+n 
&= S(1)+n\ \tag{By Def.~of 2} \\
&= S(1+n)\ \tag{By the second rule} \\
&= S(S(n))\ \tag{By the result of \(1+n\)}
\end{align*}

Use induction (Suppose we have known 
$m+n=\underbrace{S(S(\dots(}_{m \text{ times}}n\underbrace{)\dots))}_{m \text{ times}}$ ):
\begin{align*}
S(m)+n
&= S(m+n)\ \tag{By the second rule} \\
&=\underbrace{S(S(\dots(}_{m+1 \text{ times}}n\underbrace{)\dots))}_{m+1 \text{ times}}\ 
\tag{By the result of \(m+n\)}
\end{align*}
And then the add operation is defined for every natural number.

Afterward we will turn to some properties of the newly defined operation -- addition. We are going to 
prove the commutativity and associativity of addition.

\begin{lem}
For any natural number $n$, $n+0=n$
\end{lem}
\begin{proof}
Firstly, by definition, $0 + 0 = 0$.

Secondly, if for natural number $n$, $n+0=n$ is true, then $S(n)+0=S(n+0)=S(n)$. This closes the 
induction, so the proposition is right. \qedhere
\end{proof}

\begin{lem}
For any natural number $m,n$, $n+S(m)=S(n+m)$ \label{lem2}
\end{lem}
\begin{proof}
For any fixed natural number $m$:
\begin{enumerate}
\item $0+S(m)=S(m)+0=S(m+0)$
\item Suppose that $n+S(m)=S(n+m)$, then 
\begin{align*}
S(n) + S(m) 
&= S(n+S(m)) \tag{By Def.} \\
&= S(S(n+m)) \tag{By assumption} \\
&= S(S(n)+m) \tag{By Def.}
\end{align*}
\end{enumerate} 
This closes the induction and the proof is over. \qedhere
\end{proof}

\begin{prop}
The addition of natural numbers is commutative. That is,
\[
m + n = n + m
\]
\end{prop}
\begin{proof}
First of all, $0+n=n+0$.

Then, assume that $m+n=n+m$. Thus:
\begin{align*}
S(m)+n
&= S(m+n) \tag{By Def.}\\
&= S(n+m) \tag{By Assumption}\\
&= n+S(m) \tag{By Lemma \ref{lem2}}
\end{align*}, which closes the induction. \qedhere
\end{proof}

\begin{prop}
(Exercise 2.2.1) \label{exercise2.2.1}
The addition of natural numbers is associative. That is, $(a+b)+c=a+(b+c)$.
\end{prop}
\begin{proof}
Use induction: First, $(0+b)+c=b+c=0+(b+c)=b+c$.

Then, assume that $(n+b)+c=n+(b+c)$, thus 
\begin{align*}
(S(n)+b)+(c)
&= S(n+b)+c\\
&= S(n+b+c)\\
&= S(n+(b+c)) \tag{By assumption}\\
&= S(n)+(b+c)
\end{align*}
, which closes the induction. \qedhere
\end{proof}

How fascinating! We have proven the basic properties of addition with only the definition of addition 
and the axioms. It seemed that we have to define these properties, but we did prove them!

Now we are about to prove some useful propositions about addition.

\begin{prop}
The cancellation law: If $a+b=a+c$, then $b=c$.
\end{prop}
\begin{proof}
Use induction: $0+b=0+c \Rightarrow b=c$.

Assume that $a+b=a+c \Rightarrow b=c$, thus 
\begin{align*}
S(a)+b &= S(a)+c \Rightarrow \\
S(a+b) &= S(a+c) \Rightarrow \\
S(b) &= S(c) \Rightarrow \\
b &= c
\end{align*}
\end{proof}

Then we describe natural numbers that are not equal to 0 as \textbf{positive}. 

\begin{prop}
If $a$ is positive, then for any natural number $b$, $a+b$ is positive. \label{prop1}
\end{prop}
\begin{proof}
Use induction: $a+0=a$ is positive.

Assume that $a+b$ is positive, then 
\[
a+S(b) = S(a+b)
\]
can not be 0, for 0 is not a successor of any natural number. This closes the induction. \qedhere
\end{proof}

\begin{coro}
If for natural number $a,b$, $a+b=0$, then $a=0 \wedge b=0$ \label{coro1}
\end{coro}
\begin{proof}
Presume the contradiction, that there exist $a \neq 0, b \neq 0$, $a+b=0$. 
\[
a \neq 0 \Rightarrow a \text{ is positive}
\]
Then according to proposition \ref{prop1}, $a+b$ is also positive, which can not be 0. \qedhere
\end{proof}

We may wonder something like is it true for every natural number $n \neq 0$, $n$ is always the 
successor of some other natural number. That is, 0 is the only natural number that is not the successor 
of any natural number. Or we can convey it in such a way as following: 
\begin{prop}
(Exercise 2.2.2) \label{exercise2.2.2}
For any positive natural number $n$, there is exactly one natural number $m$ that $S(m) = n$. 
\label{prop5}
\end{prop} 

\begin{proof}
Use induction: When $n=0$, the statement is vacuously right.

Assume that the statement is true for a natural number $n$, thus
\begin{description}
\item[Existence] 
\[
S(S(m)) = S(n)
\]
\item[Uniqueness]
It is obvious according to axiom 3.
\end{description}
\end{proof}

\begin{prop}
For any natural number $n$, $S(n) \neq n$
\end{prop}
\begin{proof}
Use induction: $S(0) \neq 0$ for 0 is not the successor of any natural number.

Assume that $S(n) \neq n$. Suppose that $S(S(n)) = S(n)$, then by axiom 3, $n=S(n)$, then we have a 
contradiction. This closes the induction. \qedhere
\end{proof}

\subsection{Order of Natural Numbers}
Then I learned the order of natural numbers.

Here I introduce one my own lemma:
\begin{lem}
$a=a+n \Leftrightarrow n=0$  \label{lem3}
\end{lem}
\begin{proof}
On one hand, suppose that $a=a+n$ but $n \neq 0$. Try to prove the contradiction. Use induction:
First, $n$ is positive, so $0+n \neq 0$.

Assume that $n \neq 0 \Rightarrow a \neq a+n$, thus 
\[
S(a)+n=S(a+n) \neq S(a) \text{ by assumption}
\], which closes the induction. Then by the axiom of induction, we have a contradiction, so 
$a=a+n \Rightarrow n=0$.

On the other hand, $n=0 \Rightarrow a+n=a$. \qedhere
\end{proof}

Hereby proposition 2.2.12 of Tao's book is proven.
\begin{prop}
(Exercise 2.2.3) \label{exercise2.2.3}
\begin{enumerate}
\item Order is reflective $a \geq a$
\item Order is transitive $a \geq b \wedge b \geq c \Rightarrow a \geq c$
\item Order is anti-symmetric $a \geq b \wedge b \geq a \Rightarrow a=b$
\item Addition preserves order $a \geq b \Rightarrow a+c \geq b+c$
\item $a<b \Leftrightarrow S(a) \leq b$
\item $a<b$ Iff for positive natural number $c$, $b=a+c$
\end{enumerate}
\end{prop}
\begin{proof}
(1) It is immediately proven by $a=a+0$.

(2) 
\begin{align*}
a \geq b &\wedge b \geq c \Rightarrow \\
a = b+m &\wedge b = c+n \Rightarrow \\
a = c+n+m &= c+(n+m) \Rightarrow \\
a &\geq c 
\end{align*}

(3)
By definition of order, 
\begin{align*}
a \geq b &\wedge b \geq a \Rightarrow \\
a = b+m &\wedge b = a+n \Rightarrow \\
a &= a+(n+m) \Rightarrow \\
n+m&=0 \Rightarrow \tag{By Lemma \ref{lem3}}\\
n=m&=0 \Rightarrow \tag{By Corollary \ref{coro1}} \\
b=a+0&=a
\end{align*}

(4)
\begin{align*}
a \geq b &\Rightarrow \\
a = b+n &\Rightarrow \\
a+c=(b+n)+c=b+(n+c)=b+(c+n)=(b+c)+n &\Rightarrow \\
a+c \geq b+c
\end{align*}

(5)
\begin{align*}
a<b &\Leftrightarrow \\
b=a+p &\Leftrightarrow \\
b+1=a+p+1=a+1+p=S(a)+p \\
=S(a)+S(n) \tag{By proposition \ref{prop5}, $p$ is always some natural number $n$'s successor}\\
=S(a)+n+1 &\Leftrightarrow \\
b=S(a)+n &\Leftrightarrow \tag{By cancellation law} \\
b \geq S(a)
\end{align*} \label{prop.5}

(6)
On one hand, $b=a+c$ immediately gives $a<b$.

On the other hand, according to (5), $a<b$ gives 
\begin{align*}
S(a) \leq b  &\Rightarrow\\
b = S(a) + n& \\
= a+1+n = a+(n+1)&
\end{align*}, where $n+1$ is positive. \label{prop.6}
\end{proof}

\begin{prop}
(Exercise 2.2.4) \label{exercise2.2.4}
For two natural number $m,n$, $m$ either $>$, or $=$, or $<n$.
\end{prop}
\begin{proof}
Tao's book has proven that at most one statement can be true at a time.

Now we are proving the remnant. Use induction: When $m=0$, for any natural number $n$, 
$0=n$, or $0 \neq n$. Under the latter case:
\begin{lem}
$n$ is positive $\Leftrightarrow n>0$
\end{lem}
\begin{proof}
On one hand, $n>0$ immediately gives $n$ is positive.

On the other hand, $n=0+n$ gives $n \geq 0$. And $n$ being positive implies that $n \neq 0$. 
So $n > 0$.
\end{proof}

According to the lemma, in this situation, $0<n$. So 0 either $<$ or $=$ $n$.

Assume that we have proven the statement for a natural number $m$, thus when $m<n$, according to 
Proposition \ref{prop.5}, $S(m) \leq n$, so $S(m)$ either $<$ or $=n$. When $m=n$, 
$S(m)=n+1 \Rightarrow S(m) > n$ by Proposition \ref{prop.6}. When $m>n$, according to Proposition 
\ref{prop.6}, 
\[
m=n+p \Rightarrow S(m)=n+(p+1) \Rightarrow S(m)>n
\]. This closes the induction, implying that at least one of the three statements is true.
\end{proof}

\paragraph{Exercise 2.2.5} \label{exercise2.2.5}
\begin{proof}
Let $Q(n)$ be a property of a natural number $n$ such that $Q(n)$ is true iff for all $m_0\leq m'<n$, 
$P(m')$ is always true. Use induction: $Q(0)$ is vacuously true.

Assume that $Q(n)$ is true. Here we will be using the proposition we just proved, for because we have 
known that there will and only will be one true statement, we can classify the conditions as 
following: When $S(n)<m_0$, $Q(S(n))$ is also vacuously true. When $S(n)=m_0$, $Q(S(n))$ is true 
because $P(m_0)$ is true. And when $S(n)>m_0$:

First we need to prove that $n$ is the only natural number $\geq m_0$ which satisfies $m_0 \leq n<S(n)$ 
but doesn't satisfy $m_0 \leq n<n$, so that we only need to prove $P(n)$ is true in the induction, 
which is obvious.
\begin{lem}
There is no natural number between $n$ and $S(n)$. That is, there is no such natural number $m$ that 
$n<m<S(n)$. \label{lem5}
\end{lem}
\begin{proof}
Presume the contradiction. Thus, $m=n+p \wedge S(n)=m+q$, where $p,q$ are positive. Substituting $m$ 
with $n+p$ we have $S(n)=n+p+q$. Let $p=S(a)=a+1$, which is always possible according to Proposition 
\ref{prop5}. Thus $n+1=n+1+a+q \Rightarrow n=n+a+q$, which means $a+q$ has to be 0, and which is 
impossible.
\end{proof}

Given a natural number $a$, it either $\geq$ or $<S(n)$, and also either $\geq$ or $<n$. Should it 
satisfy $m_0 \leq a<S(n)$ but doesn't satisfy $m_0 \leq a<n$, it then must satisfy $n \leq a<S(n)$, 
that is, either $a=n$ or $n<a<S(n)$. The latter, according to the lemma, is impossible. So $n$ is the 
only natural number $\geq m_0$ which satisfies $m_0 \leq n<S(n)$ but doesn't satisfy $m_0 \leq n<n$.

Then $Q(S(n)) \Leftrightarrow Q(n) \wedge P(n)$, which is true. This closes the induction. So $Q(n)$ is 
true for all natural number $n \geq m_0$. And this implies that $P(n)$ is true.
\end{proof}

\paragraph{Exercise 2.2.6} \label{exercise2.2.6}
\begin{proof}
Use induction: When $n=0$, for all natural number $m\leq 0$, $P(m)$ is true.

Assume that we have proven for a natural number $n$ that if $P(n)$ is true, then for all natural number 
$m\leq n$, $P(m)$ is also true. Thus, $P(S(n)) \Rightarrow P(n) \Rightarrow \forall m\leq n, P(m)$
is true. According to Lemma \ref{lem5}, 
$(\forall m\leq n, P(m)) \wedge P(S(n)) \Leftrightarrow \forall m \leq S(n), P(m)$. This closes the 
induction.
\end{proof}

\subsection{Multiplication of Natural Numbers}
\begin{lem}
(Exercise 2.3.1) \label{exercise2.3.1}
Multiplication is commutative. That is, $a \times b = b \times a$.
\end{lem}
\begin{proof} I

Try to imitate the way we prove the commutativity of addition.

\begin{lem}
\[
0 \times a = a \times 0 \label{lem2.3.1}
\]
\end{lem}
\begin{proof}
Use induction: $0 \times 0 = 0$. Assume that $n \times 0 = 0$ is 
true. Thus, $S(n) \times 0 = (n \times 0) + 0 = 0$, which closes the induction.
\end{proof}

\begin{lem}
\[
a \times S(b) = a \times b + a \label{lem2.3.2}
\]
\end{lem}
\begin{proof}
Use induction: $0 \times S(b) = 0 = 0 \times b + 0$.

Assume that $a \times S(b) = ab + a$ is true. Thus, 
\begin{align*}
S(a)S(b) 
&= aS(b) + S(b) \tag{By Def.}\\
&= ab + a + S(b) \tag{By assumption}\\
&= ab+S(a)+b \tag{By addition's properties}\\
&= (ab+b)+S(a) \tag{By addition's properties}\\
&= S(a)b + S(a) \tag{By Def.}
\end{align*}, which closes the induction.
\end{proof}

Now use induction on $a$. First, when $a=0$, by Lemma \ref{lem2.3.1} we have $ab=ba$. 

Assume that $ab=ba$ is true. Thus,
\begin{align*}
S(a)b
&=ab+b \\
&=ba+b  \\
&=bS(a) \tag{Lemma \ref{lem2.3.2}}
\end{align*}, which close the induction.

\end{proof}

\begin{proof} II In this proof we will use the distribution law of multiplication.

First, we have Lemma \ref{lem2.3.1}

Before we prove the remnant, we need to prove the distribution law. That 
is, $a \times (b+c) = ab + ac$ \label{distriLaw}
\begin{proof}
Use induction: $0 \times (b+c)= 0\times b+0\times c=0$.

Assume that $a \times (b+c) = ab + ac$ is ture. Thus, 
\begin{align*}
S(a) \times (b+c)
&= (a(b+c))+(b+c) \\
&= (ab+ac)+(b+c) \tag{By assumption} \\
&= (ab)+b+(ac)+c \\
&= S(a)b+S(a)c
\end{align*}, which closes the induction.
\end{proof}

We still have to prove $n \times 1 = n$ before proceeding. Use induction: $0 \times 1 = 0$. Assume that 
$n \times 1 = n$. Thus, $S(n) \times 1 = (n\times 1)+1=n+1=S(n)$.

Now we can proceed the proof. Assume that $a \times b = b \times a$. Thus, 
\begin{align*}
S(a)b
&= (ab)+b \\
&= (ba)+b \tag{By assumption} \\
&= b(a+1) \tag{By $b\times 1=b$ and the distribution law}\\
&= b \times S(a)
\end{align*}. This closes the induction.
\end{proof}

\begin{lem} \label{lem2.3.3}
(Exercise 2.3.2) \label{exercise2.3.2}
\[mn \neq 0 \Leftrightarrow m \neq 0 \wedge n \neq 0\]
\end{lem}

\begin{proof}
On one hand,
let $m=S(a),\ n=S(b)$, where $a,b$ are natural numbers.
\begin{align*}
mn 
&= S(a)S(b) \\
&= aS(b) + S(b)
\end{align*}

which, if $a \neq 0$, is the sum of two positive numbers, and is thus positive, and which, if $a = 0$,
is a positive number $S(b)$.

On the other hand, if either of $m,n$ is 0, then $mn$ must be zero. So $mn \neq 0 \Rightarrow m \neq 0 
\wedge n \neq 0$.
\end{proof}

Distribution law has been proved \hyperref[distriLaw]{here}.

\begin{prop}
(Exercise 2.3.3) \label{exercise2.3.3}
\[
(ab)c = a(bc)
\]
\end{prop}
\begin{proof}
Use induction on $a$.
First, $(0b)c=0c=0=0(bc)$.

Assume that $(ab)c = a(bc)$ is true. Thus, 
\begin{align*}
(S(a)b)c
&= (ab+b)c \\
&= c(ab+b) \tag{Commutativity} \\
&= c(ab) + cb \tag{Distribution law} \\
&= (ab)c + bc \tag{Commutativity} \\
&= a(bc) + bc \tag{The induction hypothesis} \\
&= S(a)(bc)
\end{align*}. And now we can close the induction.
\end{proof}

\begin{prop}
\label{prop2.3.1}
Multiplication preserves order. That is, if $a>b \wedge c>0$, then $ac > bc$.
\end{prop}
\begin{proof}
\begin{align*}
a>b 
&\Rightarrow \\ a = b + p 
&\Rightarrow \\ ac = bc + pc
\end{align*}
According to Lemma \ref{lem2.3.3}, $pc$ is positive. Therefore, $ac>bc$.
\end{proof}

\begin{coro}
Cancellation law. 
\[
ac=bc \wedge c \neq 0 \Rightarrow a=b
\]
\end{coro}
\begin{proof}
Either $a=b$, or $a<b$, or $a>b$. Suppose that $a \neq b$. Therefore, $ac$ either $<$ or $>bc$, which, 
according to Proposition \ref{prop2.3.1}, gives a contradiction. So $a=b$.
\end{proof}

\begin{prop}
(Exercise 2.3.4) \label{exercise2.3.4}
\[
(a+b)^2=a^2+2ab+b^2
\]
(Suppose that we have known $n^2=n \times n$)
\end{prop}
\begin{proof}
\begin{align*}
(a+b)(a+b) 
&= (a+b)a+(a+b)b \tag{Distribution law} \\
&= a(a+b) + b(a+b) \tag{Commutativity} \\
&= a^2+ab+ba+b^2 \tag{Distribution law} \\
&= a^2+ab+ab+b^2 \tag{Commutativity}
\end{align*}

Now we prove that $2ab = ab+ab$.
\begin{align*}
2ab
&=S(1)ab \\
&=(1ab)+ab\\
&=(S(0)ab)+ab\\
&=(0ab+ab)+ab\\
&=ab+ab
\end{align*}

The proof is over. \qedhere
\end{proof}

\begin{prop}
(Exercise 2.3.5) \label{exercise2.3.5}
Euclidean algorithm. For any natural number $n$, positive number $p$, there exist natural numbers $m,r$ 
such that $n=mp+r$.
\end{prop}
\begin{proof}
For any natural number $p$, we induct on $n$. Firstly, $0=0p+0$. 

Assume that the statement for $n$ is true. We know that $r<p$. Then $S(r)$ either $=$ or $<p$ 
(Proposition \ref{prop.5}). On the latter case, simply let $r'=S(r),\ m'=m$, which satisfies the 
restriction $0\leq r' <p$ 

On the former case, let $m'=S(m),\ r'=0$, and we have
\begin{align*}
m'p+r'
&= S(m)p+0\\
&= mp + p\\
&= mp + S(r) \tag{$p=S(r)$}\\
&= S(mp+r)\\
&= S(m) \qedhere
\end{align*}
\end{proof}