% Copyright (C) He Guanyuming 2020
% The file is licensed under the MIT license.

\part{Integers and Rationals}
Now we are going to extend natural numbers to integers and rationals.

\section{The Integers}

\paragraph{Exercise 4.1.1} \label{exercise4.1.1}
\begin{proof}
It is immediately given by the fact that 
\[
a+b = a+b \equiv a -- b = a -- b
\]
\end{proof}

\paragraph{Lemma 4.1.3}
\[
(m--0)+(n--0) = (m+n)--0
\]
\[
(m--0) \times (n--0) = (mn) -- 0
\]
ensures that the definition $m--0:=m$ is consistent with addition and multiplication.

\paragraph{Exercise 4.1.2} \label{exercise4.1.2}
\begin{proof}
\[
a--b = a'--b' \equiv a=b \wedge a'=b'
\]
Then, 
\[
(b--a) = (b'--a') \equiv -(a--b) = -(a'--b')
\]
\end{proof}

\paragraph{Exercise 4.1.3} \label{exercise4.1.3}
\begin{proof}
\begin{align*}
-1 \times a 
&= (0 -- 1) \times (a -- 0) \\
&= (0\times a + 1 \times 0) -- (0 \times 0 + 1 \times a) \\
&= 0 -- a \\
&= -a
\end{align*}
\end{proof}

\paragraph{Exercise 4.1.4} \label{exercise4.1.4}
\begin{proof}
Let $x=(a--b),y=(c--d),z=(e--f)$.

(1)
\begin{align*}
(a--b) + (c--d) 
&= (a+c) -- (b+d) \\
&= (c+a) -- (d+b) \\
&= (c--d) + (a--b)
\end{align*}

(2)
\begin{align*}
((a--b) + (c--d)) + (e--f)
&= ((a+c)+e) -- ((b+d)+f) \\
&= (a+(c+e)) -- (b+(d+f)) \\
&= (a--b) + ((c--d) + (e--f))
\end{align*}

(3)
First ,
\[
(a--b) + (0--0) = (a--b)
\].

Second, by (1) we have $0+x=x+0$.

(4)
First, 
\begin{align*}
(a--b) + (b--a) 
&= (a+b) -- (a+b) \\
&= 0 -- 0 \tag{$a+b+0=a+b+0$}
\end{align*}

Second, by (1) we have $x+(-x) = (-x) + x$.

(5)
\begin{align*}
(a--b)(c--d)
&= (ac + bd) -- (ad + bc) \\
&= (ca + db) -- (cb + da) \\
&= (c--d)(a--b)
\end{align*}

(6)
The book proved this.

(7)
First,
\[
(1--0)(a--b) = (1a + 0b) -- (1b+0a) = (a--b)
\]

Second, by (5) we have $1x=x1$.

(8)
\begin{align*}
&(a--b)((c--d)+(e--f)) \\
&= (a--b)((c+e)--(d+f)) \\
&= (a(c+e) + b(d+f)) -- (a(d+f) + b(c+e)) \\
&= ((ac + bd)+(ae + bf)) -- ((ad + bc)+(af + be)) \\
&= (ac+bd)--(ad+bc) + (ae+bf)--(af+be) \\
&= (a--b)(c--d) + (a--b)(e--f)
\end{align*}

(9)
This can be easily concluded from (5) and (8).
\end{proof}

\paragraph{Exercise 4.1.5} \label{exercise4.1.5}
\begin{proof}
We need to show that 
\[
a \neq 0 \wedge b \neq 0 \Longrightarrow ab \neq 0
\]

Since $a,b$ are not 0, they can be either positive or negative. If they are both positive, the case is 
already proven.

When at least one of them is negative, we can divide the $-1$ from the negative ones. That is, if $a=-m$, 
where $m$ is positive, then we substitute $a$ with $-1 \times m$. Then we may get $ab$ in either the form 
$(-1)(-1) mn$ or $(-1) mn$, where the former is a positive number because $(-1)(-1) =1$ and the latter is 
negative.
\end{proof}

\paragraph{Exercise 4.1.6} \label{exercise4.1.6}
\begin{proof}
We check the value of $ac-bc$. We know that $ac=bc$, so $ac - bc = 0 - 0 = 0$. According to (9) in 
Proposition 4.1.6, 
\[
ac - bc = ac+(-b)c = (a+(-b))c = 0
\]

As stated by Proposition 4.1.8, since that $c \neq 0$, $a+(-b) = 0$, which means $a-b=0$. Then we have 
$a=b$.
\end{proof}

\paragraph{Exercise 4.1.7} \label{exercise4.1.7}
In the following contents, $p$ stands for a positive natural number, $n$ stands for a natural number.

\begin{proof}
(a)
\begin{align*}
a>b 
&\equiv a = b+p \\
&\equiv a+(-b) = b + (-b) + p  \tag{See the following explanation} \\
&\equiv a-b = p
\end{align*}
We now explain why $a = b+p \equiv a+(-b) = b + (-b) + p$. Using the substitution law and the 
commutativity of addition, it is clear to see that $a = b+p \Longrightarrow a+(-b) = b + (-b) + p$. We now 
show the cancellation law of addition, that is,
\begin{lem}
\[
a+c = b+c \Longrightarrow a = b
\]
\end{lem}
\begin{proof}
\begin{align*}
a+c=b+c
&\Longrightarrow a+c+(-c) = b+c+(-c) \\
&\Longrightarrow a+(c+(-c)) = b + (c+(-c)) \\
&\Longrightarrow a=b
\end{align*}
\end{proof}

So we get the inverse result: $a = b+p \Longleftarrow a+(-b) = b + (-b) + p$.

Note that by the definition of integer and what we have know now, we can conclude that 
\begin{lem}
For every integer 
$i = a - b, j = c - d$, there exists exactly one integer $k$ such that $i = j+k$.
\end{lem}

(b)
\begin{align*}
a>b
&\equiv a = b + p \\
&\Longrightarrow a+c = b+c+p \\
&\Longrightarrow a+c>b+c
\end{align*}

(c)
\begin{align*}
a>b
&\equiv a=b+p \\
&\Longrightarrow ac = (b+p)c = bc + pc \\
&\Longrightarrow ac > bc \tag{$pc > 0$ by Lemma 2.3.3}
\end{align*}

(d)
\[
a>b \equiv a = b+p
\]
Then
\[
-a = -(b+p) = (-1)(b+p) = -b - p
\]
So
\[
-a+p=-b-p+p
\]
That is,
\[
-b=-a+p \equiv -b>-a
\]

(e)
Let
\[
a = b+p_1,b=c+p_2
\]
Then $a = c+(p_1+p_2)$. Obviously $p_1+p_2$ is positive, so $a>c$.

Note that $-a,-b$ are also integers, and plus that $-(-a)=a$, so we can give a stronger conclusion:
\[
a>b \equiv -a<-b
\]

(f)
If $a,b$ are all natural numbers, the statement was proven before. 

If one of them (say $a$) is negative, 
the other ($b$) is a natural number, then $a=-n$, and we know that $b>0$ and 
$0 = a+n \Longrightarrow 0 >-a$, so by (e) we have $b>a$.

If they are both negative, then their negations satisfy the statement. Then
\[
-a<-b \equiv a>b, -a=-b \equiv a=b, -a>-b \equiv a<b
\].
\end{proof}

\paragraph{Exercise 4.1.8} \label{exercise4.1.8}
An example: $P(i): i>=0$. It is obvious that $P(0)$ and $P(i) \Longrightarrow P(i+1)$ is true. But for any  
negative integer $n$, $P(n)$ is not true.

We additionally prove one more property:
\begin{lem}
For integers $a,b$,
\[
a-b=0\Longrightarrow a=b
\]
\end{lem}
\begin{proof}
We can add $b$ to both side to obtain $a=b$.
\end{proof}

\section{The Rationals}
\paragraph{Exercise 4.2.1} \label{exercise4.2.1}
\begin{proof}
Reflectivity:
\[
a//b = a//b \equiv ab=ab
\] 

Being Symmetric:
\begin{align*}
a//b = c//d 
&\equiv ad = bc \\
&\equiv cb = da \\
&\equiv c//d = a//d
\end{align*}

Transitivity:
\[
a//b = c//d \equiv ad = bc
\]
\[
c//d = e//f \equiv cf = de
\]
Thus,
\[
(ad)(cf) = (bc)(de)
\]
We then have
\[
afcd = becd
\]
We can cancel $d$ since $d \neq 0$ to obtain $afc=bec$. If $a=0$, we can conclude that $c,e$ also must be 
$0$. Under this occasion, $af=be$ is also true because they all equal to $0$. 
\end{proof}

\paragraph{Definition 4.2.2}
It is useful to prove that 
\begin{lem} \label{lem4.2.3}
\[
(-a)//b = a//(-b)
\]
,
\[
a//b = (-a)//(-b)
\]
\end{lem}
\begin{proof}
The first is immediately given since $(-a)(-b) = ab$. The latter is proven as $a(-b) = b(-a) = -ab$.
\end{proof}

We may notice that subtraction is not mentioned here. This is because that we can get $a-b$ by adding $a$ 
and $-b$, where addition $+$ and negation $-$ are mentioned.

\paragraph{Exercise 4.2.2} \label{exercise4.2.2}
\begin{proof}
(1) is deduced in the book.

(2)
I don't quiet understand why Tao used this $*$ sign instead of $\times$. I know it is a new definition, 
but the $\times$ sign is undefined for rationals (except for integers, but for which we can verify that 
the two definitions are the same). We will use the $\times $ sign or just leave it off here.

\[
(a'//b')(c//d) \equiv a'd = b'c \equiv ad=bc \equiv (a//b)(c//d)
\]
Similarly we can verify this for $c'//d'$.

(3)
\[
-ab'=-a'b \equiv (-a)//b = (-a')//b'
\]
\end{proof}

For the sake of simplification, we hereby introduce some useful lemmas:
\begin{lem} \label{lem4.2.1}
\[
b=d\neq 0 \Longrightarrow (a//b = c//d \equiv a=c)
\]
\end{lem}
\begin{proof}
Assume that $b=d\neq 0$.

On one hand, if $a//b=c//d$, then $ad=bc$. Since that $b=d \neq 0$, we can cancel them to obtain $a=c$.

On the other hand, if $a=c$, then if we multiply them by the same integer (namely $b=d$), and the 
results are still equal ($ad=bc$). So $a//b=c//d$.
\end{proof}
\begin{lem} \label{lem4.2.2}
\[
c \neq 0 \Longrightarrow a//b=ac//bc
\]
\end{lem}
\begin{proof}
Assume that $c\neq 0$.

First we know that $ab=ab$. Then we can further obtain $abc=abc$, which 
means $a//b=ac//bc$.
\end{proof}

\paragraph{Exercise 4.2.3} \label{exercise4.2.3}
\begin{proof}
(1)
We have
\[
a//b + c//d = (ad+bc)//(bd)
\]
\[
c//d + a//b = (cb+da)//(db)
\]
It is easy to see that they are equal.

(2)
It is proven in the book.

(3)
We just deduce $x+0=x$ here, for we have $0+x=x+0$ according to (1).
\[
a//b + 0//1 = (a1+b0)//(b1) = a//b
\]

(4)
We only prove $x+(-x)=0$ here, for we have $x+(-x)=(-x)+x$ according to (1).
\[
a//b+(-a)//b= (ab-ab)//bb = 0//b^2=0
\]

(5)
\begin{align*}
a//b \times c//d
&= ac//bd \\
&= ca//db \\
&= c//d \times a//b
\end{align*}

(6)
\begin{align*}
&(a//b \times c//d) \times e//f \\
&= ac//bd \times e//f \\
&= ace//bdf \\
&= a//b \times ce//df \\
&= a//b \times (c//d \times e//f)
\end{align*}

(7)
We only prove $x1=x$ here, for we have $x1=1x$ according to (4).
\[
a//b \times 1//1 = a1//b1 = a//b
\]

(8)
\begin{align*}
&a//b (c//d + e//f) \\
&= a//b ((cf+ed) // (df)) \\
&= a(cf+ed)//bdf \\
&= ab(cf+ed) // b^2df \tag{See Lemma \ref{lem4.2.2}} \\
&= ((ac)(bf) + (bd)(ae)) // (bd)(bf) \\
&= ac//bd + ae//bf \\
&= (a//b \times c//d) + (a//b + e//f)
\end{align*}

(9)
This can be deduced from (5) and (8).

(10)
We merely conclude $xx^{-1} = 1$ here, since we have $xx^{-1} = x^{-1}x$ from (5).

\[
a//b \times b//a  = ab//ba = (ab)1//(ab)1 = 1//1
\]
The last step is done by Lemma \ref{lem4.2.2}.
\end{proof}

\paragraph{Exercise 4.2.4} \label{exercise4.2.4}
\begin{proof}
For any rational $r = a/b$, $a,b$ are integers.
They are either positive, $0$, or negative (except that $b$ cannot be 0). When $a,b$ are both positive, 
then $r$ is also positive. When $a$ is positive but $b$ is negative, then let $b=-p$, where $p$ is 
positive, thus $a/b = a/(-p) = (-a)/p$ is negative. When $a=0$, $r=0$. When $a$ is negative, and $b$ is 
positive, then by definition $r$ is negative. When $a,b$ are both negative, according to Lemma 
\ref{lem4.2.3}, $r$ is positive.

Therefore, we have iterated through all possible situations and verified that there is and only is one 
statement for a rational is true.
\end{proof}

\paragraph{Exercise 4.2.5} \label{exercise4.2.5}
\begin{proof}
Let $x=a/b,y=c/d,z=e/f$. Before proving the following components, we will introduce some useful 
propositions here.
\begin{lem} \label{lem4.2.4}
\begin{enumerate}
\item $x>0$ is logically equivalent to $x$ being positive.
\item $x<0$ is logically equivalent to $x$ being negative.
\end{enumerate}
\end{lem}
\begin{proof}
\[
x-0 =x
\]
is itself, so whether $x$ is positive or negative, the same is $x-0$, then we can deduce $x>0$ or $x<0$.
\end{proof}

We can now use simplified notation $x>0$ to express the same meaning: $x$ is positive. 

(a)
We check the value of 
\[
\delta = x-y = a/b +(-c)/d = (ad-bc)/bd
\]
$\delta$ is also a rational number. According to the previous exercise, it is either positive, negative, 
or $0$. So $x$ either $>y$, $<y$, or $=y$ (We haven't yet proven $x-y=0 \Longrightarrow x = y$. Let's 
prove it now. We can add $y$ to both side of $x-y=0$ to obtain the result).

(b)
According to Lemma \ref{lem4.2.4}, $x<y \Longrightarrow x-y<0$. Then we multiply $-1/1$ with $x-y$ to 
obtain (It is easy to see that for rational number $r$, $-1r = -r$ and $-(-r)=r$)
\[
-1/1 (x + (-y)) = -x + -(-y) = y - x
\]
Since $x-y$ is negative, and the negation of a positive number is negative, so the negation of $x-y$, 
$y-x$, is positive, which means that $y>x$.

(c)
By the hypothesis, $x-y<0 \wedge y-z<0$. We are now proving that $i,j<0 \Longrightarrow i + j <0$. We can 
write $i,j$ as $o/p,q/s$ respectively. Let $p,s>0$, then $o,q<0$. Then $o/p+q/s = (os+pq)/ps$. We know 
that $os,pq<0$ (Write a negative integer as a negation of a positive integer to see that the product of a 
positive and a negative is also negative). 

Now we show that for two positive integers, their sum is still 
positive. Integers who are positive are also natural number, and their sum remains a natural number. So 
the sum itself equals to $0$ plus itself, which means it is positive. The negation of this sum, which is 
also $-m+(-n)$, is thus negative. Since that $-m,-n$ can present any negative integer, the fact means that 
the sum of two negative integers remains a negative integer.

So $os+pq<0$. But $ps>0$, so $i+j<0$. Thus, $(x-y) + (y-z) = x-z<0$, which means $x<z$.

(d)
\begin{align*}
x+z-(y+z) 
&= x+z + (-)(y+z) \\
&= x+z + (-1)(y+z) \\
&= x+z -z - y \\
&= x-y <0
\end{align*}

(e)
It is easy to verify that the product of two positive rationals is still positive (Writing them as 
$a/b,c/d$, where $a,b,c,d>0$, then $ac/bd$ also $>0$). Then $xz-yz=z(x-y)$, which is the product of a 
positive number and a negative number, and is thus a negative number.
\end{proof}

\paragraph{Exercise 4.2.6} \label{exercise4.2.6}
\begin{proof}
According to (e) of Proposition 4.2.9, we need only to show that $x<y \Longrightarrow -x>-y$. Then we can 
multiply $xz>yz$ with $-1$ to obtain what we want.

We know that the negation operation will turn a positive into negative and vice versa. Now we have 
$x-y<0$, so the negation $-(x-y) = -x+y = -x - (-y)>0$, which means that $-x>-y$.
\end{proof}

There are still many properties about rationals that we use for granted (e.g. $x^-1$ has the same sign as 
$x$; the two definitions of order are the same, that is, $x-y>0 \equiv x=y+p \equiv x>y$, where $p>0$). 
Although they need to be proven prior to being used, we can not cover all of them here. We will 
prove some of them in the future only if they are used. Also, most of them are not hard to prove. We need 
not to worry.