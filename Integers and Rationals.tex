\part{Integers and Rationals}
Now we are going to extend natural numbers to integers and rationals.

\section{Integers}

\paragraph{Exercise 4.1.1} \label{exercise4.1.1}
\begin{proof}
It is immediately given by the fact that 
\[
a+b = a+b \equiv a -- b = a -- b
\]
\end{proof}

\paragraph{Lemma 4.1.3}
\[
(m--0)+(n--0) = (m+n)--0
\]
\[
(m--0) \times (n--0) = (mn) -- 0
\]
ensures that the definition $m--0:=m$ is consistent with addition and multiplication.

\paragraph{Exercise 4.1.2} \label{exercise4.1.2}
\begin{proof}
\[
a--b = a'--b' \equiv a=b \wedge a'=b'
\]
Then, 
\[
(b--a) = (b'--a') \equiv -(a--b) = -(a'--b')
\]
\end{proof}

\paragraph{Exercise 4.1.3} \label{exercise4.1.3}
\begin{proof}
\begin{align*}
-1 \times a 
&= (0 -- 1) \times (a -- 0) \\
&= (0\times a + 1 \times 0) -- (0 \times 0 + 1 \times a) \\
&= 0 -- a \\
&= -a
\end{align*}
\end{proof}

\paragraph{Exercise 4.1.4} \label{exercise4.1.4}
\begin{proof}
Let $x=(a--b),y=(c--d),z=(e--f)$.

(1)
\begin{align*}
(a--b) + (c--d) 
&= (a+c) -- (b+d) \\
&= (c+a) -- (d+b) \\
&= (c--d) + (a--b)
\end{align*}

(2)
\begin{align*}
((a--b) + (c--d)) + (e--f)
&= ((a+c)+e) -- ((b+d)+f) \\
&= (a+(c+e)) -- (b+(d+f)) \\
&= (a--b) + ((c--d) + (e--f))
\end{align*}

(3)
First ,
\[
(a--b) + (0--0) = (a--b)
\].

Second, by (1) we have $0+x=x+0$.

(4)
First, 
\begin{align*}
(a--b) + (b--a) 
&= (a+b) -- (a+b) \\
&= 0 -- 0 \tag{$a+b+0=a+b+0$}
\end{align*}

Second, by (1) we have $x+(-x) = (-x) + x$.

(5)
\begin{align*}
(a--b)(c--d)
&= (ac + bd) -- (ad + bc) \\
&= (ca + db) -- (cb + da) \\
&= (c--d)(a--b)
\end{align*}

(6)
The book proved this.

(7)
First,
\[
(1--0)(a--b) = (1a + 0b) -- (1b+0a) = (a--b)
\]

Second, by (5) we have $1x=x1$.

(8)
\begin{align*}
&(a--b)((c--d)+(e--f)) \\
&= (a--b)((c+e)--(d+f)) \\
&= (a(c+e) + b(d+f)) -- (a(d+f) + b(c+e)) \\
&= ((ac + bd)+(ae + bf)) -- ((ad + bc)+(af + be)) \\
&= (ac+bd)--(ad+bc) + (ae+bf)--(af+be) \\
&= (a--b)(c--d) + (a--b)(e--f)
\end{align*}

(9)
This can be easily concluded from (5) and (8).
\end{proof}

\paragraph{Exercise 4.1.5} \label{exercise4.1.5}
\begin{proof}
We need to show that 
\[
a \neq 0 \wedge b \neq 0 \Longrightarrow ab \neq 0
\]

Since $a,b$ are not 0, they can be either positive or negative. If they are both positive, the case is 
already proven.

When at least one of them is negative, we can divide the $-1$ from the negative ones. That is, if $a=-m$, 
where $m$ is positive, then we substitute $a$ with $-1 \times m$. Then we may get $ab$ in either the form 
$(-1)(-1) mn$ or $(-1) mn$, where the former is a positive number because $(-1)(-1) =1$ and the latter is 
negative.
\end{proof}

\paragraph{Exercise 4.1.6} \label{exercise4.1.6}
\begin{proof}
We check the value of $ac-bc$. We know that $ac=bc$, so $ac - bc = 0 - 0 = 0$. According to (9) in 
Proposition 4.1.6, 
\[
ac - bc = ac+(-b)c = (a+(-b))c = 0
\]

As stated by Proposition 4.1.8, since that $c \neq 0$, $a+(-b) = 0$, which means $a-b=0$. Then we have 
$a=b$.
\end{proof}

\paragraph{Exercise 4.1.7} \label{exercise4.1.7}
In the following contents, $p$ stands for a positive natural number, $n$ stands for a natural number.

\begin{proof}
(a)
\begin{align*}
a>b 
&\equiv a = b+p \\
&\equiv a+(-b) = b + (-b) + p  \tag{See the following explanation} \\
&\equiv a-b = p
\end{align*}
We now explain why $a = b+p \equiv a+(-b) = b + (-b) + p$. Using the substitution law and the 
commutativity of addition, it is clear to see that $a = b+p \Longrightarrow a+(-b) = b + (-b) + p$. We now 
show the cancellation law of addition, that is,
\begin{lem}
\[
a+c = b+c \Longrightarrow a = b
\]
\end{lem}
\begin{proof}
\begin{align*}
a+c=b+c
&\Longrightarrow a+c+(-c) = b+c+(-c) \\
&\Longrightarrow a+(c+(-c)) = b + (c+(-c)) \\
&\Longrightarrow a=b
\end{align*}
\end{proof}

So we get the inverse result: $a = b+p \Longleftarrow a+(-b) = b + (-b) + p$.

Note that by the definition of integer and what we have know now, we can conclude that 
\begin{lem}
For every integer 
$i = a - b, j = c - d$, there exists exactly one integer $k$ such that $i = j+k$.
\end{lem}

(b)
\begin{align*}
a>b
&\equiv a = b + p \\
&\Longrightarrow a+c = b+c+p \\
&\Longrightarrow a+c>b+c
\end{align*}

(c)
\begin{align*}
a>b
&\equiv a=b+p \\
&\Longrightarrow ac = (b+p)c = bc + pc \\
&\Longrightarrow ac > bc \tag{$pc > 0$ by Lemma 2.3.3}
\end{align*}

(d)
\[
a>b \equiv a = b+p
\]
Then
\[
-a = -(b+p) = (-1)(b+p) = -b - p
\]
So
\[
-a+p=-b-p+p
\]
That is,
\[
-b=-a+p \equiv -b>-a
\]

(e)
Let
\[
a = b+p_1,b=c+p_2
\]
Then $a = c+(p_1+p_2)$. Obviously $p_1+p_2$ is positive, so $a>c$.

Note that $-a,-b$ are also integers, and plus that $-(-a)=a$, so we can give a stronger conclusion:
\[
a>b \equiv -a<-b
\]

(f)
If $a,b$ are all natural numbers, the statement was proven before. 

If one of them (say $a$) is negative, 
the other ($b$) is a natural number, then $a=-n$, and we know that $b>0$ and 
$0 = a+n \Longrightarrow 0 >-a$, so by (e) we have $b>a$.

If they are both negative, then their negations satisfy the statement. Then
\[
-a<-b \equiv a>b, -a=-b \equiv a=b, -a>-b \equiv a<b
\].
\end{proof}

\paragraph{Exercise 4.1.8} \label{exercise4.1.8}
An example: $P(i): i>=0$. It is obvious that $P(0)$ and $P(i) \Longrightarrow P(i+1)$ is true. But for any  
negative integer $n$, $P(n)$ is not true.