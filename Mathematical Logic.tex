\part{Mathematical Logic}

\section{Mathematical Statements}

\paragraph{Exercise A.1.1}
\label{exercisea.1.1}
It is ( (both $X, Y$ are false) or (both $X, Y$ are true) ).

\paragraph{Exercise A.1.2}
\label{exercisea.1.2}
It is ( ($Y$ can be true even if $X$ is false) or ($Y$ can be false even if $X$ is true) ).

\paragraph{Exercise A.1.3}
\label{exercisea.1.3}
Yes. That's the definition of logical equivalent.

\paragraph{Exercise A.1.4}
\label{exercisea.1.4}
No. It is still possible that (even if $X$ is false, $Y$ is still true).

Consider a statement $Y$ that satisfies:
\begin{enumerate}
\item If $X$, then $Y$.
\item If $X$ is false, then $Y$ or (exclusively) $Y$ is false.
\end{enumerate}

$X,Y$ satisfy the description in the exercise, but they are not logical equivalent.

\paragraph{Exercise A.1.5}
\label{exercisea.1.5}
Yes. (Now I'm using the symbols defined in the A.2 for the sake of simplification)
$X \Longleftrightarrow Y$ means $X \Longrightarrow Y \wedge \neg X \Longrightarrow \neg Y$. So does $Y$ and 
$Z$. So 
\begin{align*}
(X \Longrightarrow Y \Longrightarrow Z \wedge \neg X \Longrightarrow \neg Y \Longrightarrow \neg Z)
&\Longrightarrow \\
(X \Longrightarrow Z \wedge \neg X \Longrightarrow \neg Z)
\end{align*}
, which means $X$ and $Z$ are logical equivalent.

(Note that $A \Longrightarrow B$ can also be interpreted as a statement, meaning ``If $A$ is true, then $B$ 
is true'', just like we did in this example.)

\paragraph{Exercise A.1.6}
\label{exercisea.1.6}
Yes. $(X \Longrightarrow Y \Longrightarrow Z) \Longrightarrow (X \Longrightarrow Z)$. 

Now we are proving that 
$Z \Longrightarrow X \equiv \neg X \Longrightarrow \neg Z$. Assume that $\neg X \wedge Z$. Since 
$Z \Longrightarrow X$, we have a contradiction: $X \wedge \neg X$.

So $X \Longrightarrow Z \wedge \neg X \Longrightarrow \neg Z$. Therefore, $X,Z$ are logical equivalent. 
Besides, we can conclude that $Y \Longrightarrow X$. Thus $X,Y$ are also logical equivalent.

\section{Implication}
Why did Tao say
\begin{quotation}
If $X$, then $Y$ can also be written as ``$X$ can only be true when $Y$ is true''
\end{quotation}?

Assume the $X \wedge \neg Y$, thus $X \Longrightarrow Y$. So we have a contradiction 
$Y \wedge \neg Y$.

Define ``when $x \neq 2$, $X:x=2 \Longrightarrow x^2=4$ is vacuously true'' to ensure that $X$ is always 
true regardless of the value of $x$.

