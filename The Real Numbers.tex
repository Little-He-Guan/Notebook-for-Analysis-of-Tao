\part{The Real Numbers}

\section{Cauchy Sequences}
\paragraph{Exercise 5.1.1} \label{exercise5.1.1}
\begin{proof}
Since $(a_n)^\infty_{n=1}$ is a Cauchy sequence, it is 1-steady for some $N$. That implies,
\[
\forall n \geq N(|a_n-a_N| \leq 1)
\]
And we know that $a_1,a_2,\dots,a_N$ is bounded by some number $M$, which means $|a_N| \leq M$. Expand $|a_n-a_N| \leq 1$ to obtain 
\[
a_N-1 \leq a_n \leq a_N+1
\]
If $a_N \geq 0$, then $a_n \geq a_N-1 \geq -a_N-1$, then 
\[
-(a_N+1) \leq a_n \leq a_N+1 \equiv |a_n| \leq |a_N+1|
\]
So 
\[
|a_n| \leq |a_N+1| \leq |a_N| + |1| \leq M+1
\]
If $a_N < 0$, then $a_n \leq a_N +1 < -a_N+1$, then
\[
a_N - 1 \leq a_n \leq -(a_N-1) \equiv |a_n| \leq |a_N-1|
\]
We can also get 
\[
|a_n| \leq |a_N-1| \leq |a_N| + |1| \leq M+1
\]
Therefore, we know that $(a_n)^\infty_1$ is bounded by $M+1$.
\end{proof}

\section{Equivalent Cauchy Sequences}
\paragraph{Exercise 5.2.1} \label{exercise5.2.1}
\begin{proof}
Although we need to prove that $a_n$ being a Cauchy sequence is logically equivalent to $b_n$ being a Cauchy sequence, showing that one 
implies another is enough due to the structure of these statements.

Now we show that $a_n$ being a Cauchy sequence implies that $b_n$ being a Cauchy sequence. We need to show that for any $\varepsilon >0$, 
there exists a $N$ such that for all $m,n\geq N$, $|a_m-a_n| \leq \varepsilon$. 

First we know that for any $\varepsilon >0$, there exists a $N$ such that $\forall m,j\geq N(|a_m-b_j|\leq \varepsilon)$. By substituting $m$ 
with $n$ we have both $|a_m-b_j|\leq \varepsilon$ and $|a_n-b_j|\leq \varepsilon$.
Thus, 
\[
|a_m-a_n| = |(a_m-b_j) - (a_n-b_j)| \leq |a_m-b_j| + |a_n-b_j| \leq 2\varepsilon
\]

Then we find the $N'$ such that $\forall m,n\geq N(|a_m-b_n| \leq \varepsilon/2)$, and then for $i,j \geq N$, $|a_i-a_j| \leq \varepsilon$.
\end{proof}

\paragraph{Exercise 5.2.2} \label{exercise5.2.2}
\begin{proof}
We need only to show that $a_n$ being bounded implies that $b_n$ being so because of the structure of these statements. 

Consider \exerciseref{5.1.1}. Since that $(a_n)^\infty_{n=1},(b_n)^\infty_{n=1}$ are eventually $\varepsilon$-close, the sequence 
$(a_n-b_n)^\infty_{n=1}$ is eventually $\varepsilon$-steady. Thus, according to the exercise mentioned, it is bounded by some number $M$. And 
we say that $a_n$ is bounded by some number $N$.

So $|a_n-b_n| \leq M$. Again, similar to the proof of $|a_n| \leq |a_N|+1$ in \exerciseref{5.1.1}, we can obtain that 
$|b_n| \leq |a_n| + M \leq N+M$. Therefore, $b_n$ is also bounded.
\end{proof}

\section{The Construction of the Real Numbers}
\paragraph{Exercise 5.3.1} \label{exercise5.3.1}
\begin{proof}
Reflectivity: It is immediately derived from $|a_n-a_n| = 0$.

Symmetry: It is immediately derived from $|a_n-b_n| = |b_n-a_n|$.

Transitivity: For any $\varepsilon >0$, we can find $M,N$ such that $\forall n\geq M(|a_n-b_n| \leq \varepsilon)$ and 
$\forall n\geq N(|b_n-c_n| \leq \varepsilon)$. Let $B = \max{(M,N)}$. Then for $n\geq B$,
\[
|a_n-c_n| \leq |a_n-b_n| + |b_n-c_n| \leq 2\varepsilon
\]
This can also be deduced by (c) in Proposition 4.3.7
So $a_n$ and $c_n$ are also equal.
\end{proof}

\paragraph{Exercise 5.3.2} \label{exercise5.3.2}
\begin{proof}
(1)
We need to show that $(a_nb_n)^\infty_{n=1}$ is a Cauchy sequence. For any $\varepsilon>0$, we can find $M,N$ such that 
$\forall i,j\geq M(|a_i-a_j| \leq \varepsilon)$ and $\forall i,j\geq N(|b_i-b_j| \leq \varepsilon)$. Let $B = \max{(M,N)}$. Then for 
$i,j \geq B$, (See (h) in Proposition 4.3.7)
\[
|a_ib_i - a_jb_j| \leq \varepsilon(|a_i| + |a_j|) + \varepsilon^2
\]
Note that $(a_n)^\infty_{n=1}$ is a Cauchy sequence, so it is bounded by some number $M$. Thus, 
$|a_i| + |a_j| \leq 2M$. 

For any $\varepsilon' >0$, we need to find a $\varepsilon >0$ such that 
$\varepsilon(|a_i| + |a_j|) + \varepsilon^2 \leq \varepsilon'$. First, if $\varepsilon' \geq 1$, then by 
setting $\varepsilon < 1$ we can obtain $\varepsilon^2 < 1 \leq \varepsilon'$; if $\varepsilon' < 1$, then we let $\varepsilon < \varepsilon'$, and multiply it with $\varepsilon < 1$, we then have 
$\varepsilon^2 < \varepsilon'$. After these steps, we can ensure that $\varepsilon' - \varepsilon^2>0$. 

Consider the number $t = \dfrac{\varepsilon' - \varepsilon^2}{2M}>0$. If $\varepsilon$ already satisfies 
$\varepsilon <t$, then it is the number we want. If it doesn't, then we can shrink it. That is, let
$\varepsilon'' < t \leq \varepsilon$. $\varepsilon'' < \varepsilon$ gives 
$(\varepsilon'')^2 < (\varepsilon)^2$, then $-(\varepsilon'')^2 > -(\varepsilon)^2$, and finally 
\[
t'' = \frac{\varepsilon' - (\varepsilon'')^2}{2M}>\frac{\varepsilon' - \varepsilon^2}{2M}
\]
So $\varepsilon'' < t < t''$. We can set $\varepsilon$ to this $\varepsilon''$. Then 
\begin{align*}
\varepsilon < \frac{\varepsilon' - \varepsilon^2}{2M} &\Longrightarrow \\
\varepsilon \times 2M < \varepsilon' - \varepsilon^2 &\Longrightarrow \\
\varepsilon \times 2M + \varepsilon^2 < \varepsilon'
\end{align*}

So no matter what $\varepsilon'>0$ is, we can always find $\varepsilon >0$ such that 
\[
\varepsilon(|a_i| + |a_j|) + \varepsilon^2 \leq \varepsilon'
\].
And for this $\varepsilon'$, there exists $N\geq 1$ such that 
\[
\forall i,j\geq N(|a_ib_i - a_jb_j| \leq \varepsilon(|a_i| + |a_j|) + \varepsilon^2 \leq \varepsilon')
\]
Then, $(a_nb_n)^\infty_{n=1}$ is a Cauchy sequence. So is $xy$ a real number.

(2)
For any $\varepsilon >0$, we can find $N$ such that $\forall n\geq N(|a_n-a'_n|\leq \varepsilon)$. Thus, for 
such $n$, 
\[
|a_nb_n - a'_nb_n| = |b_n||a_n - a'_n| \leq \varepsilon|b_n|
\]
Note that $(b_n)^\infty_{n=1}$ is bounded by some number $M$. So $|a_nb_n - a'_nb_n| \leq \varepsilon M$. 
Therefore, we find the $N'$ such that $\forall n\geq N'(|a_n-a'_n|\leq \varepsilon/M)$. Then for such $n$, 
$|a_nb_n - a'_nb_n| \leq \varepsilon$. Thus $\sequence{a_nb_n}{1} = \sequence{a'_nb_n}{1}$.
\end{proof}

\paragraph{Exercise 5.3.3} \label{exercise5.3.3}
\begin{proof}
On one hand, if $a=b$, then obviously $a,a,\cdots = b,b,\cdots$. 

On the other hand, if $a,a,\cdots \neq b,b,\cdots$, we try to show that $a=b$. Presume the negation, that is, 
$a \neq b$. Then, $|a_n-b_n| = |a-b| \geq |a-b|$. For any $0<\varepsilon< |a-b|$, the two Cauchy sequences cannot be 
$\varepsilon-$close, which is impossible.
\end{proof}

\paragraph{Lemma 5.3.14}
Here it is asked that why can we deduce $|b_n| \geq \varepsilon/2$ from $|b_{n0} - b_n| \leq \varepsilon/2$ and 
$|b_{n0} > \varepsilon$. The book says that the triangle inequality can be used. In fact, we use the fact 
\[
||b_{n0}| - |b_n|| \leq |b_{n0} - b_n|
\]
instead of $|b_{n0} - b_n| \leq |b_{n0}| + |b_n|$. Since that $|b_{n0}| - |b_n| \leq ||b_{n0}| - |b_n||$, 
we have 
\[
|b_{n0}| - |b_n| \leq \varepsilon/2 \equiv |b_{n0}| \leq \varepsilon/2 + |b_n|
\]
But $|b_{n0}| > \varepsilon$, so $\varepsilon/2 + |b_n| > \varepsilon \equiv |b_n| > \varepsilon/2$.

It is not mentioned in (b) of Proposition 4.3.3, but it can be easily proven if we divide conditions, though 
the process is indeed very tedious.

\paragraph{Exercise 5.3.4} \label{exercise5.3.4}
\begin{proof}
Since that the two Cauchy sequences are equivalent, they are eventually $\varepsilon-$steadiness for any $\varepsilon>0$. 
Then, according to \exerciseref{5.2.2}, $(a_n)^\infty_{n=1}$ being bounded implies that $(b_n)^\infty_{n=1}$ being so.
\end{proof}

\paragraph{Exercise 5.3.5} \label{exercise5.3.5}
\begin{proof}
We show that $(\frac{1}{n})^\infty_{n=1} = (0)^\infty_{n=1}$.

For each $\varepsilon>0$, we want to find $N \in \mathbb{N}$ such that 
$n\geq N \longrightarrow |\frac{1}{n}-0|\leq \varepsilon$. Note that 
\[
|\frac{1}{n}-0|\leq \varepsilon \equiv \frac{1}{n} \leq \varepsilon \equiv \frac{1}{\varepsilon} \leq n
\], 
which means that we need to find $N \geq \frac{1}{\varepsilon}$. This is always possible as stated by Proposition 4.4.1.

Then the two sequences are equivalent, which proves our proposition.
\end{proof}

\section{Ordering the Reals}
\paragraph{Exercise 5.4.1} \label{exercise5.4.1}
\begin{proof}
We try to show that if a real number $a$ is non-zero, then it must be either positive or negative (not both). We already know from 
Lemma 5.3.14 that $a$ can equal to $\LIM{a_n}$, where $\sequence{a_n}{1}$ is bounded away from zero. Now we show that every Cauchy 
sequence that is bounded away from zero can always equal to either a positively bounded away from zero or a negatively bounded away 
from zero Cauchy sequence.

Let $\sequence{a_n}{1}$ be a Cauchy sequence that is bounded away from zero. Then for every $n$, $|a_n| \geq c$. Choose $\varepsilon$ 
so that $0<\varepsilon<c$. We can find $N$ such that $m,n \geq N \longrightarrow |a_n-a_m| \leq \varepsilon$. We know that 
$|a_n| \geq c \longrightarrow a_n\leq -c \vee a_n \geq c$. We will only show that $\sequence{a_n}{1}$ equals to some sequences 
positively bounded away from zero on the latter condition. It is easy to derive that $\sequence{a_n}{1}$ equals to some sequences 
negatively bounded away from zero on the former condition.

We have
\[
|a_n-a_m| \leq \varepsilon \longrightarrow a_n -\varepsilon <a_m
\]
and hence $a_m >c-\varepsilon>0$ since $\varepsilon<c \wedge a_n \geq c$.

Let $\sequence{b_n}{1}$ be such a sequence that
\begin{itemize}
\item $m \geq N \longrightarrow b_n = a_n$,
\item $0<n<N \longrightarrow b_n$ be any value bigger than $c-\varepsilon$.
\end{itemize}

Thus, $\sequence{b_n}{1}$ is positively bounded away from zero, and is also equivalent to $\sequence{a_n}{1}$ since that 
$m \geq N \longrightarrow b_n = a_n$.

Now we show that it cannot equal to both. Denote it by $a_n$. Presume the negation, that is, it is equal to both a sequence 
$x_n \geq c$ and $y_n \leq -c$ for $c>0$. Choose $\varepsilon$ so that $0<\varepsilon<c$. It equals to $x_n$ implies that we can always 
find $N_1$ such that 
\[
n \geq N \longrightarrow |a_n-x_n| \leq \varepsilon \longrightarrow a_n > x_n -\varepsilon >c-\varepsilon >0
\]
Similarly, we can find $N_2$ such that
\[
a_n < -(c-\varepsilon)<0
\]

This is impossible, as $a_n$ cannot be eventually $2(c-\varepsilon)$-close.

From what we have shown, we can easily derive that a real number is either positive, negative, or zero.

Now we show that $x$ is positive iff $-x$ is negative. We know $x = \LIM{a_n}$, where $a_n > c>0$. Then 
$-x = \LIM{-a_n}$. Since that $-a_n < -c <0$, $\sequence{-a_n}{1}$ is negatively bounded away from zero. So $-x$ is 
negative, as desired.

Finally we show that if $x=\LIM{a_n},y=\LIM{b_n}$ are both positive, then $x+y=\LIM{a_n+b_n},xy = \LIM{a_nb_n}$ is also 
positive. It is immediately leaded to since 
\[
a_n >c>0 \wedge b_n >d>0 \Longrightarrow a_nb_n > cd>0 \wedge a_n+b_n > c+d>0
\]
\end{proof}

\paragraph{Exercise 5.4.2} \label{exercise5.4.2}
\begin{proof}
(a) It is immediately derived since $x-y$ satisfies Proposition 5.4.4.

(b)
Denote $x,y$ as $\LIM{a_n},\LIM{b_n}$, respectively. Note that by definition 
$y-x = \LIM{b_n-a_n} = \LIM{-(a_n-b_n)} = -(x-y)$. So from Proposition 5.4.4 we can see that
\[
x>y \equiv x-y>0 \equiv y-x<0 \equiv y<x
\]
One might notice that we use $x-y>0$ to represent $x-y$ being positive here. It is quite easy to prove. Just notice that 
$x$ being positive is logically equivalent to $x-0$ being so, and thus is to $x>0$. We can further prove that $x<0$ is 
equivalent to $x$ being negative.

(c)
We know by Proposition 5.4.4 that 
\[
x-y >0 \wedge y-z>0 \longrightarrow (x-y)+(y-z) = x-z >0
\]
So $x>y \wedge y>z \longrightarrow x>z$.

(d)
It is immediately deduced as $(x+z)-(y+z) = x-y$.

(e)
$x>y \equiv x-y>0$. As stated by Proposition 5.4.4, $z(x-y) >0$. So $xz-yz>0 \longrightarrow xz>yz$.

\end{proof}

\begin{prop} \label{prop.5.4.basicproperties}
Since that reals numbers possess the same basic algebraic properties as the properties possessed by rational numbers, we 
can ascertain that the corollaries of them are also right. For example,
\[
a<b\wedge c<d \longrightarrow a+c<b+d
\]
\[
a,b,c,d>0\wedge a<b\wedge c<d \longrightarrow ac<bd
\]
\end{prop}

\paragraph{Exercise 5.4.3} \label{exercise5.4.3}
\begin{proof}
Existence:
For a $\varepsilon>0$, there exists a $N$ such that $n\geq N \longrightarrow |a_n-a_N| \leq \varepsilon$. Choose an 
arbitrary $c>0$, and let a rational number $y$ be $\LIM{a_N-\varepsilon-c}$. On this occasion, the real number 
$y-x<0$, which means that $y<x$. 

Since $y$ is a rational number, there exists a natural number $M$ such that $M \leq y$. So $M<x$ the number $M+1$ may be 
bigger than $x$, and if it is, $M$ is the number we want.

If $M+1 \leq x$, then we check if $(M+1)+1$ is bigger than $x$, and we repeat the step until we find the first natural 
number $M'$ such that $M'+1>x$. Hence $M'$ is the number we want.

Uniqueness:
We have already shown that $\exists M(M\leq x<M+1)$.
Suppose that there exists another natural number $K$ such that $K\leq x<K+1$. We show that $K=M$. Suppose the negation. 
Then $K$ either $<M$ or $>M$. Under the former condition, $K+1\leq M\leq x$, which is impossible. Under the latter 
condition, $x<M+1\leq K$, which is also impossible.
\end{proof}

\paragraph{Exercise 5.4.4} \label{exercise5.4.4}
\begin{proof}
\[
x>0\rightarrow \frac{1}{x}>0\rightarrow \exists N(N>\frac{1}{x}>0)
\]
So, according to Proposition 5.4.8,
\[
0<\frac{1}{N}<x
\]
, as desired.
\end{proof}

\paragraph{Exercise 5.4.5} \label{exercise5.4.5}
\begin{proof}
Let $x=\LIM{x_n},y=\LIM{y_n}$. Since $x>y$, the sequence $\sequence{x_n-y_n}{1}$ equals to a sequence that is positively 
bounded 
away from zero. We may just let $\sequence{x_n-y_n}{1}$ be such a sequence. (This is always possible. Given a sequence 
$\sequence{z_n}{1}$, which is positively bounded away from zero, and satisfies $\LIM{z_n}=x-y$, we can define $x_n$ as 
$y_n+z_n$) This way, for some $c>0$, $x_n-y_n>c\equiv x_n>c+y_n$. 

Moreover, we can find $N$ such that for $0<\varepsilon<\frac{c}{2}$ and $n\geq N$, 
$|x_n-x_N|\leq \varepsilon\wedge|y_n-y_N|\leq \varepsilon$, which means
\[
x_n\geq x_N -\varepsilon\wedge y_n \leq y_N+\varepsilon
\]
Adding $c>2\varepsilon$ to the inequality $x_N \geq y_N+c$ gives
\[
x_N>y_N+2\varepsilon \equiv x_N-\varepsilon>y_N+\varepsilon
\]

This simplifies our work because both $x_N-\varepsilon$ and $y_N+\varepsilon$ are rational numbers, and we know that 
there exists a rational number $q$ such that $x_N-\varepsilon<q<y_N+\varepsilon$. We now know that for $n\geq N$,
\[
x_n \geq x_N -\varepsilon \longrightarrow x_n-q \geq x_N-\varepsilon-q>0
\]
and
\[
y_n \leq y_N +\varepsilon \longrightarrow q-y_n \geq q-(y_N+\varepsilon)>0
\]
Define a new sequence as the following: $x_n'=x_n$ if $n\geq N$, and $x_n'$ be any rational number such that $
|x_n'-x_N|\leq \varepsilon$, and define $y_n'$ in the same way. Obviously $x=\LIM{x_n'},y=\LIM{y_n'}$. And the sequences
$\sequence{x_n'-q}{1},\sequence{q-y_n'}{1}$ are both positively bounded away from zero. Hence $x<q<y$, as desired.
\end{proof}

\paragraph{Exercise 5.4.6} \label{exercise5.4.6}
\begin{proof}
On one hand, suppose that $|x-y|<\varepsilon$. If $x=y$, then $y-\varepsilon<x<y+\varepsilon$ is obvious. If $x>y$, then 
$|x-y|=x-y<\varepsilon \rightarrow x<y+\varepsilon$, and $x>y \rightarrow x>y-\varepsilon$. If $x<y$, then 
$x<y+\varepsilon$ and $|x-y|=y-x<\varepsilon\rightarrow x>y-\varepsilon$.

On the other hand, conversely, suppose that 
\[
y-\varepsilon<x<y+\varepsilon \rightarrow x-y<\varepsilon \wedge y-x<\varepsilon
\]. If $x>y$, then $|x-y|=x-y<\varepsilon$. If $x=y$, then $|x-y|=0<\varepsilon$. If $x<y$, then 
$|x-y|=y-x<\varepsilon$.
\end{proof}

\paragraph{Exercise 5.4.7} \label{exercise5.4.7}
\begin{proof}
(a)
On one hand, if $x\geq y$, then add $0<\varepsilon$ to it (See Proposition \ref{prop.5.4.basicproperties}), we have 
$x \geq y+\varepsilon$.

On the other hand, conversely, if for all real number $\varepsilon>0$, $x \leq y+\varepsilon$, we show that $x\leq y$. 
Presume the negation, that is, $x>y$, then as stated by Proposition 5.4.14, we can find $q$ such that $x>q>y$, so 
$x>y+(q-y)$, a contradiction.

(b)
On one hand, suppose that $|x-y|\leq \varepsilon$ for all $\varepsilon>0$. If $x\geq y$, then 
$|x-y|=x-y \leq \varepsilon$, so we have $x\leq y$ by (a). If $x\leq y$, we can conclude that $x\geq y$ as well. So 
either way we will have $x=y$.

On the other hand, if $x=y$, then $|x-y|=0<\varepsilon$.
\end{proof}

\paragraph{Exercise 5.4.8} \label{exercise5.4.8}
\begin{proof}
We shall just prove the first one here.

Presume the negation, that is, $\LIM{a_n}>x$. Then we can find a rational $q$ such that $x<q<\LIM{a_n}$
(Proposition 5.4.14). Thus
$a_n\leq x<q$,
and then $\LIM{a_n}\leq\LIM{q}=q$ (Corollary 5.4.10). But we have $\LIM{a_n}>q$, a contradiction.
\end{proof}

\section{The Least Upper Bound Property}

\paragraph{Example 5.5.3}
This set has no upper bound because
\begin{itemize}
\item If $x$ is greater than an element in the set, then $x$ must also be an element in the set;
\item $\forall x \in \mathbb{R}^+(\exists y(y \in \mathbb{R}^+ \wedge y>x))$.
\end{itemize}
\begin{proof}
(1)
This is immediately derived from the fact that order is transitive (Proposition 5.4.7).

(2)
Consider the number $x+1$, which, according to (1), is also in the set, and which, is greater than $x$ as $y-x = 1>0$.
\end{proof}

\paragraph{Exercise 5.5.1} \label{exercise5.5.1}
\begin{proof}
First we show that every upper bound of $E$, $N$, is a lower bound for $-E$, and vice versa. This can be concluded from
\[
x \leq N \equiv -x \geq -N
\]
Then we show that $-M$ is the greatest lower bound. In fact,
\[
M \leq N \to -M \geq -N
\]
\end{proof}

\paragraph{Exercise 5.5.2} \label{exercise5.5.2}
\begin{proof}
We show that for all $L<m\leq K$, there at least exists one $m_0$ such that we can not have both $\frac{m_0}{n}$ and $\frac{m_0-1}{n}$ being upper bounds and they not being upper bounds. That is, for this $m_0$, either 
\[
\frac{m_0}{n} \text{ is an upper bound} \wedge \frac{m_0-1}{n} \text{ is not}
\]
, which is impossible as $\frac{m_0}{n}>\frac{m_0-1}{n}$, or
\[
\frac{m_0-1}{n} \text{ is an upper bound} \wedge \frac{m_0}{n} \text{ is not}.
\]

Presume the negation, that is, (Note that this statement includes the situation when $\frac{m_0}{n}$ and $\frac{m_0-1}{n}$ are both not upper bounds)
\[
(\forall L<m\leq K)(\frac{m_0}{n} \text{ is an upper bound} \equiv \frac{m_0-1}{n} \text{ is an upper bound})
\]
We know that when $m=K$, $m/n$ is an upper bound. Then $(m-1)/n$ also is. Repeat the step until $m=L+1$, then we can 
conclude that $\frac{m-1=L}{n}$ is an upper bound, a contradiction.
\end{proof}

\paragraph{Exercise 5.5.3} \label{exercise5.5.3}
\begin{proof}
\emph{This is a different approach.}

On one hand, for all $x<m_n \to x \leq m_n-1$, $x/n\leq (m_n-1)/n$, which means such $x/n$ cannot be upper bounds. On the other hand, 
for all $x>m_n \to x-1\geq m_n$, $(x-1)/n\geq m_n/n$.
\end{proof}

\paragraph{Exercise 5.5.4} \label{exercise5.5.4}
\begin{proof}
(1) Simply take $M \geq \varepsilon$, then $|m_n/n-m_{n'}/n'|\leq 1/M\leq \varepsilon$.

(2)
Since we want to use \exerciseref{5.4.8}, we need to prove that for some $N$, $(\forall n \geq N, q_n < q_M) \vee (\forall n \geq N, q_n > q_M)$. This thought introduces the following lemma:
\begin{lem}
For any Cauchy sequence $a_n$, if $a_M \neq \LIM{a_n}$, then there exists a $N$ such that for all $n$ greater than or equal to it, $a_n$ are either all lesser than $a_M$ or all greater than $a_M$.
\end{lem}
\begin{proof}
Since $q_M \neq \LIM{a_n}$,
\[
\neg (\forall \varepsilon>0(\exists N(n\geq N \to |a_n-a_M| \leq \varepsilon) ))
\]
, which is
\[
\exists \varepsilon >0 (\forall N(\exists n\geq N(|a_n-a_M| > \varepsilon)))
\]
For this $\varepsilon$, $\exists N_1(n,n'\geq N_1 \to |a_n-a_{n'}| \leq \varepsilon)$. But we know $\exists N_2 \geq N_1(|a_{N_2}-a_M| > \varepsilon)$. So for all $n \geq N_2$, we have
\[
|a_n-a_{N_2}| \leq \varepsilon \wedge |a_{N_2} - a_M| > \varepsilon
\]
This means either $a_n < q_{M}$ (when $a_M > a_{N_2}$), or $a_n > a_{M}$ (when $a_M < a_{N_2}$).
\end{proof}

If $q_M = S$, then $|q_M-S| = 0 \leq \frac{1}{M}$.

If $q_M \neq S$, then according to the previous lemma, for some $N$ and all $n\geq N$, $q_n > q_M$ or $q_n<q_M$. We suppose that $q_n > q_M$ here. We also know that for all $n\geq M$, $|q_M - q_n| \leq \frac{1}{M}$. Then for $n \geq \max(N,M)$, we can say that $\frac{1}{M} \leq q_n - q_M >0$. The limit of the sequence $(q_n-q_M)_1^n$ is $S-q_M$ (To really make it a sequence, we define its value to be any number between $\frac{1}{M}$ and 0 when $n < \max(N,M)$). We can now apply \exerciseref{5.4.8} to it to obtain $\frac{1}{M}\leq S-q_M >0$ (Note that $S \neq q_M$ is our hypothesis.) A similar process produces $\frac{1}{M}\leq q_M-S >0$ when $q_n<q_M$. 

We can now finish the proof.
\end{proof}

\paragraph{Exercise 5.5.5} \label{exercise5.5.5}
\begin{proof}
We will prove that there exists an irrational number $0<i<y-x$ and then $x+i$ automatically becomes the number we want.

Then we only need to prove that irrational numbers can be arbitrarily small. So we think of $\frac{\sqrt{2}}{N}$, which becomes smaller than any given $\varepsilon>0$ as $N$ approaches to infinite.

But first we need to prove that it is an irrational number. Presume the negation, that is, it is rational. We know that rationals are closed under multiplication, which means $\frac{\sqrt{2}}{N} \cdot N = \sqrt{2}$ is rational, a contradiction.
\end{proof}